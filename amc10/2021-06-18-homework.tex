\documentclass[12pt]{article}
\newif\ifanswer\answertrue\answerfalse% comment out \answertrue to show/hide answers
\usepackage{../preamble3}% preamble always after \newif\ifanswer
%\pagenumbering{gobble}
\title{Art Of Problem Solving - AMC 10 \\ June 11, 2021}
\author{Patrick \& James Toche}
\date{Revised:~\today}

\begin{document}
\maketitle
\begin{minipage}{\textwidth}
\begin{abstract}\setlength{\parindent}{0pt}%
Notes on the AMC-10 Course by Art Of Problem Solving (AOPS).
Copyright restrictions may apply. Written for personal use. 
Please report typos and errors over at \url{https://github.com/ptoche/Math/tree/master/aops}. 
\end{abstract}
\end{minipage}

\thispagestyle{empty}
\clearpage


%%%%%%%%%%%%%%%%%%%%%%%%%%%%%%%%%%%%%%%%%%%%%%%%%%%%%%%%%%%%%%%%%%%%%%%%
\subsection*{1.}

\nopagebreak

Yan is somewhere between his home and the stadium. To get to the stadium he can walk directly to the stadium, or else he can walk home and then ride his bicycle to the stadium. He rides 7 times as fast as he walks, and both choices require the same amount of time. What is the ratio of Yan's distance from his home to his distance from the stadium?

\fbox{$\dfrac{2}{3}$, \quad $\dfrac{3}{4}$, \quad $\dfrac{4}{5}$, \quad $\dfrac{5}{6}$, \quad $\dfrac{6}{7}$}

\begin{answer}
\begin{align*}
x
\end{align*}
\begin{empheq}[box={\mathbox[colback=white]}]{equation*}
    x
\end{empheq} 
\end{answer}
%%%%%%%%%%%%%%%%%%%%%%%%%%%%%%%%%%%%%%%%%%%%%%%%%%%%%%%%%%%%%%%%%%%%%%%%

\iftoggle{showAnswers}{\newpage}

%%%%%%%%%%%%%%%%%%%%%%%%%%%%%%%%%%%%%%%%%%%%%%%%%%%%%%%%%%%%%%%%%%%%%%%%
\subsection*{2.}

\nopagebreak

Angelina drove at an average rate of 80 kph and then stopped 20 minutes for gas. After the stop, she drove at an average rate of 100 kph. Altogether she drove 250 km in a total trip time of 3 hours including the stop. Which equation could be used to solve for the time $t$ in hours that she drove before her stop?

\fbox{$80t+100(8/3-t)$, \quad $80t=250$, \quad $100t=250$, \quad $90t=250$, \quad $80(8/3-t)+100t=250$}

\begin{answer}
\begin{align*}
x
\end{align*}
\begin{empheq}[box={\mathbox[colback=white]}]{equation*}
    x
\end{empheq} 
\end{answer}
%%%%%%%%%%%%%%%%%%%%%%%%%%%%%%%%%%%%%%%%%%%%%%%%%%%%%%%%%%%%%%%%%%%%%%%%

\iftoggle{showAnswers}{\newpage}

%%%%%%%%%%%%%%%%%%%%%%%%%%%%%%%%%%%%%%%%%%%%%%%%%%%%%%%%%%%%%%%%%%%%%%%%
\subsection*{3.}

\nopagebreak

When a bucket is two-thirds full of water, the bucket and water weigh $a$ kilograms. When the bucket is one-half full of water the total weight is $b$ kilograms. In terms of $a$ and $b$, what is the total weight in kilograms when the bucket is full of water?

\fbox{$\dfrac{2}{3}a+\dfrac{1}{3}b$, \quad $\dfrac{3}{2}a-\dfrac{1}{2}a$, \quad $\dfrac{3}{2}a+b$, \quad $\dfrac{3}{2}a+2b$, \quad $3a-2b$}

\begin{answer}
\begin{align*}
x
\end{align*}
\begin{empheq}[box={\mathbox[colback=white]}]{equation*}
    x
\end{empheq} 
\end{answer}
%%%%%%%%%%%%%%%%%%%%%%%%%%%%%%%%%%%%%%%%%%%%%%%%%%%%%%%%%%%%%%%%%%%%%%%%

\iftoggle{showAnswers}{\newpage}

%%%%%%%%%%%%%%%%%%%%%%%%%%%%%%%%%%%%%%%%%%%%%%%%%%%%%%%%%%%%%%%%%%%%%%%%
\subsection*{4.}

\nopagebreak

On a 50-question multiple choice math contest, students receive 4 points for a correct answer, 0 points for an answer left blank, and $-1$ point for an incorrect answer. Jesse's total score on the contest was 99. What is the maximum number of questions that Jesse could have answered correctly?

\fbox{$25$, \quad $27$, \quad $29$, \quad $31$, \quad $33$}

\begin{answer}
\begin{align*}
x
\end{align*}
\begin{empheq}[box={\mathbox[colback=white]}]{equation*}
    x
\end{empheq} 
\end{answer}
%%%%%%%%%%%%%%%%%%%%%%%%%%%%%%%%%%%%%%%%%%%%%%%%%%%%%%%%%%%%%%%%%%%%%%%%

\iftoggle{showAnswers}{\newpage}

%%%%%%%%%%%%%%%%%%%%%%%%%%%%%%%%%%%%%%%%%%%%%%%%%%%%%%%%%%%%%%%%%%%%%%%%
\subsection*{5.}

\nopagebreak

Roy bought a new battery-gasoline hybrid car. On a trip the car ran exclusively on its battery for the first 40 miles, then ran exclusively on gasoline for the rest of the trip, using gasoline at a rate of 0.02 gallons per mile. On the whole trip he averaged 55 miles per gallon. How long was the trip in miles?

\fbox{$140$, \quad $240$, \quad $440$, \quad $640$, \quad $840$}

\begin{answer}
\begin{align*}
x
\end{align*}
\begin{empheq}[box={\mathbox[colback=white]}]{equation*}
    x
\end{empheq} 
\end{answer}
%%%%%%%%%%%%%%%%%%%%%%%%%%%%%%%%%%%%%%%%%%%%%%%%%%%%%%%%%%%%%%%%%%%%%%%%

\iftoggle{showAnswers}{\newpage}

%%%%%%%%%%%%%%%%%%%%%%%%%%%%%%%%%%%%%%%%%%%%%%%%%%%%%%%%%%%%%%%%%%%%%%%%
\subsection*{6.}

\nopagebreak

Sarah pours four ounces of coffee into an eight-ounce cup and four ounces of cream into a second cup of the same size. She then transfers half the coffee from the first cup to the second and, after stirring thoroughly, transfers half the liquid in the second cup back to the first. What fraction of the liquid in the first cup is now cream?

\fbox{$1/4$, \quad $1/3$, \quad $3/8$, \quad $2/5$, \quad $1/2$}

\begin{answer}
\begin{align*}
x
\end{align*}
\begin{empheq}[box={\mathbox[colback=white]}]{equation*}
    x
\end{empheq} 
\end{answer}
%%%%%%%%%%%%%%%%%%%%%%%%%%%%%%%%%%%%%%%%%%%%%%%%%%%%%%%%%%%%%%%%%%%%%%%%

\iftoggle{showAnswers}{\newpage}

%%%%%%%%%%%%%%%%%%%%%%%%%%%%%%%%%%%%%%%%%%%%%%%%%%%%%%%%%%%%%%%%%%%%%%%%
\subsection*{7.}

\nopagebreak

Andrea and Lauren are 20 kilometers apart. They bike toward one another with Andrea traveling three times as fast as Lauren, and the distance between them decreasing at a rate of 1 kilometer per minute. After 5 minutes, Andrea stops biking because of a flat tire and waits for Lauren. After how many minutes from the time they started to bike does Lauren reach Andrea?

\fbox{$20$, \quad $30$, \quad $55$, \quad $65$, \quad $80$}

\begin{answer}
\begin{align*}
x
\end{align*}
\begin{empheq}[box={\mathbox[colback=white]}]{equation*}
    x
\end{empheq} 
\end{answer}
%%%%%%%%%%%%%%%%%%%%%%%%%%%%%%%%%%%%%%%%%%%%%%%%%%%%%%%%%%%%%%%%%%%%%%%%

\iftoggle{showAnswers}{\newpage}

%%%%%%%%%%%%%%%%%%%%%%%%%%%%%%%%%%%%%%%%%%%%%%%%%%%%%%%%%%%%%%%%%%%%%%%%
\subsection*{8.}

\nopagebreak

Andy's lawn has twice as much area as Beth's lawn and three times as much area as Carlos' lawn. Carlos' lawn mower cuts half as fast as Beth's mower and one third as fast as Andy's mower. If they all start to mow their lawns at the same time, who will finish first?

\fbox{Andy, \quad Beth, \quad Carlos, \quad Andy and Carlos tie for first, \quad All three tie.}

\begin{answer}
\begin{align*}
x
\end{align*}
\begin{empheq}[box={\mathbox[colback=white]}]{equation*}
    x
\end{empheq} 
\end{answer}
%%%%%%%%%%%%%%%%%%%%%%%%%%%%%%%%%%%%%%%%%%%%%%%%%%%%%%%%%%%%%%%%%%%%%%%%

\iftoggle{showAnswers}{\newpage}

%%%%%%%%%%%%%%%%%%%%%%%%%%%%%%%%%%%%%%%%%%%%%%%%%%%%%%%%%%%%%%%%%%%%%%%%
\subsection*{9.}

\nopagebreak

It takes $A$ algebra books (all the same thickness) and $H$ geometry books (all the same thickness, which is greater than that of an algebra book) to completely fill a certain shelf. Also, $S$ of the algebra books and $M$ of the geometry books would fill the same shelf. Finally, $E$ of the algebra books alone would fill this shelf. Given that $A$, $H$, $S$, $M$, and $E$ are distinct positive integers, it follows that $E$ is

\fbox{$\dfrac{AM+SH}{M+H}$, \quad $\dfrac{AM^2+SH^2}{M^2+H^2}$, \quad $\dfrac{AH-SM}{M-H}$, \quad $\dfrac{AM-SH}{M-H}$, \quad $\dfrac{AM^2-SH^2}{M^2-H^2}$}

\begin{answer}
\begin{align*}
x
\end{align*}
\begin{empheq}[box={\mathbox[colback=white]}]{equation*}
    x
\end{empheq} 
\end{answer}
%%%%%%%%%%%%%%%%%%%%%%%%%%%%%%%%%%%%%%%%%%%%%%%%%%%%%%%%%%%%%%%%%%%%%%%%

\iftoggle{showAnswers}{\newpage}

%%%%%%%%%%%%%%%%%%%%%%%%%%%%%%%%%%%%%%%%%%%%%%%%%%%%%%%%%%%%%%%%%%%%%%%%
\subsection*{10.}

\nopagebreak

One morning each member of Angela's family drank an 8-ounce mixture of coffee with milk. The amounts of coffee and milk varied from cup to cup, but were never zero. Angela drank a quarter of the total amount of milk and a sixth of the total amount of coffee. How many people are in the family?

\fbox{$3$, \quad $4$, \quad $5$, \quad $6$, \quad $7$}

\begin{answer}
\begin{align*}
x
\end{align*}
\begin{empheq}[box={\mathbox[colback=white]}]{equation*}
    x
\end{empheq} 
\end{answer}
%%%%%%%%%%%%%%%%%%%%%%%%%%%%%%%%%%%%%%%%%%%%%%%%%%%%%%%%%%%%%%%%%%%%%%%%



\end{document}
