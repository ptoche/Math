\documentclass[12pt]{article}
\newif\ifanswer\answertrue\answerfalse% comment out \answertrue to show/hide answers
\usepackage{../preamble3}% preamble always after \newif\ifanswer
%\pagenumbering{gobble}
\title{Art Of Problem Solving - AMC 10 \\ June 25, 2021}
\author{Patrick \& James Toche}
\date{Revised:~\today}

\begin{document}
\maketitle
\begin{minipage}{\textwidth}
\begin{abstract}\setlength{\parindent}{0pt}%
Notes on the AMC-10 Course by Art Of Problem Solving (AOPS).
Copyright restrictions may apply. Written for personal use. 
Please report typos and errors over at \url{https://github.com/ptoche/Math/tree/master/aops}. 
\end{abstract}
\end{minipage}

\thispagestyle{empty}
\clearpage


%%%%%%%%%%%%%%%%%%%%%%%%%%%%%%%%%%%%%%%%%%%%%%%%%%%%%%%%%%%%%%%%%%%%%%%%
\subsection*{1.}

\nopagebreak

Consider the set of numbers $\{1, 10, 10^2, 10^3, \dots, 10^{10}\}$. The ratio of the largest element of the set to the sum of the other ten elements of the set is closest to which integer?

\fbox{(A) 1 \quad (B) 9 \quad (C) 10 \quad (D) 11 \quad (E) 101}

\begin{answer}
\begin{align*}
x
\end{align*}
\begin{empheq}[box={\mathbox[colback=white]}]{equation*}
    x
\end{empheq} 
\end{answer}
%%%%%%%%%%%%%%%%%%%%%%%%%%%%%%%%%%%%%%%%%%%%%%%%%%%%%%%%%%%%%%%%%%%%%%%%

\iftoggle{showAnswers}{\newpage}

%%%%%%%%%%%%%%%%%%%%%%%%%%%%%%%%%%%%%%%%%%%%%%%%%%%%%%%%%%%%%%%%%%%%%%%%
\subsection*{2.}

\nopagebreak

For each positive integer $n$, the mean of the first $n$ terms of a sequence is $n$. What is the $2008^{\text{th}}$ term of the sequence?

\fbox{(A) 2008 \quad (B) 4015 \quad (C) 4016 \quad (D) 4,030,056 \quad (E) 4,032,064}

\begin{answer}
\begin{align*}
x
\end{align*}
\begin{empheq}[box={\mathbox[colback=white]}]{equation*}
    x
\end{empheq} 
\end{answer}
%%%%%%%%%%%%%%%%%%%%%%%%%%%%%%%%%%%%%%%%%%%%%%%%%%%%%%%%%%%%%%%%%%%%%%%%

\iftoggle{showAnswers}{\newpage}


%%%%%%%%%%%%%%%%%%%%%%%%%%%%%%%%%%%%%%%%%%%%%%%%%%%%%%%%%%%%%%%%%%%%%%%%
\subsection*{3.}

\nopagebreak

On Monday, Millie puts a quart of seeds, $25\%$ of which are millet, into a bird feeder. On each successive day she adds another quart of the same mix of seeds without removing any seeds that are left. Each day the birds eat only $25\%$ of the millet in the feeder, but they eat all of the other seeds. On which day, just after Millie has placed the seeds, will the birds find that more than half the seeds in the feeder are millet?

\fbox{(A) Tuesday (B) Wednesday (C) Thursday (D) Friday (E) Saturday}


\begin{answer}
\begin{align*}
x
\end{align*}
\begin{empheq}[box={\mathbox[colback=white]}]{equation*}
    x
\end{empheq} 
\end{answer}
%%%%%%%%%%%%%%%%%%%%%%%%%%%%%%%%%%%%%%%%%%%%%%%%%%%%%%%%%%%%%%%%%%%%%%%%

\iftoggle{showAnswers}{\newpage}


%%%%%%%%%%%%%%%%%%%%%%%%%%%%%%%%%%%%%%%%%%%%%%%%%%%%%%%%%%%%%%%%%%%%%%%%
\subsection*{4.}

\nopagebreak

In the five-sided star shown, the letters $A$, $B$, $C$, $D$, and $E$ are replaced by the numbers 3, 5, 6, 7, and 9, although not necessarily in this order. The sums of the numbers at the ends of the line segments $\overline{AB}$, $\overline{BC}$, $\overline{CD}$, $\overline{DE}$, and $\overline{EA}$ form an arithmetic sequence, although not necessarily in this order. What is the middle term of the arithmetic sequence?

\fbox{(A) 9 \quad (B) 10 \quad (C) 11 \quad (D) 12 \quad (E) 13}

\begin{answer}
\begin{align*}
x
\end{align*}
\begin{empheq}[box={\mathbox[colback=white]}]{equation*}
    x
\end{empheq} 
\end{answer}
%%%%%%%%%%%%%%%%%%%%%%%%%%%%%%%%%%%%%%%%%%%%%%%%%%%%%%%%%%%%%%%%%%%%%%%%

\iftoggle{showAnswers}{\newpage}


%%%%%%%%%%%%%%%%%%%%%%%%%%%%%%%%%%%%%%%%%%%%%%%%%%%%%%%%%%%%%%%%%%%%%%%%
\subsection*{5.}

\nopagebreak

In the eight-term sequence $A$, $B$, $C$, $D$, $E$, $F$, $G$, $H$, the value of $C$ is 5 and the sum of any three consecutive terms is 30. What is $A + H$?

\fbox{(A) 17 \quad (B) 18 \quad (C) 25 \quad (D) 26 \quad (E) 43}

\begin{answer}
\begin{align*}
x
\end{align*}
\begin{empheq}[box={\mathbox[colback=white]}]{equation*}
    x
\end{empheq} 
\end{answer}
%%%%%%%%%%%%%%%%%%%%%%%%%%%%%%%%%%%%%%%%%%%%%%%%%%%%%%%%%%%%%%%%%%%%%%%%

\iftoggle{showAnswers}{\newpage}


%%%%%%%%%%%%%%%%%%%%%%%%%%%%%%%%%%%%%%%%%%%%%%%%%%%%%%%%%%%%%%%%%%%%%%%%
\subsection*{6.}

\nopagebreak

Let $a_1$, $a_2$, $\dots$ be a sequence for which $a_1 = 2$, $a_2 = 3$, and $a_n = a_{n - 1}/a_{n - 2}$ for each positive integer $n \ge 3$. What is $a_{2006}$?

\fbox{(A) $\frac{1}{2}$ \quad (B) $\frac{2}{3}$ \quad (C) $\frac{3}{2}$ \quad (D) 2 \quad (E) 3}

\begin{answer}
\begin{align*}
x
\end{align*}
\begin{empheq}[box={\mathbox[colback=white]}]{equation*}
    x
\end{empheq} 
\end{answer}
%%%%%%%%%%%%%%%%%%%%%%%%%%%%%%%%%%%%%%%%%%%%%%%%%%%%%%%%%%%%%%%%%%%%%%%%

\iftoggle{showAnswers}{\newpage}


%%%%%%%%%%%%%%%%%%%%%%%%%%%%%%%%%%%%%%%%%%%%%%%%%%%%%%%%%%%%%%%%%%%%%%%%
\subsection*{7.}

\nopagebreak

Suppose that $\{a_n\}$ is an arithmetic sequence with $a_1 + a_2 + \dots + a_{100} = 100$ and $a_{101} + a_{102} + \dots + a_{200} = 200$. What is the value of $a_2 - a_1$?

\fbox{(A) 0.0001 \quad (B) 0.001 \quad (C) 0.01 \quad (D) 0.1 \quad (E) 1}

\begin{answer}
\begin{align*}
x
\end{align*}
\begin{empheq}[box={\mathbox[colback=white]}]{equation*}
    x
\end{empheq} 
\end{answer}
%%%%%%%%%%%%%%%%%%%%%%%%%%%%%%%%%%%%%%%%%%%%%%%%%%%%%%%%%%%%%%%%%%%%%%%%

\iftoggle{showAnswers}{\newpage}


%%%%%%%%%%%%%%%%%%%%%%%%%%%%%%%%%%%%%%%%%%%%%%%%%%%%%%%%%%%%%%%%%%%%%%%%
\subsection*{8.}

\nopagebreak

Let $\{a_k\}$ be a sequence of integers such that $a_1 = 1$ and $a_{m + n} = a_m + a_n + mn$, for all positive integers $m$ and $n$. Then $a_{12}$ is

\fbox{(A) 45 \quad (B) 56 \quad (C) 67 \quad (D) 78 \quad (E) 89}

\begin{answer}
\begin{align*}
x
\end{align*}
\begin{empheq}[box={\mathbox[colback=white]}]{equation*}
    x
\end{empheq} 
\end{answer}
%%%%%%%%%%%%%%%%%%%%%%%%%%%%%%%%%%%%%%%%%%%%%%%%%%%%%%%%%%%%%%%%%%%%%%%%

\iftoggle{showAnswers}{\newpage}


%%%%%%%%%%%%%%%%%%%%%%%%%%%%%%%%%%%%%%%%%%%%%%%%%%%%%%%%%%%%%%%%%%%%%%%%
\subsection*{9.}

\nopagebreak

The first four terms in an arithmetic sequence are $x + y$, $x - y$, $xy$, and $x/y$, in that order. What is the fifth term?

\fbox{(A) $-\frac{15}{8}$ \quad (B) $-\frac{6}{5}$ \quad (C) $0$ \quad (D) $\frac{27}{20}$ \quad (E)  $\frac{123}{40}$}

\begin{answer}
\begin{align*}
x
\end{align*}
\begin{empheq}[box={\mathbox[colback=white]}]{equation*}
    x
\end{empheq} 
\end{answer}
%%%%%%%%%%%%%%%%%%%%%%%%%%%%%%%%%%%%%%%%%%%%%%%%%%%%%%%%%%%%%%%%%%%%%%%%

\iftoggle{showAnswers}{\newpage}


%%%%%%%%%%%%%%%%%%%%%%%%%%%%%%%%%%%%%%%%%%%%%%%%%%%%%%%%%%%%%%%%%%%%%%%%
\subsection*{10.}

\nopagebreak

Let $a_1$, $a_2$, $\dots$ be a sequence with the following properties.

(i) $a_1 = 1$, and

(ii) $a_{2n} = n \cdot a_n$ for any positive integer $n$.

What is the value of $a_{2^{100}}$? (The subscript is $2^{100}$.)

\fbox{(A) 1 \quad (B) $2^{99}$ \quad (C) $2^{100}$ \quad (D) $2^{4950}$ \quad (E) $2^{9999}$}

\begin{answer}
\begin{align*}
x
\end{align*}
\begin{empheq}[box={\mathbox[colback=white]}]{equation*}
    x
\end{empheq} 
\end{answer}
%%%%%%%%%%%%%%%%%%%%%%%%%%%%%%%%%%%%%%%%%%%%%%%%%%%%%%%%%%%%%%%%%%%%%%%%


\end{document}
