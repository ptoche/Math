\documentclass[12pt]{article}
\newif\ifanswer\answertrue\answerfalse% comment out to show/hide answers
\usepackage{../preamble3}% preamble always after \newif\ifanswer
%\pagenumbering{gobble}
\title{Art Of Problem Solving - AMC 10 \\ June 25, 2021}
\author{Patrick \& James Toche}
\date{Revised:~\today}

\begin{document}
\maketitle
\begin{minipage}{\textwidth}
\begin{abstract}\setlength{\parindent}{0pt}%
Notes on the AMC-10 Course by Art Of Problem Solving (AOPS).
Copyright restrictions may apply. Written for personal use. 
Please report typos and errors over at \url{https://github.com/ptoche/Math/tree/master/aops}. 
\end{abstract}
\end{minipage}

\thispagestyle{empty}
\clearpage


%%%%%%%%%%%%%%%%%%%%%%%%%%%%%%%%%%%%%%%%%%%%%%%%%%%%%%%%%%%%%%%%%%%%%%%%
\subsection*{1.}

\nopagebreak

For how many integers $n$ between 1 and 100 does $x^2 + x - n$ factor into the product of two linear factors with integer coefficients?

\fbox{(A) $0$ \quad (B) $1$ \quad (C) $2$ \quad (D) $9$ \quad (E) $10$}

\begin{answer}
\begin{align*}
x
\end{align*}
\begin{empheq}[box={\mathbox[colback=white]}]{equation*}
    x
\end{empheq} 
\end{answer}
%%%%%%%%%%%%%%%%%%%%%%%%%%%%%%%%%%%%%%%%%%%%%%%%%%%%%%%%%%%%%%%%%%%%%%%%

\iftoggle{showAnswers}{\newpage}

%%%%%%%%%%%%%%%%%%%%%%%%%%%%%%%%%%%%%%%%%%%%%%%%%%%%%%%%%%%%%%%%%%%%%%%%
\subsection*{2.}

\nopagebreak

Suppose that $a$ and $b$ are nonzero real numbers, and that the equation $x^2 + ax + b = 0$ has solutions $a$ and $b$. Then the pair $(a,b)$ is

\fbox{(A) $(-2,1)$ \quad (B) $(-1,2)$ \quad (C) $(1,-2)$ \quad (D) $(2,-1)$ \quad (E) $(4,4)$}

\begin{answer}
\begin{align*}
x
\end{align*}
\begin{empheq}[box={\mathbox[colback=white]}]{equation*}
    x
\end{empheq} 
\end{answer}
%%%%%%%%%%%%%%%%%%%%%%%%%%%%%%%%%%%%%%%%%%%%%%%%%%%%%%%%%%%%%%%%%%%%%%%%

\iftoggle{showAnswers}{\newpage}


%%%%%%%%%%%%%%%%%%%%%%%%%%%%%%%%%%%%%%%%%%%%%%%%%%%%%%%%%%%%%%%%%%%%%%%%
\subsection*{3.}

\nopagebreak

Let $f$ be the function defined by $f(x) = ax^2 - \sqrt{2}$ for some positive $a$. If $f(f(\sqrt{2})) = -\sqrt{2}$, then 

\fbox{(A) $\frac{2-\sqrt{2}}{2}$ \quad (B) $\frac{1}{2}$ \quad (C) $2-\sqrt{2}$ \quad (D) $\frac{\sqrt{2}}{2}$ \quad (E) $\frac{2+\sqrt{2}}{2}$}

\begin{answer}
\begin{align*}
x
\end{align*}
\begin{empheq}[box={\mathbox[colback=white]}]{equation*}
    x
\end{empheq} 
\end{answer}
%%%%%%%%%%%%%%%%%%%%%%%%%%%%%%%%%%%%%%%%%%%%%%%%%%%%%%%%%%%%%%%%%%%%%%%%

\iftoggle{showAnswers}{\newpage}


%%%%%%%%%%%%%%%%%%%%%%%%%%%%%%%%%%%%%%%%%%%%%%%%%%%%%%%%%%%%%%%%%%%%%%%%
\subsection*{4.}

\nopagebreak

Both roots of the quadratic equation $x^2 - 63x + k = 0$ are prime numbers. The number of possible values of $k$ is

\fbox{(A) $0$ \quad (B) $1$ \quad (C) $2$ \quad (D) $4$ \quad (E) more than four}

\begin{answer}
\begin{align*}
x
\end{align*}
\begin{empheq}[box={\mathbox[colback=white]}]{equation*}
    x
\end{empheq} 
\end{answer}
%%%%%%%%%%%%%%%%%%%%%%%%%%%%%%%%%%%%%%%%%%%%%%%%%%%%%%%%%%%%%%%%%%%%%%%%

\iftoggle{showAnswers}{\newpage}


%%%%%%%%%%%%%%%%%%%%%%%%%%%%%%%%%%%%%%%%%%%%%%%%%%%%%%%%%%%%%%%%%%%%%%%%
\subsection*{5.}

\nopagebreak

Let $@$ denote the ``averaged with'' operation: $a @ b = \frac{a + b}{2}$. Which of the following distributive laws holds for all numbers $x$, $y$, and $z$?

\begin{align*}
\text{I}.  \qquad & x @ (y + z) = (x @ y) + (x @ z) \\
\text{III}.\qquad & x + (y @ z) = (x + y) @ (x + z) \\
\text{III}.\qquad & x @ (y @ z) = (x @ y) @ (x @ z)
\end{align*}

\fbox{(A) I only \quad (B) II only (C) \quad III only (D) \quad I and III only (E) \quad II and III only }

\begin{answer}
\begin{align*}
x
\end{align*}
\begin{empheq}[box={\mathbox[colback=white]}]{equation*}
    x
\end{empheq} 
\end{answer}
%%%%%%%%%%%%%%%%%%%%%%%%%%%%%%%%%%%%%%%%%%%%%%%%%%%%%%%%%%%%%%%%%%%%%%%%

\iftoggle{showAnswers}{\newpage}

%%%%%%%%%%%%%%%%%%%%%%%%%%%%%%%%%%%%%%%%%%%%%%%%%%%%%%%%%%%%%%%%%%%%%%%%
\subsection*{6.}

\nopagebreak

If $f(x) = ax^4 - bx^2 + x + 5$ and $f(-3) = 2$, then $f(3) =$


\fbox{(A) $-5$ \quad (B) $-2$ \quad (C) $1$ \quad (D) $3$ \quad (E) $8$}

\begin{answer}
\begin{align*}
x
\end{align*}
\begin{empheq}[box={\mathbox[colback=white]}]{equation*}
    x
\end{empheq} 
\end{answer}
%%%%%%%%%%%%%%%%%%%%%%%%%%%%%%%%%%%%%%%%%%%%%%%%%%%%%%%%%%%%%%%%%%%%%%%%

\iftoggle{showAnswers}{\newpage}

%%%%%%%%%%%%%%%%%%%%%%%%%%%%%%%%%%%%%%%%%%%%%%%%%%%%%%%%%%%%%%%%%%%%%%%%
\subsection*{7.}

\nopagebreak

What is the sum of the reciprocals of the roots of the equation
\[\frac{2003}{2004} x + 1 + \frac{1}{x} = 0?\]

\fbox{(A) $-\frac{2004}{2003}$ \quad (B) $-1$  \quad (C) $\frac{2003}{2004}$ \quad (D) 1 \quad (E) $\frac{2004}{2003}$}

\begin{answer}
\begin{align*}
x
\end{align*}
\begin{empheq}[box={\mathbox[colback=white]}]{equation*}
    x
\end{empheq} 
\end{answer}
%%%%%%%%%%%%%%%%%%%%%%%%%%%%%%%%%%%%%%%%%%%%%%%%%%%%%%%%%%%%%%%%%%%%%%%%

\iftoggle{showAnswers}{\newpage}


%%%%%%%%%%%%%%%%%%%%%%%%%%%%%%%%%%%%%%%%%%%%%%%%%%%%%%%%%%%%%%%%%%%%%%%%
\subsection*{8.}

\nopagebreak

Let $f$ be a polynomial function such that, for all real $x$,
\[f(x^2 + 1) = x^4 + 5x^2 + 3.\]

\fbox{(A) $x^4 + 5x^2 + 1$ \quad (B) $x^4 + x^2 - 3$ \quad (C) $x^4 - 5x^2 + 1$ \quad (D) $x^4 + x^2 + 3$ \quad (E) none of these }

\begin{answer}
\begin{align*}
x
\end{align*}
\begin{empheq}[box={\mathbox[colback=white]}]{equation*}
    x
\end{empheq} 
\end{answer}
%%%%%%%%%%%%%%%%%%%%%%%%%%%%%%%%%%%%%%%%%%%%%%%%%%%%%%%%%%%%%%%%%%%%%%%%

\iftoggle{showAnswers}{\newpage}


%%%%%%%%%%%%%%%%%%%%%%%%%%%%%%%%%%%%%%%%%%%%%%%%%%%%%%%%%%%%%%%%%%%%%%%%
\subsection*{9.}

\nopagebreak

The polynomial $x^3 - ax^2 + bx - 2010$ has three positive integer roots. What is the smallest possible value of $a$?

\fbox{(A) $78$ \quad (B) $88$ \quad (C) $98$ \quad (D) $108$ \quad (E) $118$}

\begin{answer}
\begin{align*}
x
\end{align*}
\begin{empheq}[box={\mathbox[colback=white]}]{equation*}
    x
\end{empheq} 
\end{answer}
%%%%%%%%%%%%%%%%%%%%%%%%%%%%%%%%%%%%%%%%%%%%%%%%%%%%%%%%%%%%%%%%%%%%%%%%

\iftoggle{showAnswers}{\newpage}


%%%%%%%%%%%%%%%%%%%%%%%%%%%%%%%%%%%%%%%%%%%%%%%%%%%%%%%%%%%%%%%%%%%%%%%%
\subsection*{10.}

\nopagebreak

Let $f$ be a function for which $f(x/3) = x^2 + x + 1$. Find the sum of all values of $z$ for which $f(3z) = 7$.

\fbox{(A) $-1/3$ \quad (B) $-1/9$ \quad (C) $0$ \quad (D) $5/9$ \quad (E) $5/3$}

\begin{answer}
\begin{align*}
x
\end{align*}
\begin{empheq}[box={\mathbox[colback=white]}]{equation*}
    x
\end{empheq} 
\end{answer}
%%%%%%%%%%%%%%%%%%%%%%%%%%%%%%%%%%%%%%%%%%%%%%%%%%%%%%%%%%%%%%%%%%%%%%%%


\end{document}
