\documentclass[12pt]{article}
\newif\ifanswer\answertrue%\answerfalse% comment out to show/hide answers
\usepackage{../preamble}% preamble always after \newif\ifanswer
%\pagenumbering{gobble}
\title{Art Of Problem Solving - AMC 10 \\ Week 11}
\author{Patrick \& James Toche}
\date{August 21, 2021}

\begin{document}
\maketitle
\begin{minipage}{\textwidth}
\begin{abstract}\setlength{\parindent}{0pt}%
Notes on the AMC-10 Course by Art Of Problem Solving (AOPS).
Copyright restrictions may apply. Written for personal use. 
Please report typos and errors over at \url{https://github.com/ptoche/Math/tree/master/aops}. 
\end{abstract}
\end{minipage}

\thispagestyle{empty}
\clearpage


%%%%%%%%%%%%%%%%%%%%%%%%%%%%%%%%%%%%%%%%%%%%%%%%%%%%%%%%%%%%%%%%%%%%%%%%
\subsection*{1.}

\nopagebreak

Mrs. Walter gave an exam in a mathematics class of five students. She entered the scores in random order into a spreadsheet, which recalculated the class average after each score was entered. Mrs. Walter noticed that after each score was entered, the average was always an integer. The scores (listed in ascending order) were $71$, $76$, $80$, $82$, and $91$. What was the last score Mrs. Walter entered?

\nopagebreak

\fbox{(A) $71$ \quad (B) $76$ \quad (C) $80$ \quad (D) $82$ \quad (E) $91$}

\begin{answer}
The first number is divisible by $1$.

The sum of the first two numbers is even.

The sum of the first three numbers is divisible by $3$.

The sum of the first four numbers is divisible by $4$.

The sum of the first five numbers is $400$.

Since $400$ is divisible by $4$. the last score must also be divisible by $4.$ Therefore, the last score is either $76$ or $80$.

Case 1: $76$ is the last number entered.

Since $400\equiv 76\equiv 1\pmod{3}$, the fourth number must be divisible by $3,$ but none of the scores are divisible by $3$.

Case 2: $80$ is the last number entered.

Since $80\equiv 2\pmod{3}$, the fourth number must be $2\pmod{3}$. That number is $71$ and $71$ only. The next number must be $91$ since the sum of the first two numbers is even. So the only arrangement of the scores $76, 82, 91, 71, 80$ 
\begin{empheq}[box={\mathbox[colback=white]}]{equation*}
    80
\end{empheq} 
\end{answer}
%%%%%%%%%%%%%%%%%%%%%%%%%%%%%%%%%%%%%%%%%%%%%%%%%%%%%%%%%%%%%%%%%%%%%%%%

\iftoggle{showAnswers}{\newpage}

%%%%%%%%%%%%%%%%%%%%%%%%%%%%%%%%%%%%%%%%%%%%%%%%%%%%%%%%%%%%%%%%%%%%%%%%
\subsection*{2.}

\nopagebreak

The digits $1$, $2$, $3$, $4$, $5$, $6$, $7$, and $9$ are used to form four two-digit prime numbers, with each digit used exactly once. What is the sum of these four primes?

\nopagebreak

\fbox{(A) $150$ \quad (B) $160$ \quad (C) $170$ \quad (D) $180$ \quad (E) $190$}

\begin{answer}
Since a multiple-digit prime number is not divisible by either $2$ or $5$, it must end with $1$, $3$, $7$, or $9$ in the units place. The remaining digits given must therefore appear in the tens place.
\begin{align*}
20 + 40 + 50 + 60 + 1 + 3 + 7 + 9 = 190
\end{align*}
\begin{empheq}[box={\mathbox[colback=white]}]{equation*}
    190
\end{empheq} 
\end{answer}
%%%%%%%%%%%%%%%%%%%%%%%%%%%%%%%%%%%%%%%%%%%%%%%%%%%%%%%%%%%%%%%%%%%%%%%%

\iftoggle{showAnswers}{\newpage}

%%%%%%%%%%%%%%%%%%%%%%%%%%%%%%%%%%%%%%%%%%%%%%%%%%%%%%%%%%%%%%%%%%%%%%%%
\subsection*{3.}

\nopagebreak

Suppose that $m$ and $n$ are positive integers such that $75m=n^3$. What is the minimum possible value of $m+n$?

\nopagebreak

\fbox{(A) $15$ \quad (B) $30$ \quad (C) $50$ \quad (D) $60$ \quad (E) $5700$}

\begin{answer}
$3 \cdot 5^2m$ must be a perfect cube, so each power of a prime in the factorization for $3 \cdot 5^2m$ must be divisible by $3$. Thus the minimum value of $m$ is $3^2 \cdot 5 = 45$, which makes 
\begin{align*}
n=\sqrt[3]{3^3 \cdot 5^3} = 15
\end{align*}
The minimum possible value for the sum of $m$ and $n$ is $60$. 
\begin{empheq}[box={\mathbox[colback=white]}]{equation*}
    60
\end{empheq} 
\end{answer}
%%%%%%%%%%%%%%%%%%%%%%%%%%%%%%%%%%%%%%%%%%%%%%%%%%%%%%%%%%%%%%%%%%%%%%%%

\iftoggle{showAnswers}{\newpage}

%%%%%%%%%%%%%%%%%%%%%%%%%%%%%%%%%%%%%%%%%%%%%%%%%%%%%%%%%%%%%%%%%%%%%%%%
\subsection*{4.}

\nopagebreak

What is the largest integer that is a divisor of $(n+1)(n+3)(n+5)(n+7)(n+9)$ for all positive even integers $n$?

\nopagebreak

\fbox{(A) $3$ \quad (B) $5$ \quad (C) $11$ \quad (D) $15$ \quad (E) $165$}

\begin{answer}
For all consecutive odd integers, one of every five is a multiple of $5$ and one of every three is a multiple of $3$. The answer is $3 \cdot 5 = 15$. 
\begin{empheq}[box={\mathbox[colback=white]}]{equation*}
    15
\end{empheq} 
\end{answer}
%%%%%%%%%%%%%%%%%%%%%%%%%%%%%%%%%%%%%%%%%%%%%%%%%%%%%%%%%%%%%%%%%%%%%%%%

\iftoggle{showAnswers}{\newpage}

%%%%%%%%%%%%%%%%%%%%%%%%%%%%%%%%%%%%%%%%%%%%%%%%%%%%%%%%%%%%%%%%%%%%%%%%
\subsection*{5.}

\nopagebreak

Let $S$ be the set of the $2005$ smallest positive multiples of $4$, and let $T$ be the set of the $2005$ smallest positive multiples of $6$. How many elements are common to $S$ and $T$?

\nopagebreak

\fbox{(A) $166$ \quad (B) $333$ \quad (C) $500$ \quad (D) $668$ \quad (E) $1001$}

\begin{answer}
Since the least common multiple $\mathrm{lcm}(4,6)=12$, the elements that are common to $S$ and $T$ must be multiples of $12$.

Since $4\cdot2005=8020$ and $6\cdot2005=12030$, several multiples of $12$ that are in $T$ won't be in $S$, but all multiples of $12$ that are in $S$ will be in $T$. So we just need to find the number of multiples of $12$ that are in $S$.

Since $4\cdot3=12$, every $3$rd element of $S$ will be a multiple of $12$.

Therefore the answer is 
\begin{align*}
\left\lfloor\frac{2005}{3}\right\rfloor = 668
\end{align*}
\begin{empheq}[box={\mathbox[colback=white]}]{equation*}
    668
\end{empheq} 
\end{answer}
%%%%%%%%%%%%%%%%%%%%%%%%%%%%%%%%%%%%%%%%%%%%%%%%%%%%%%%%%%%%%%%%%%%%%%%%

\iftoggle{showAnswers}{\newpage}

%%%%%%%%%%%%%%%%%%%%%%%%%%%%%%%%%%%%%%%%%%%%%%%%%%%%%%%%%%%%%%%%%%%%%%%%
\subsection*{6.}

\nopagebreak

\nopagebreak

For how many positive integers $n$ less than or equal to $24$ is $n!$ evenly divisible by $1+2+\dots+n$?

\fbox{(A) $8$ \quad (B) $12$ \quad (C) $16$ \quad (D) $17$ \quad (E) $21$}

\begin{answer}
Since $1 + 2 + \cdots + n = \frac{n(n+1)}{2}$, the condition is equivalent to having an integer value for 
\begin{align*}
\frac{n!}{\dfrac{n(n+1)}{2}}
\end{align*}
This reduces, when $n\ge 1$, to having an integer value for
\begin{align*}
\frac{2(n-1)!}{n+1}
\end{align*}
This fraction is an integer unless $n+1$ is an odd prime. There are $8$ odd primes less than or equal to $24$, so there are $24-8=16$ numbers less than or equal to $24$ that satisfy the condition. 
\begin{empheq}[box={\mathbox[colback=white]}]{equation*}
    16
\end{empheq} 
\end{answer}
%%%%%%%%%%%%%%%%%%%%%%%%%%%%%%%%%%%%%%%%%%%%%%%%%%%%%%%%%%%%%%%%%%%%%%%%

\iftoggle{showAnswers}{\newpage}

%%%%%%%%%%%%%%%%%%%%%%%%%%%%%%%%%%%%%%%%%%%%%%%%%%%%%%%%%%%%%%%%%%%%%%%%
\subsection*{7.}

\nopagebreak

A finite sequence of three-digit integers has the property that the tens and units digits of each term are, respectively, the hundreds and tens digits of the next term, and the tens and units digits of the last term are, respectively, the hundreds and tens digits of the first term. For example, such a sequence might begin with terms $247$, $475$, and $756$ and end with the term $824$. Let $S$ be the sum of all the terms in the sequence. What is the largest prime number that always divides $S$?

\nopagebreak

\fbox{(A) $3$ \quad (B) $7$ \quad (C) $13$ \quad (D) $37$ \quad (E) $43$}

\begin{answer}
A given digit appears as the hundreds digit, the tens digit, and the units digit of a term the same number of times. Let $k$ be the sum of the units digits in all the terms. Then 
\begin{align*}
S = 111k = 3 \cdot 37k
\end{align*}
so $S$ must be divisible by $37$. To see that it need not be divisible by any larger prime, the sequence $123$, $231$, $312$ gives. 
\begin{align*}
S = 666 = 2 \cdot 3^2 \cdot 37
\end{align*}
\begin{empheq}[box={\mathbox[colback=white]}]{equation*}
    37
\end{empheq} 
\end{answer}
%%%%%%%%%%%%%%%%%%%%%%%%%%%%%%%%%%%%%%%%%%%%%%%%%%%%%%%%%%%%%%%%%%%%%%%%

\iftoggle{showAnswers}{\newpage}

%%%%%%%%%%%%%%%%%%%%%%%%%%%%%%%%%%%%%%%%%%%%%%%%%%%%%%%%%%%%%%%%%%%%%%%%
\subsection*{8.}

\nopagebreak

Sally has five red cards numbered $1$ through $5$ and four blue cards numbered $3$ through $6$. She stacks the cards so that the colors alternate and so that the number on each red card divides evenly into the number on each neighboring blue card. What is the sum of the numbers on the middle three cards?

\nopagebreak

\fbox{(A) $8$ \quad (B) $9$ \quad (C) $10$ \quad (D) $11$ \quad (E) $12$}

\begin{answer}
Let $R_i$ and $B_j$ designate the red card numbered $i$ and the blue card numbered $j$, respectively.

$B_5$ is the only blue card that $R_5$ evenly divides, so $R_5$ must be at one end of the stack and $B_5$ must be the card next to it.

$R_1$ is the only other red card that evenly divides $B_5$, so $R_1$ must be the other card next to $B_5$.

$B_4$ is the only blue card that $R_4$ evenly divides, so $R_4$ must be at one end of the stack and $B_4$ must be the card next to it.

$R_2$ is the only other red card that evenly divides $B_4$, so $R_2$ must be the other card next to $B_4$.

$R_2$ doesn't evenly divide $B_3$, so $B_3$ must be next to $R_1$, $B_6$ must be next to $R_2$, and $R_3$ must be in the middle.

This yields the following arrangement from top to bottom: $\{R_5,B_5,R_1,B_3,R_3,B_6,R_2,B_4,R_4\}$

Therefore, the sum of the numbers on the middle three cards is $3+3+6=12$. 
\begin{empheq}[box={\mathbox[colback=white]}]{equation*}
    12
\end{empheq} 
\end{answer}
%%%%%%%%%%%%%%%%%%%%%%%%%%%%%%%%%%%%%%%%%%%%%%%%%%%%%%%%%%%%%%%%%%%%%%%%

\iftoggle{showAnswers}{\newpage}

%%%%%%%%%%%%%%%%%%%%%%%%%%%%%%%%%%%%%%%%%%%%%%%%%%%%%%%%%%%%%%%%%%%%%%%%
\subsection*{9.}

\nopagebreak

Let $x$ and $y$ be two-digit integers such that $y$ is obtained by reversing the digits of $x$. The integers $x$ and $y$ satisfy $x^2-y^2=m^2$ for some positive integer $m$. What is $x+y+m$?

\nopagebreak

\fbox{(A) $88$ \quad (B) $112$ \quad (C) $116$ \quad (D) $144$ \quad (E) $154$}

\begin{answer}
Let $x = 10a+b, y = 10b+a$. The given conditions imply $x>y$, which implies $a>b$, and they also imply that both $a$ and $b$ are nonzero. Then 
\begin{align*}
x^2 - y^2 
  & = (x-y)(x+y) \\ 
  & = (9a - 9b)(11a + 11b) \\ 
  & = 99(a-b)(a+b) \\
  & = m^2
\end{align*}
Since this must be a perfect square, all the exponents in its prime factorization must be even. $99$ factorizes into $3^2 \cdot 11$, so $11|(a-b)(a+b)$. However, the maximum value of $a-b$ is $9-1=8$, so $11|a+b$. The maximum value of $a+b$ is $9+8=17$, so $a+b=11$. Then we have $33^2(a-b) = m^2$, so $a-b$ is a perfect square, but the only perfect squares that are within our bound on $a-b$ are $1$ and $4$. We know $a+b=11$, and, for $a-b=1$, adding equations to eliminate $b$ gives us $2a=12 \Longrightarrow a=6, b=5$. Testing $a-b=4$ gives us $2a=15 \Longrightarrow a=\frac{15}{2}, b=\frac{7}{2}$, which is impossible, as $a$ and $b$ must be digits. Therefore, $(a,b)=(6,5)$, and
\begin{align*}
x + y + m = 65 + 56 + 33 = 154
\end{align*}
\begin{empheq}[box={\mathbox[colback=white]}]{equation*}
    154
\end{empheq} 
\end{answer}
%%%%%%%%%%%%%%%%%%%%%%%%%%%%%%%%%%%%%%%%%%%%%%%%%%%%%%%%%%%%%%%%%%%%%%%%

\iftoggle{showAnswers}{\newpage}

%%%%%%%%%%%%%%%%%%%%%%%%%%%%%%%%%%%%%%%%%%%%%%%%%%%%%%%%%%%%%%%%%%%%%%%%
\subsection*{10.}

\nopagebreak

A high school basketball game between the Raiders and the Wildcats was tied at the end of the first quarter. The number of points scored by the Raiders in each of the four quarters formed an increasing geometric sequence, and the number of points scored by the Wildcats in each of the four quarters formed an increasing arithmetic sequence. At the end of the fourth quarter, the Raiders had won by one point. Neither team scored more than $100$ points. What was the total number of points scored by the two teams in the first half?

\nopagebreak

\fbox{(A) $30$ \quad (B) $31$ \quad (C) $32$ \quad (D) $33$ \quad (E) $34$}

\begin{answer}
Let $a,ar,ar^{2},ar^{3}$ be the quarterly scores for the Raiders. We know that the Raiders and Wildcats both scored the same number of points in the first quarter so let $a,a+d,a+2d,a+3d$ be the quarterly scores for the Wildcats. The sum of the Raiders scores is $a(1+r+r^{2}+r^{3})$ and the sum of the Wildcats scores is $4a+6d$. Now we can narrow our search for the values of $a,d$, and $r$. Because points are always measured in positive integers, we can conclude that $a$ and $d$ are positive integers. We can also conclude that $r$ is a positive integer by writing down the equation:
\begin{align*}
a(1+r+r^{2}+r^{3}) = 4a + 6d + 1
\end{align*}
Now we can start trying out some values of $r$. We try $r=2$, which gives
\begin{align*}
15a & = 4a + 6d + 1 \\
11a & = 6d + 1
\end{align*}
We need the smallest multiple of $11$ (to satisfy the <100 condition) that is $\equiv 1 \pmod{6}$. We see that this is $55$, and therefore $a=5$ and $d=9$.
So the Raiders' first two scores were $5$ and $10$ and the Wildcats' first two scores were $5$ and $14$.
\begin{align*}
5 + 10 + 5 + 14 = 34
\end{align*}
\begin{empheq}[box={\mathbox[colback=white]}]{equation*}
    34
\end{empheq} 
\end{answer}
%%%%%%%%%%%%%%%%%%%%%%%%%%%%%%%%%%%%%%%%%%%%%%%%%%%%%%%%%%%%%%%%%%%%%%%%

\end{document}
