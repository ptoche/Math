\documentclass[12pt]{article}
\newif\ifanswer\answertrue%\answerfalse% comment out to show/hide answers
\usepackage{../preamble3}% preamble always after \newif\ifanswer
%\pagenumbering{gobble}
\title{Art Of Problem Solving - AMC 10 \\ Week 4}
\author{Patrick \& James Toche}
\date{July 2nd, 2021}

\begin{document}
\maketitle
\begin{minipage}{\textwidth}
\begin{abstract}\setlength{\parindent}{0pt}%
Notes on the AMC-10 Course by Art Of Problem Solving (AOPS).
Copyright restrictions may apply. Written for personal use. 
Please report typos and errors over at \url{https://github.com/ptoche/Math/tree/master/aops}. 
\end{abstract}
\end{minipage}

\thispagestyle{empty}
\clearpage


%%%%%%%%%%%%%%%%%%%%%%%%%%%%%%%%%%%%%%%%%%%%%%%%%%%%%%%%%%%%%%%%%%%%%%%%
\subsection*{1.}

\nopagebreak

For how many integers $n$ between 1 and 100 does $x^2 + x - n$ factor into the product of two linear factors with integer coefficients?

\nopagebreak

\fbox{(A) $0$ \quad (B) $1$ \quad (C) $2$ \quad (D) $9$ \quad (E) $10$}

\begin{answer}
Does ``between'' mean $n \in [1, 100]$ or $n \in (1, 100)$?

The quadratic factors as:
\begin{align*}
x^2 + x - n 
  & = \left(x + \frac{1}{2}\right)^2 - \left(\frac{1}{2}\right)^2 - n \\
  & = \left(x + \frac{1}{2}\right)^2 - \frac{1+4n}{2^2} \\
  & = \left(x + \frac{1}{2} - \frac{\sqrt{1+4n}}{2}\right) \left(x + \frac{1}{2} + \frac{\sqrt{1+4n}}{2}\right) \\
  & = \left(x + \frac{1-\sqrt{1+4n}}{2}\right) \left(x + \frac{1+\sqrt{1+4n}}{2}\right)
\end{align*}
or skip these steps and write down the quadratic formula.

Both factors have integer coefficients if
\begin{align*}
\frac{1-\sqrt{1+4n}}{2}, \qquad \frac{1+\sqrt{1+4n}}{2}
\end{align*}
are integers. Equivalently, if
\begin{align*}
1-\sqrt{1+4n}, \qquad 1+\sqrt{1+4n}
\end{align*}
are even integers. Equivalently, if there is an integer $k$ such that
\begin{align*}
\sqrt{1+4n} 
  & = 2k + 1 \\
\implies
1+4n 
  & = (2k + 1)^2 \\
  & = 4k^2 + 4k + 1 \\
\implies
4n 
  & = 4k^2 + 4k \\
n & = k(k+1)
\end{align*}
The condition is satisfied for any $n \in [1,100]$ that is the product of two successive integers. 
\begin{align*}
n = 1 \times 2, \qquad 2 \times 3, \qquad 3 \times 4, \qquad \ldots \qquad 9 \times 10
\end{align*}
since $10\times11=110>100$. Thus, there are $1\rightarrow9$ cases:
\begin{empheq}[box={\mathbox[colback=white]}]{equation*}
    9
\end{empheq} 
\end{answer}
%%%%%%%%%%%%%%%%%%%%%%%%%%%%%%%%%%%%%%%%%%%%%%%%%%%%%%%%%%%%%%%%%%%%%%%%

\iftoggle{showAnswers}{\newpage}

%%%%%%%%%%%%%%%%%%%%%%%%%%%%%%%%%%%%%%%%%%%%%%%%%%%%%%%%%%%%%%%%%%%%%%%%
\subsection*{2.}

\nopagebreak

Suppose that $a$ and $b$ are nonzero real numbers, and that the equation $x^2 + ax + b = 0$ has solutions $a$ and $b$. Then the pair $(a,b)$ is

\nopagebreak

\fbox{(A) $(-2,1)$ \quad (B) $(-1,2)$ \quad (C) $(1,-2)$ \quad (D) $(2,-1)$ \quad (E) $(4,4)$}

\begin{answer}
The pair $(a,b)$ can be used in the factorization $(x-a)(x-b)$, so that:
\begin{align*}
x^2 + ax + b 
  & = (x-a)(x-b) \\
  & = x^2 - (a+b)x + ab
\end{align*}
Equating the coefficients of the polynomial yields the linear system in $(a,b)$:
\begin{align*}
a & = -(a + b) \\
b & = ab
\end{align*}
With solution $(1,-2)$ and $(0,0)$, but only non-zero solutions are acceptable.
\begin{empheq}[box={\mathbox[colback=white]}]{equation*}
    (1, -2)
\end{empheq} 
\end{answer}
%%%%%%%%%%%%%%%%%%%%%%%%%%%%%%%%%%%%%%%%%%%%%%%%%%%%%%%%%%%%%%%%%%%%%%%%

\iftoggle{showAnswers}{\newpage}


%%%%%%%%%%%%%%%%%%%%%%%%%%%%%%%%%%%%%%%%%%%%%%%%%%%%%%%%%%%%%%%%%%%%%%%%
\subsection*{3.}

\nopagebreak

Let $f$ be the function defined by $f(x) = ax^2 - \sqrt{2}$ for some positive $a$. If $f(f(\sqrt{2})) = -\sqrt{2}$, then $a=$

\nopagebreak

\fbox{(A) $\dfrac{2-\sqrt{2}}{2}$ \quad (B) $\dfrac{1}{2}$ \quad (C) $2-\sqrt{2}$ \quad (D) $\dfrac{\sqrt{2}}{2}$ \quad (E) $\dfrac{2+\sqrt{2}}{2}$}

\begin{answer}
Apply function $f$ twice to the $\sqrt{2}$ argument:
\begin{align*}
f(x) 
  & = ax^2 - \sqrt{2} \\
\implies
f(\sqrt{2})
  & = a(\sqrt{2})^2 - \sqrt{2} = 2a - \sqrt{2} \\
f(f(\sqrt{2})) 
  & = a\left(2a - \sqrt{2}\right)^2 - \sqrt{2} \\
  & = a(4a^2 -4\sqrt{2}a + 2) - \sqrt{2} \\
  & = 4a^3 - 4\sqrt{2}a^2 + 2a - \sqrt{2} 
\end{align*}
We equate with the given value:
\begin{align*}
4a^3 - 4\sqrt{2}a^2 + 2a - \sqrt{2} = -\sqrt{2} \\
\implies
4a^3 - 4\sqrt{2}a^2 + 2a & = 0 \\
a \left(a^2 - \sqrt{2}a + \frac{1}{2} \right) & = 0 \\
\left(a - \frac{\sqrt{2}}{2}\right)^2 - \left(\frac{\sqrt{2}}{2}\right)^2 + \frac{1}{2} & = 0 \qquad\text{because}~a>0\\
\left(a - \frac{\sqrt{2}}{2}\right)^2 & = 0 \\
\end{align*}
\begin{empheq}[box={\mathbox[colback=white]}]{equation*}
    a = \frac{\sqrt{2}}{2}
\end{empheq} 
\end{answer}
%%%%%%%%%%%%%%%%%%%%%%%%%%%%%%%%%%%%%%%%%%%%%%%%%%%%%%%%%%%%%%%%%%%%%%%%

\iftoggle{showAnswers}{\newpage}


%%%%%%%%%%%%%%%%%%%%%%%%%%%%%%%%%%%%%%%%%%%%%%%%%%%%%%%%%%%%%%%%%%%%%%%%
\subsection*{4.}

\nopagebreak

Both roots of the quadratic equation $x^2 - 63x + k = 0$ are prime numbers. The number of possible values of $k$ is

\nopagebreak

\fbox{(A) $0$ \quad (B) $1$ \quad (C) $2$ \quad (D) $4$ \quad (E) more than four}

\begin{answer}
Vi\`{e}te's formula for two roots $r,s$ is
\begin{align*}
(x-r)(x-s) = x^2 - (r+s)x + rs
\end{align*}
In the present case, the implication is that the sum of the roots is $63$ and their product is $k$. For the sum of the roots to be odd, one of them must be odd and the other one must be even (since odd$+$odd$=$even, even$+$even$=$even). The only even prime is $2$. Thus, the roots must be $61$ and $2$, which gives the only possible value of $k$, $k=122$.
\begin{empheq}[box={\mathbox[colback=white]}]{equation*}
  1~\text{possible value of $k$}
\end{empheq} 
\end{answer}
%%%%%%%%%%%%%%%%%%%%%%%%%%%%%%%%%%%%%%%%%%%%%%%%%%%%%%%%%%%%%%%%%%%%%%%%

\iftoggle{showAnswers}{\newpage}


%%%%%%%%%%%%%%%%%%%%%%%%%%%%%%%%%%%%%%%%%%%%%%%%%%%%%%%%%%%%%%%%%%%%%%%%
\subsection*{5.}

\nopagebreak

Let $@$ denote the ``averaged with'' operation: $a @ b = \frac{a + b}{2}$. Which of the following distributive laws holds for all numbers $x$, $y$, and $z$?
\begin{align*}
\text{I}.  \qquad & x @ (y + z) = (x @ y) + (x @ z) \\
\text{III}.\qquad & x + (y @ z) = (x + y) @ (x + z) \\
\text{III}.\qquad & x @ (y @ z) = (x @ y) @ (x @ z)
\end{align*}

\nopagebreak

\fbox{(A) I only \quad (B) II only (C) \quad III only (D) \quad I and III only (E) \quad II and III only }

\begin{answer}
Apply the rule to the left-hand side of I:
\begin{align*}
x @ (y + z) 
  & =  \frac{x + (y+z)}{2}  \\
  & = \frac{x+y}{2} + \frac{z}{2} \\
  & = x @ y + 0 @ z \\
  & \ne x @ y + x @ z
\end{align*}
so I is false. 

Apply the rule to the left-hand side of II:
\begin{align*}
x + (y @ z)
  & = \frac{2x}{2} + \frac{y+z}{2}  \\
  & = \frac{x+y}{2} + \frac{x+z}{2}  \\
  & = (x + y) @ (x + z) 
\end{align*}
so II is true.

Apply the rule to the left-hand side of III:
\begin{align*}
x @ (y @ z)
  & = \frac{x + \frac{y+z}{2}}{2}  \\
  & = \frac{2x+y+z}{4} \\
  & = \frac{\frac{x+y}{2} + \frac{x+z}{2}}{2} \\
  & = (x @ y) @ (x @ z)
\end{align*}
so III is true.

\begin{empheq}[box={\mathbox[colback=white]}]{equation*}
    \text{II and III only}
\end{empheq} 
\end{answer}
%%%%%%%%%%%%%%%%%%%%%%%%%%%%%%%%%%%%%%%%%%%%%%%%%%%%%%%%%%%%%%%%%%%%%%%%

\iftoggle{showAnswers}{\newpage}

%%%%%%%%%%%%%%%%%%%%%%%%%%%%%%%%%%%%%%%%%%%%%%%%%%%%%%%%%%%%%%%%%%%%%%%%
\subsection*{6.}

\nopagebreak

If $f(x) = ax^4 - bx^2 + x + 5$ and $f(-3) = 2$, then $f(3) =$

\nopagebreak

\fbox{(A) $-5$ \quad (B) $-2$ \quad (C) $1$ \quad (D) $3$ \quad (E) $8$}

\begin{answer}
Calculating $f(3)$ and $f(-3)$ yields: 
\begin{align*}
f(-3) 
  & = a(-3)^4 - b(-3)^2 - 3 + 5 
    = 81a - 9b + 2 = 2 \\
f(+3) 
  & = a(+3)^4 - b(+3)^2 + 3 + 5 
    = 81a - 9b + 8
    = 81a - 9b + 2 + 6
    = 2 + 6
    = 8
\end{align*}
\begin{empheq}[box={\mathbox[colback=white]}]{equation*}
    f(3) = 8
\end{empheq} 
\end{answer}
%%%%%%%%%%%%%%%%%%%%%%%%%%%%%%%%%%%%%%%%%%%%%%%%%%%%%%%%%%%%%%%%%%%%%%%%

\iftoggle{showAnswers}{\newpage}

%%%%%%%%%%%%%%%%%%%%%%%%%%%%%%%%%%%%%%%%%%%%%%%%%%%%%%%%%%%%%%%%%%%%%%%%
\subsection*{7.}

\nopagebreak

What is the sum of the reciprocals of the roots of the equation
\begin{align*}
\frac{2003}{2004} x + 1 + \frac{1}{x} = 0?
\end{align*}

\nopagebreak

\fbox{(A) $-\dfrac{2004}{2003}$ \quad (B) $-1$  \quad (C) $\dfrac{2003}{2004}$ \quad (D) 1 \quad (E) $\dfrac{2004}{2003}$}

\begin{answer}
Let $r,s$ denote the roots. The sum of the reciprocals is:
\begin{align*}
\frac{1}{r} + \frac{1}{s} 
 = \frac{r+s}{rs} 
\end{align*}
We can find the sum and product of the roots by rearranging the equation and applying Vi\`{e}te's formula. Rearranging:
\begin{align*}
x^2  - \frac{-2004}{\hfill2003}\ x + \frac{2004}{2003} & = 0
\end{align*}
The roots must satisfy
\begin{align*}
r + s & = -\frac{2004}{\hfill2003}\\[1ex]
  r s & = +\frac{2004}{\hfill2003} \\[1ex]
\implies
\frac{r+s}{rs} 
      & = -1
\end{align*}
And thus,
\begin{empheq}[box={\mathbox[colback=white]}]{equation*}
    \frac{1}{r} + \frac{1}{s} = -1
\end{empheq} 
\end{answer}
%%%%%%%%%%%%%%%%%%%%%%%%%%%%%%%%%%%%%%%%%%%%%%%%%%%%%%%%%%%%%%%%%%%%%%%%

\iftoggle{showAnswers}{\newpage}


%%%%%%%%%%%%%%%%%%%%%%%%%%%%%%%%%%%%%%%%%%%%%%%%%%%%%%%%%%%%%%%%%%%%%%%%
\subsection*{8.}

\nopagebreak

Let $f$ be a polynomial function such that, for all real $x$,
\begin{align*}
f(x^2 + 1) = x^4 + 5x^2 + 3
\end{align*}
For all real $x$, $f(x^2 - 1)$ is

\nopagebreak

\fbox{(A) $x^4 + 5x^2 + 1$ \quad (B) $x^4 + x^2 - 3$ \quad (C) $x^4 - 5x^2 + 1$ \quad (D) $x^4 + x^2 + 3$ \quad (E) none of these }

\begin{answer}
Let $X=x^2+1$ for any real $x$. This implies:
\begin{align*}
x^2 = X - 1 
\implies
   x^4 = (X - 1)^2
\end{align*}
Substituting these back into the polynomial function gives:
\begin{align*}
f(x^2 + 1) & = x^4 + 5x^2 + 3 \\
      f(X) & = (X-1)^2 + 5(X-1) + 3 \\
           & = X^2 + 3X -1 
\end{align*}
This is true for any $X\geq1$. Now substitute $x^2-1$ for $X$ and simplify:
\begin{align*}
f(x^2-1) & = (x^2-1)^2 + 3(x^2-1) - 1 \\
         & = x^4 + x^2 - 3 
\end{align*}
\begin{empheq}[box={\mathbox[colback=white]}]{equation*}
    f(x^2-1) = x^4 + x^2 - 3
\end{empheq} 
\end{answer}
%%%%%%%%%%%%%%%%%%%%%%%%%%%%%%%%%%%%%%%%%%%%%%%%%%%%%%%%%%%%%%%%%%%%%%%%

\iftoggle{showAnswers}{\newpage}


%%%%%%%%%%%%%%%%%%%%%%%%%%%%%%%%%%%%%%%%%%%%%%%%%%%%%%%%%%%%%%%%%%%%%%%%
\subsection*{9.}

\nopagebreak

The polynomial $x^3 - ax^2 + bx - 2010$ has three positive integer roots. What is the smallest possible value of $a$?

\nopagebreak

\fbox{(A) $78$ \quad (B) $88$ \quad (C) $98$ \quad (D) $108$ \quad (E) $118$}

\begin{answer}
The smallest root must be at least $0$. Vi\`{e}te's formula for a cubic with roots $r,s,t$ may be written as:
\begin{align*}
x^3 - (r+s+t)x^2 + (rs + rt + st)x - rst = 0
\end{align*}
As with the quadratic equation, the formula features the sum and the product of the roots --- but also the sum of the products of each pair. Note also the alternating signs. 
The product of the roots is
\begin{align*}
rst = 2010
\end{align*}
Since $r,s,t$ are integers, we consider all the possible combination of factors of $2010$:
\begin{align*}
2010 = 2 \times 3 \times 5 \times 67
\end{align*}
We seek the smallest possible sum formed by any three combinations of these factors. To this effect, we multiply the smallest numbers together, $2$ and $3$ and keep the other two:
\begin{align*}
6, \quad 5, \quad 67 
\quad\rightarrow\quad 
6 + 5 + 67 = 78
\end{align*}
\begin{empheq}[box={\mathbox[colback=white]}]{equation*}
    a \rightarrow 78
\end{empheq} 
\end{answer}
%%%%%%%%%%%%%%%%%%%%%%%%%%%%%%%%%%%%%%%%%%%%%%%%%%%%%%%%%%%%%%%%%%%%%%%%

\iftoggle{showAnswers}{\newpage}


%%%%%%%%%%%%%%%%%%%%%%%%%%%%%%%%%%%%%%%%%%%%%%%%%%%%%%%%%%%%%%%%%%%%%%%%
\subsection*{10.}

\nopagebreak

Let $f$ be a function for which $f(x/3) = x^2 + x + 1$. Find the sum of all values of $z$ for which $f(3z) = 7$.

\nopagebreak

\fbox{(A) $-1/3$ \quad (B) $-1/9$ \quad (C) $0$ \quad (D) $5/9$ \quad (E) $5/3$}

\begin{answer}
Let $x=9z$ and substitute:
\begin{align*}
 f(x/3) & = x^2 + x + 1 \\
\implies
f(9z/3) & = (9z)^2 + 9z + 1 \\
  f(3z) & = 81z^2 + 9z + 1 
\end{align*}
Solve the following quadratic equation for $z$:
\begin{align*}
         f(3z) & = 7 \\
81z^2 + 9z + 1 & = 7 \\
 81z^2 + 9z -6 & = 0 \\
z^2 + \frac{1}{9}z - \frac{2}{27} & = 0 
\end{align*}
By Vi\`{e}te's formula, the sum of the roots is $-1/9$. 
\begin{empheq}[box={\mathbox[colback=white]}]{equation*}
    -\frac{1}{9}
\end{empheq} 
\end{answer}
%%%%%%%%%%%%%%%%%%%%%%%%%%%%%%%%%%%%%%%%%%%%%%%%%%%%%%%%%%%%%%%%%%%%%%%%

\end{document}
