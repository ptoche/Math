\documentclass[12pt, a4]{article}
\usepackage{amsfonts}
\usepackage{mathtools}
\usepackage[colorlinks = true,
            linkcolor = blue,
            urlcolor  = blue,
            citecolor = blue,
            anchorcolor = blue]{hyperref}
\usepackage[margin=0.5in]{geometry}

\title{TP Length}
\author{for James Toche}

\begin{document}

\abstract{Professor Po-Shen Loh discussed the following problem: A toilet roll with inner circle of radius $r$ and outer circle $R$ has $300$ layers of toilet paper: What is the length of the unrolled paper? 

Loved his answer. His method was different from ours and forced us to think. 

Reference:~ 
\href{https://www.youtube.com/watch?v=0iRTzKA1VvA}{Ask Math Anything with Po-Shen Loh - 06/19 Fri}.}
\maketitle

\section*{Problem}
Professor Po-Shen Loh discussed the following problem (rephrased a bit): A toilet roll with inner circle of radius $r$ and outer circle $R$ has $300$ layers of toilet paper: What is the length of the unrolled paper? 

\section*{Approach 1: Length from Area}



\section*{Approach 2: Length from Circumference}


\end{document}