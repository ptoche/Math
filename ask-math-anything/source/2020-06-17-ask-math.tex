\documentclass[12pt]{article}
\usepackage{etoolbox}\newtoggle{showAnswers}\togglefalse{showAnswers}%
\toggletrue{showAnswers}% comment out this line to show/hide answers
\usepackage{preamble}
\usepackage{xlop}% multiplications/divisions

\title{Ask Math Anything}
\author{Daily Challenge with Po-Shen Loh}
\date{17 June 2020}

\begin{document}
\begin{minipage}{\textwidth}
\maketitle
\begin{abstract}
Professor Po-Shen Loh solves problems on his YouTube channel. A selection for practice. 

Reference:~ 
\href{https://www.youtube.com/channel/UCf78EJOm4wQ4xXwSS15PuxQ}{Ask Math Anything - Daily Challenge with Po-Shen Loh}
\end{abstract}
\end{minipage}


\section*{Squaring Large Numbers}
Calculate:
\begin{align*}
(111,111,111)^{2}
\end{align*}

\begin{answer}
Try with smaller numbers, say $(111)^{2}$:
\begin{center}
\opmul{111}{111}\qquad
\end{center}
and $(1,111)^{2}$:
\begin{center}
\opmul{1111}{1111}\qquad
\end{center}
See the pattern? Yes, the answer is:
\begin{align*}
(111,111,111)^{2} = 12,345,678,987,654,321
\end{align*}
\end{answer}

Calculate the even larger $11$-digit square of ones:
\begin{align*}
(11,111,111,111)^{2}
\end{align*}

\begin{answer}
We could write it out:
\begin{center}
\opmul{11111111111}{11111111111}\qquad
\end{center}
but that's a little insane.

Instead, we can exploit the pattern we noticed:
\begin{align*}
(11,111,111,111)^{2}
& = 123456789\\
& \hspace{5.3em} 10\\
& \hspace{5.9em} 11\\
& \hspace{6.9em} 0987654321 \\
& = 123456790120987654321 
\end{align*}
The digits that are written within the same column are carried from the 

\end{answer}


%\includegraphics[width=0.4\textwidth]{}



\end{document}