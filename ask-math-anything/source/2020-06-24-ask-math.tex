\documentclass[12pt]{article}
\usepackage{etoolbox}\newtoggle{showAnswers}\togglefalse{showAnswers}%
%\toggletrue{showAnswers}% comment out this line to show/hide answers
\usepackage{preamble}
\usepackage{longdivision}% multiplications/divisions

\title{Ask Math Anything}
\author{Daily Challenge with Po-Shen Loh}
\date{24 June 2020}

\begin{document}
\begin{minipage}{\textwidth}
\maketitle
\begin{abstract}
Professor Po-Shen Loh solves problems on his YouTube channel. A selection for practice. 

Reference:~ 
\href{https://www.youtube.com/channel/UCf78EJOm4wQ4xXwSS15PuxQ}{Ask Math Anything - Daily Challenge with Po-Shen Loh}
\end{abstract}
\end{minipage}


\section*{Lengths of Sides of a Triangle}
The lengths of the sides of a triangle are $3$ consecutive integers. The length of the shortest side is equal to $30\%$ of the perimeter. What is the length of the longest side?
\begin{center}
\begin{tikzpicture}[scale=0.5]
  \tkzDefPoint(0,0){A} 
  \tkzDefPoint(14,0){B}
  \tkzInterCC[R](A,7 cm)(B,13 cm) 
  \tkzGetPoints{C}{D}
  \tkzDrawPolygon(A,B,C)
\end{tikzpicture}
\end{center}

\begin{answer}
The available data states that the Short ($S$), Middle ($M$), and Long ($L$) sides are a certain proportion of the (unknown) perimeter $P$:
\begin{align*}
\text{Short}:~  & S = 0.3 P \\
\text{Middle}:~ & 0.3 P < M < 0.35 P\\
\text{Long}:~   & 0.3 P < L < 0.35 P
\end{align*}
The inequalities follow from the simple consideration that $M$ and $L$ must each be greater than $S=0.3P$, while $M$ and $L$ cannot have a sum greater than $0.7P$, from which it follows that $M$ certainly cannot be greater than $0.35P$. And of course $L>M$. But we also know that the sides differ by $1$, so 
\begin{align*}
M-S = L-M = 1
\end{align*}
Thus, whatever $P$ may be, $M$ must be exactly one third of the perimeter, $P/3$. This is the crucial insight that allows you to solve the problem quickly. In words, the percentage taken away by the short side must be exactly offset by the long side. To sum up,
\begin{align*}
S & = \tfrac{3}{10}~P \\
M & = \tfrac{1}{3}~P 
\end{align*}
We know that $M$ and $S$ differ by $1$. And from the above, we know that they differ by $\tfrac{1}{30}P$. That is,
\begin{align*}
M - S 
& = \tfrac{1}{3}~P - \tfrac{3}{10}~P 
= \tfrac{1}{30}~P = 1 \\
\implies \hspace{3em} P 
& = 30
\end{align*}

From this, the long side follows immediately,
\begin{align*}
L & = P - M - S \\
& = P - \frac{1}{3}~P - \frac{3}{10}~P \\
& = P \left(1 - \frac{1}{3} - \frac{3}{10}\right) \\
& = 30 \frac{30-10-9}{30}
& = 11
\end{align*}
In conclusion: The perimeter of the triangle is $30$. The long length is $11$. The middle length is $10$. The short length is $9$. Answer: \framebox{$(9, 10, 11)$}

This problem could also be solved with an equation. The problem is to find the perimeter $x$ such that:
\begin{align*}
\frac{3}{10} x + 1 = \frac{1}{3} x
\end{align*}
You may verify that this yields the same answer. It may seem to be quicker. However, it requires that you put together all the constraints in one step. And the right-hand side of the equation still must be deduced from an argument like the one above, exploiting knowledge that the middle length is the mean of the perimeter (for three lengths, the mean length is one third of the perimeter). 
\end{answer}


\section*{Base Problem}
What is the value in base $10$ of the base-$3$ number $0.121212\ldots$? 

\bigskip 

Fun fact: Base $3$ is also called the ``ternary numeral system.'' 

\begin{answer}
This base-$3$ number may be expressed in base $10$ as: 
\begin{align*}
0.121212\ldots|_{3} 
= 1 \times \frac{1}{3} + 2 \times \frac{1}{3^2} + 1 \times \frac{1}{3^3} + 2 \times \frac{1}{3^4} + \ldots 
\end{align*}
The sum on the right-hand side involves an infinite sum of fractions with powers of $3$ in the denominator that are multiplied by either $1$ or $2$. Group together the multiples of $1$ and separately the multiples of $2$, on two lines: 
\begin{align*}
0.121212\ldots|_{3} =
& ~\frac{1}{3} + \frac{1}{3^{3}} + \frac{1}{3^{5}} + \ldots 
\hspace{5em}\leftarrow S\\
& \frac{2}{3^2} + \frac{2}{3^{4}} + \frac{2}{3^{6}} + \ldots 
\hspace{5em}\leftarrow \frac{2}{3} \times S
\end{align*}
We denote the odd-powers-of-$3$ sum as $S$ and note that the sum involving even powers of $3$ is related to $S$ as follows:
\begin{align*}
\frac{2}{3} \times \left(\frac{1}{3} + \frac{1}{3^{3}} + \frac{1}{3^{5}} + \ldots\right)
= \frac{2}{3^2} + \frac{2}{3^{4}} + \frac{2}{3^{6}} + \ldots 
\end{align*}

So now to express $0.121212\ldots|_{3}$ in base-$10$, we need only calculate the infinite sum $S$. Does this sum converge? If it didn't, it would mean that the number $0.121212\ldots|_{3}$ couldn't be expressed in base $10$, which would be extraordinary. A base is a representation of a number, so all numbers that exist in one base must exist in another base.

To calculate $S$, note
\begin{align*}
S & = \frac{1}{3} + \frac{1}{3^{3}} + \frac{1}{3^{5}} + \ldots \\
  & = \frac{1}{3} + \frac{1}{3^2} \left(\frac{1}{3^{1}} + \frac{1}{3^{3}} + \ldots\right) \\
& = \frac{1}{3} + \frac{S}{9}
\end{align*}

And from this calculating $S$ is easy:
\begin{align*}
9S  & = 3 + S \\
S & = \frac{3}{8}
\end{align*}
And for the second term:
\begin{align*}
\frac{2}{3} S & = \frac{2}{3} \times \frac{3}{8} = \frac{1}{4}
\end{align*}

Now putting it together:
\begin{align*}
0.121212\ldots|_{3} 
= \frac{3}{8} + \frac{1}{4} 
= \frac{5}{8}
\end{align*}

Answer: \fbox{$0.121212\ldots|_{3} = \tfrac{5}{8}$}.
\bigskip

There is a ``nifty'' trick to get this answer in fewer steps:
\begin{align*}
0.121212\ldots|_{3} = \frac{12|_{3}}{22|_{3}} = \frac{5}{8}
\end{align*}
To get $12$ and $22$ in base $3$ requires simple mental math: 
\begin{align*}
12|_{3} & = 1 \times 3 + 2 = 5 \\
22|_{3} & = 2 \times 3 + 2 = 8
\end{align*}

But where does $\tfrac{12|_{3}}{22|_{3}}$ come from? Consider the analogy:
\begin{align*}
0.121212\ldots|_{10} & = \frac{12}{99} \\
0.121212\ldots|_{3}~ & = \frac{12|_{3}}{22|_{3}} 
\end{align*}
In base $10$, twelve divided by $99$ yields the repeated decimals $121212\ldots$. In base $3$, twelve divided by $99$ ``becomes'' twelve divided by $22$, where $22$ is the new $99$: it is the double-digit integer after which we jump to three-digit integers: In base $3$, we jump from $22$ to $100$. What is $0.010101\ldots$ in base $3$? Let
\begin{align*}
S = 0.010101\ldots|_{3} 
  = \frac{1}{9} + \frac{1}{9^2} + \frac{1}{9^3} +\ldots
\end{align*}
With similar steps as above, except easier, one gets:
\begin{align*}
S = \frac{1}{9} + \frac{S}{9} \implies 
S = 0.010101\ldots|_{3} = \frac{1}{8} 
\end{align*}
Finally,
\begin{align*}
0.121212\ldots|_{3} 
& = 12|_{3} \times 0.010101\ldots|_{3}\\
& = 12|_{3} \times \frac{1}{8} \\
& = \frac{12|_{3}}{22|_{3}}
\end{align*}
\end{answer}

This explains why it makes sense to treat division by $22|_{3}$ as division by $99$ in base $10$.

\section*{Large Mystery Number}
A $10$-digit number. The first digit is divisible by $1$. The number formed by the first two digits are divisible by $2$. The number formed by the first three digits are divisible by $3$. And so on for $10$ digits. What is the number?

\begin{answer}
\begin{align*}
1 ~\text{is divisble by}~1 \\
10 ~\text{is divisble by}~2 \\
102 ~\text{is divisble by}~3 \\
1020 ~\text{is divisble by}~4 \\
10200 ~\text{is divisble by}~5 \\
102000 ~\text{is divisble by}~6 \\
1020005 ~\text{is divisble by}~7 \\
10200056 ~\text{is divisble by}~8 \\
102000564 ~\text{is divisble by}~9 \\
1020005640 ~\text{is divisble by}~10 \\
\end{align*}
The divisbility by $7$ was a little tricky:
\begin{align*}
\longdivision{1020005}{7}
\end{align*}
Answer: \framebox{$1020005640$}.  We have found one number with the stated properties. It would be tedious to figure out exactly how many such numbers there may be. Perhaps there are none. Perhaps there are many.
\end{answer}


\section*{Special 64}
Show that $64$ is the smallest integer greater than $1$ which is both a perfect square and a perfect cube.  

\begin{answer}
Thus, $64$ satisfies the relation:
\begin{align*}
x = a^2 = b^3
\end{align*}
In prime factorization, every prime appears a multiple-of-$2$ times (perfect square). And every prime appears a multiple-of-$3$ times. A number that is both a multiple of $2$ and $3$ is a multiple of $6$. So every prime appears a multiple of $6$ times! And $2$ is the smallest prime. Indeed, we have:
\begin{align*}
64 & = \underbrace{2 \times 2 \times 2 \times 2 \times 2 \times 2}_{6~\text{times}} \\
   & = \underbrace{2 \times 2 \times 2}_{3~\text{times}} ~\times~ \underbrace{2 \times 2 \times 2}_{3~\text{times}} \\
   & = \underbrace{~2 \times 2~}_{~2~\text{times}} ~\times~ \underbrace{~2 \times 2~}_{2~\text{times}} ~\times~ \underbrace{~2 \times 2~}_{2~\text{times}} \\
\end{align*}
Answer: \fbox{$64 = 8^2 = 4^3$}

In general, any number of the form $p^{6}$, for $p$ prime has the desired property. For $p=3$, $3^{6}=729$. 
\end{answer}

\end{document}