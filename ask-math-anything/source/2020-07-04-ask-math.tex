\documentclass[12pt]{article}
\usepackage{etoolbox}\newtoggle{showAnswers}\togglefalse{showAnswers}%
\toggletrue{showAnswers}% comment out this line to show/hide answers
\usepackage{preamble}
\usepackage{xlop}% multiplications/divisions

\title{Ask Math Anything}
\author{Study at Home with Po-Shen Loh}
\date{4 July 2020}

\begin{document}
\begin{minipage}{\textwidth}
\maketitle
\begin{abstract}
Professor Po-Shen Loh solves problems on his YouTube channel. A selection for practice. 

Reference:~ 
\href{https://www.youtube.com/channel/UCf78EJOm4wQ4xXwSS15PuxQ}{Ask Math Anything - Daily Challenge with Po-Shen Loh}
\end{abstract}
\end{minipage}


\section*{A Triangle Puzzle}
A triangle has side lengths $7$, $7$, and $x$. Its area is the same as a different triangle with sides $7$, $7$, and $10$. Find x.

\textbf{Hint~1:} Area of a triangle:
\begin{align*}
\text{Area}~ 
= \frac{1}{2} \times \text{base} \times \text{height}
\end{align*}

\textbf{Hint~2:} Heron's Formula:
\begin{align*}
\text{Area}~ 
= \sqrt{s(s-a)(s-b)(s-c)}
\end{align*}
where $a$, $b$, $c$ are the lengths of the three sides of the triangle, and $s$ is the semi-perimeter,
\begin{align*}
s = \frac{a+b+c}{2}
\end{align*}

\begin{answer}
\begin{minipage}[b]{\textwidth}
\centering
\includegraphics[width=0.45\textwidth]%
{triangle-area-1}%
\includegraphics[height=0.4\textheight]%
{triangle-area-2}
\end{minipage}

Consider the first triangle. The semi-perimeter is:
\begin{align*}
s = \frac{a+b+c}{2} = \frac{7+7+10}{2} = 12
\end{align*}
Applying Heron's formula yields:
\begin{align*}
\text{Area}~ 
& = = \sqrt{s(s-a)(s-b)(s-c)} \\
& = \sqrt{12(12-7)(12-7)(12-10)} \\
& = \sqrt{12 \times 5 \times 5 \times 2}
\end{align*}
We may not need to calculate this square-root, so let's now consider the second triangle:
\begin{align*}
\text{Area}~ 
& = \sqrt{s(s-a)(s-b)(s-c)} \\
& = \sqrt{\left(7+\frac{x}{2}\right)\left(\frac{x}{2}\right)\left(\frac{x}{2}\right)\left(7-\frac{x}{2}\right)} \\
& = \sqrt{\left(\frac{x}{2}\right)^2  \left(49-\left(\frac{x}{2}\right)^2\right)}
\end{align*}

In the above, let $u=\left(\frac{x}{2}\right)^2$, we can rewrite the area as:
\begin{align*}
\text{Area}~ = \sqrt{u(49-u)}
\end{align*}

Now set the two areas equal to each other:
\begin{align*}
\text{Area}~ = \sqrt{12 \times 5 \times 5 \times 2} = \sqrt{u(49-u)} 
\end{align*}

Squaring both sides of the equality:
\begin{align*}
12 \times 5 \times 5 \times 2 & = u (49 -u) \\
u^2 - 49 u + 12 \times 5 \times 5 \times 2 & = 0\\
(u-25)(u-24) & = 0
\end{align*}
And thus we get $u=24$, or
\begin{align*}
u = \left(\frac{x}{2}\right)^2 & = 24 \\
x & = 2 \sqrt{24} \\
  & = 4 \sqrt{6}
\end{align*}

\end{answer}

\end{document}