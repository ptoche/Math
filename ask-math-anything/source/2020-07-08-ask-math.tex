\documentclass[12pt]{article}
\usepackage{etoolbox}\newtoggle{showAnswers}\togglefalse{showAnswers}%
%\toggletrue{showAnswers}% comment out this line to show/hide answers
\usepackage{preamble}
\usepackage{xlop}% multiplications/divisions

\title{Ask Math Anything}
\author{Study at Home with Po-Shen Loh}
\date{8 July 2020}

\begin{document}
\begin{minipage}{\textwidth}
\maketitle
\begin{abstract}
Professor Po-Shen Loh solves problems on his YouTube channel. A selection for practice. 

Reference:~ 
\href{https://www.youtube.com/channel/UCf78EJOm4wQ4xXwSS15PuxQ}{Ask Math Anything - Daily Challenge with Po-Shen Loh}
\end{abstract}
\end{minipage}


\section*{A Messy Factorization}
Find $3m^{2}n^{2}$ if $m^{2}+3m^{2}n^{2}=30n^{2}+517$.

\begin{answer}
This problem seems to call for factorization. To prepare for factorization, we start by rearranging terms.
\begin{align*}
3m^{2}n^{2} + m^{2} - 30n^{2}
  & = 517 \\
m^{2}(3n^{2}+1) -10(3n^{2}+1) 
  & = 507 \\
(m^{2}-10)(3n^{2}+1) 
  & = 507
\end{align*}
$507$ is obviously divisible by $3$. Furthermore
\begin{align*}
507 = 3 \cdot 169 = 3 \cdot 13^{2}
\end{align*}
We are now in a position to make some educated guesses. We can see that $13=3 \cdot 2^{2}+1$, which suggests:
\begin{align*}
\underbrace{(m^{2}-10)}_{3\cdot 13}~\underbrace{(3n^{2}+1)}_{13} = 3 \cdot 13^{2}
\end{align*}
Thus, $m^{2}=49$ and $3n^{2}=13-1=12$, and putting it together:
\begin{align*}
3m^{2}n^{2} & = 49 \times 12 \\
  & = 50 \times 12 - 12 \\
  & = 100 \times 6 -12 \\
  & = 588
\end{align*}

\end{answer}


\section*{A Cubic Equation}
How many roots does this cubic equation have? Are the roots real or imaginary?
\begin{align*}
3 x^{3} - 14 x^2 + 7x - 4 = 0
\end{align*}

\begin{answer}

\begin{theorem*}[Number of roots]
Every cubic equation has at least one real root.
\end{theorem*}
The proof of this is simple. As $x\rightarrow+\infty$, $x^{3}\rightarrow+\infty$ (and $x^3$ is the dominant term in the expression). As $x\rightarrow-\infty$, $x^{3}\rightarrow-\infty$. Therefore the graph must go through $0$. 

Since we have access to a computer, we can graph the cubic. The graph shows that the real root is close to $4$. The graph also shows that this is the only real root. 
\begin{center}
\begin{tikzpicture}
  \begin{axis}[
    xmin=-4,xmax=7,
    ymin=-60,ymax=60,
    axis lines=center,
    axis line style=<->]
    \addplot[-] expression[domain=-4:6,samples=101]{3*x^(3)-14*x^(2)+7*x-4}; 
  \end{axis}
\end{tikzpicture}
\end{center}

Without a computer and without explicitly computing the roots (which is always possible but sometimes messy), we can evaluate the cubic polynomial at sampled points (carefully chosen to minimize computations), e.g.
\begin{align*}
3(10)^{3} - 14(10)^2 + 7(10) - 4 & = 1,666 \\ 
3(1)^{3} - 14(1)^2 + 7(1) - 4 & = -8\\
3(0)^{3} - 14(0)^2 + 7(0) - 4 & = -4 \\
3(-1)^{3} - 14(-1)^2 + 7(-1) - 4 & = -28 \\
3(-10)^{3} - 14(-10)^2 + 7(-10) - 4 & = -4,474 \\
\end{align*}
If we had observed three sign changes, we would have been able to conclude that there were three real roots. Unfortunately, we can only see one sign change, which is inconclusive. However, from the sampled values above, we can see that if there were a sign change it would occur at some value between $0$ and $1$. You can see that without explicitly calculating the values at $-10$ and $10$, since it is obvious that cubic polynomials get very large at the extremes, with opposite signs. The values at $-1$, $0$, and $1$ are calculated very fast. 

One approach here (which could have been used right away) is to compute the derivative and find its root between $0$ and $1$. The derivative is the slope of the tangent at the point and indicates the point where the slope of the cubic polynomial changes. Since we know there is a slope change between $2$ and $4$ from negative to positive, we are looking for a slope change between $0$ and $1$ from positive to negative. 

The derivative is:
\begin{align*}
9 x^{2} - 28 x + 7 = 0
\end{align*}
with sign changes between $0$ and $1$:
\begin{align*}
9(0)^{2} - 28(0) + 7 & = +7 ~ > 0 \\
9(1)^{2} - 28(1) + 7 & = -12 < 0 
\end{align*}
but we already knew this. The point is that now we can calculate exactly where the slope changes and check whether the cubic crosses $0$ at that point. The roots are (steps omitted):
\begin{align*}
\frac{14}{9} \left(1 \pm \sqrt{\frac{19}{7}}\right) 
\approx 0.2741597, \hspace{1em} 2.836951 
\end{align*}

\end{answer}


\section*{An Equation Involving an Exponential}
Solve for $(x,y)$:
\begin{align*}
3^{x^{2}-2xy} & = 1 \\
x^{2} & = y + 3
\end{align*}

\begin{answer}
Consider the first equation:
\begin{align*}
3^{x^{2}-2xy} & = 1 \\
\implies \hspace{1em} 
x^{2}-2xy & = 0 \\
\implies \hspace{1em} 
x(x-2y) & = 0 \\
\end{align*}
There are two cases:
\begin{align*}
x = 0 ~~\text{or}~~ x=2y
\end{align*}
These may be combined with the second equation, which may be rewritten for clarity as
\begin{align*}
y = x^{2} - 3
\end{align*}
Thus,
\begin{align*}
x = 0 & ~~\text{and}~~ y = -3 \\
x = 2y & ~~\text{and}~~ y = x^{2} - 3
\end{align*}

The last two conditions may be combined:
\begin{align*}
y = (2y)^{2} - 3 \\
\implies \hspace{1em} 
4 y^{2} - y - 3 \\
\implies \hspace{1em} 
(4y + 3) (y - 1)
\end{align*}
These give two pairs of values for $(x,y)$:
\begin{align*}
x = -\frac{3}{2} & ~~\text{and}~~ y = -\frac{3}{4} \\
x = 2 & ~~\text{and}~~ y = 1
\end{align*}
\end{answer}


\section*{A Perfect Square}
Find a $5$-digit perfect square where some rearrangement of digits is also a perfect square. 

\begin{answer}
Example:
\begin{center}
\opmul{122}{122}\qquad
\end{center}
and:
\begin{center}
\opmul{221}{221}\qquad
\end{center}

This works because the multiplication does not introduce a carry!
\end{answer}

\end{document}