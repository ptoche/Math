We want to get rid of the square-roots. Gather one square-root on one side and square both sides of the equation: 
\begin{align*}
\left(x-\frac{1}{x}\right)^{\tfrac{1}{2}} 
  & = x - \left(1-\frac{1}{x}\right)^{\tfrac{1}{2}} \\
\left(x-\frac{1}{x}\right) 
  & = x^2 
      - 2x\left(1-\frac{1}{x}\right)^{\tfrac{1}{2}} 
      + \left(1-\frac{1}{x}\right) 
\end{align*}
Gather the square-root to one side and square again:
\begin{align*}
2x \left(1-\frac{1}{x}\right)^{\tfrac{1}{2}} 
  & = x^2 - x + 1 \\
4x^2 \left(1-\frac{1}{x}\right)
  & = (x^2 - x + 1)^2 \\
4 (x^2-x)
  & = (x^2 - x + 1)^2 
\end{align*}
Now we notice a substitution and solve for the new variable $a$:
\begin{align*}
 a & = x^2 - x \\
4a & = (a + 1)^2 \\
4a & = a^2 + 2a + 1 \\
a^2 - 2a + 1 & = 0 \\
(a-1)^2 & = 0
\end{align*}
Of course $a=1$ and therefore:
\begin{align*}
x^2 - x - 1 = 0
\end{align*}
a very famous quadratic equation, with one positive and one negative real root. The positive root is the valid root for the original problem since the sum of square-roots must be positive, and thus:
\begin{align*}
x = \varphi = \frac{1+\sqrt{5}}{2}
\end{align*}
