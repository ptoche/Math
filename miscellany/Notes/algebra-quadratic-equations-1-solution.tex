The possible values of $x$ satisfy:
\begin{align*}
x + \frac{1}{x} = -\frac{17}{4}
\end{align*}
This is a quadratic equation in disguise. Multiply through by $x$:
\begin{align*}
x^2 + 1 = -\frac{17}{4} x
\end{align*}
Rearrange to obtain the standard display format:
\begin{align*}
x^2 + \left(\sfrac{17}{4}\right) x + 1 = 0
\end{align*}
To find the values of $x$, we would solve this equation. One approach is to apply the well-known formula for the quadratic equation. Let's review the formula:
\begin{align*}
ax^2 + bx + c = 0 
\Rightarrow
x = \frac{-b \pm \sqrt{b^2-4ac}}{2a}
\end{align*}
Now substitute the values $a=1$, $b=\sfrac{17}{4}$, and $c=1$ into the above formula.

However, this is not needed. The question asks for the ``sum of all possible values of $x$''. Quadratic equation generally have two solutions (but note that these solutions could be complex / a solution could appear twice). Let $r_1$ and $r_2$ denote the roots (\textit{aka} solutions) of a quadratic equation. The roots clearly solve the following equation:
\begin{align*}
(x-r_1)(x-r_2) = 0
\end{align*}
Distributing the product and rearranging yields:
\begin{align*}
x^2 -(r_1+r_2)x + r_1 r_2 = 0
\end{align*}
Or, to put it in words:
\begin{align*}
\text{$X^2$ minus (sum)~$X$ plus (product)} = 0
\end{align*}
And thus we can read the answer to the question straight out of the equation. Just beware of the negative sign in front of the sum-of-roots term. 
\begin{empheq}[box={\mathbox[colback=white]}]{equation*}
    \text{sum of the roots}~ = -\frac{17}{4}
\end{empheq}
