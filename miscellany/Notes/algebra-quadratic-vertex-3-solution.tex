Rewrite the equation in the "vertex form":
\begin{align*}
-4x^2 + 2x + 3 
  & = -4 \left(x^2 - \frac{1}{2}x\right) + 3  \\
  & = -4 \left(x - \frac{1}{4}\right)^2 +4\left(\frac{1}{4}\right)^2 + 3 \\
  & = -4 \left(x - \frac{1}{4}\right)^2 +\frac{13}{4} 
\end{align*}
With the equation written in this way, it is obvious that $x=1/4$ annuls the square term and the maximum is therefore $13/4$. The turning point of the graph is $(1/4, 13/4)$. The range of the function is $(-\infty, 13/4)$. Its graph crosses the $x$-axis. This quadratic function has two real roots and can be factored. Indeed the ``vertex`` form is the difference of two squares: 
\begin{align*}
-4x^2 + 2x + 3 
  & = \left(\frac{\sqrt{13}}{2}\right)^2 -\left(2x - \frac{1}{2}\right)^2 \\
  & = \left(\frac{\sqrt{13}}{2} - 2x +\frac{1}{2}\right) \left(\frac{\sqrt{13}}{2} + 2x -\frac{1}{2}\right) \\
  & = 4 \left(x -\frac{1+\sqrt{13}}{4}\right) \left(x -\frac{1-\sqrt{13}}{4}\right)
\end{align*}
