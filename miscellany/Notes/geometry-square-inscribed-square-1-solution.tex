Add labels to the figure:

\begin{figure}[H]
\centering
\includegraphics[page=2,height=8cm]%
{square-inscribed}
\end{figure}

The smaller triangles with vertices corresponding with the corners of the larger square are congruent: 
That is, since $EB=FC=GD=HA$, triangles $\triangle EIB$, $\triangle FJC$, $\triangle GKD$, and $\triangle HLA$ are congruent. 

The larger triangles with side length corresponding to a side of the larger square are also congruent:
That is, since $EI=FJ=GK=HL$, triangles $\triangle ALB$, $\triangle BIC$, $\triangle CJD$, and $\triangle DKA$ are congruent. 

By symmetry it follows that the four trapezoids are also congruent. It is now clear that the smaller triangle and the trapezoids have the same area as the inscribed square. That is, for instance, $\triangle EBI$ and trapezoid $AEIL$ combine to form a square congruent with the shaded square. 

\begin{figure}[H]
\centering
\includegraphics[page=3,height=8cm]%
{square-inscribed}
\end{figure}

There are five such squares, so the shaded square covers a fraction $1/5$ of square $ABCD$.
