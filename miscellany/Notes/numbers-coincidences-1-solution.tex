The least common multiple of $3$,$4$, $5$ is $60$, implying that a cycle repeats every $60$ days. The year 2017 has $365$ days, or $6$ cycles of $60$ days and $5$ days from the start of the next cycle. The first five days have $2$ days without phone calls, to be added. If $k$ is the number of days without phone calls in one cycle of $60$ days, the total number of days without phone calls is
\begin{align*}
6 k + 2
\end{align*}
Let the first day be numbered $0$ and a cycle run from $0$ to $59$. 
To calculate $k$, we list all the days that are not multiples of $3$, $4$, $5$. That includes $1$; all the prime numbers except $3$ and $5$; and all the multiples of these prime numbers that are less than $59$.

The prime numbers less than $60$, except $3$ and $5$, are:
\begin{align*}
2, 7, 11, 13, 17, 19, 23, 29, 31, 37, 41, 43, 47, 53, 59
\end{align*}
and number $15$. 

The multiples of the prime numbers less than $60$ are:
\begin{align*}
14 (2 \cdot 7), 22 (2 \cdot 11), 26 (2 \cdot 13), 34 (2 \cdot 17), 38 (2 \cdot 19), 46 (2 \cdot 23), 48 (2 \cdot 29), 49 (7 \cdot 7)
\end{align*}
and number $8$. 

The total count is thus 
\begin{align*}
1 + 15 + 8 = 24
\end{align*}
which gives 
\begin{align*}
6 \cdot 24 + 2 = 146
\end{align*}

The first figure shows all the multiples of $3$, $4$, and $5$. 
\begin{figure}[H]
\centering
\includegraphics[page=1,height=8cm]%
{grid-numbers-colors}
\end{figure}

The second figure also shows the multiples of the prime numbers. 
\begin{figure}[H]
\centering
\includegraphics[page=2,height=8cm]%
{grid-numbers-colors}
\end{figure}
