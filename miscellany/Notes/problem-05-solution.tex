We notice that $4$ and $5$ can be factored
\begin{flalign*}
16^{x} + 20^{x}
& = 25^{x}
\\[1ex]
4^{2x} + 5^{x} \cdot 4^{x}
& = 5^{2x}
\\[1ex]
1 + \frac{5^{x}}{4^{x}} \cdot \frac{4^{x}}{4^{x}}
& = \frac{5^{2x}}{4^{2x}}
\\[1ex]
1 + (5/4)^{x} 
& = (5/4)^{2x}
\end{flalign*}
Let $X=(5/4)^{x}$ and solve for $X>0$:
\begin{flalign*}
X^{2} - X - 1
& = 0
\implies
X = \frac{1 + \sqrt{5}}{2}
\end{flalign*}
Now solve for $x$ by taking the logarithm on both sides of the equality:
\begin{flalign*}
\left(\frac{5}{4}\right)^{x} 
& = \frac{1 + \sqrt{5}}{2}
\implies
x \ln\left(\frac{5}{4}\right)
  = \ln\left(1+\sqrt{5}\right)-\ln(2)
\implies
x = \frac{\ln(1+\sqrt{5})-\ln(2)}{\ln(5/4)}
&
\end{flalign*}
Check the plausibility of the solution with coarse approximations:
% (log(1+sqrt(5))-log(2))/log(5/4)
% 2.156512
% (1.2-0.69)/(1.61-2*0.69)
% 2.217391
\begin{flalign*}
&
\ln(2) \approx 0.69, \
\ln(3) \approx 1.10, \
\ln(5) \approx 1.61, \
\sqrt{5} \approx 2.24,
\implies \ln(1+\sqrt{5}) \approx \ln\left(3.24\right) \approx 1.2
& \\[1ex]
& \frac{\ln(1+\sqrt{5})-\ln(2)}{\ln(5)-2\ln(2)} 
\approx \frac{1.2-0.69}{1.61 - 2 \times 0.69} 
\approx 2.2 
& \\[1ex]
& 16^{2.2} + 20^{2.2} \approx 1174
& \\[1ex]
& 25^{2.2} \approx 1190
&
% 16^(2.2) + 20^(2.2)
% 1173.948
% 25^(2.2)
% 1189.784
\end{flalign*}
