\documentclass[12pt]{article}
\usepackage[a4paper, centering, height=250mm, width=180mm, noheadfoot]{geometry}
\usepackage{amsfonts}
\usepackage[fleqn]{amsmath}
\usepackage{mathtools}
\usepackage{bm}% \boldsymbol

\pagenumbering{gobble}

\title{Math Bits}
\author{James \& Patrick Toche}

\begin{document}
\maketitle

\newpage
\section*{First Few Squares}
\begin{align*}
1^2   & = 1\\ 
2^2   & = 4\\ 
3^2   & = 9\\ 
4^2   & = 16\\ 
5^2   & = 25\\ 
6^2   & = 36\\ 
7^2   & = 49\\
8^2   & = 64\\ 
9^2   & = 81\\ 
10^2  & = 100\\
11^2  & = 121\\
12^2  & = 144\\
13^2  & = 169\\
14^2  & = 196\\
15^2  & = 225\\
16^2  & = 256\\
17^2  & = 289\\
18^2  & = 324\\
19^2  & = 361\\
20^2  & = 400\\
21^2  & = 441\\
22^2  & = 484\\
23^2  & = 529\\
24^2  & = 576\\
25^2  & = 625\\
30^2  & = 900\\
35^2  & = 1225\\
40^2  & = 1600\\
45^2  & = 2025\\
50^2  & = 2500\\
55^2  & = 3025\\
60^2  & = 3600 \\
65^2  & = 4225\\
70^2  & = 4900\\
75^2  & = 5625\\
80^2  & = 6400\\
85^2  & = 7225\\
90^2  & = 8100\\
95^2  & = 9025\\
100^2 & = 10000
\end{align*}

\newpage
\section*{Useful Cubes}
\begin{align*}
 2^3  & = 8\\
 3^3  & = 27\\
 4^3  & = 64\\
 5^3  & = 125\\
 6^3  & = 216\\
 7^3  & = 343\\
 8^3  & = 512\\
 9^3  & = 729\\
10^3 & = 1000\\
11^3 & = 1331\\
12^3 & = 1728
\end{align*}

\section*{Useful Fourth Powers}
\begin{align*}
2^4  & = 16\\
3^4  & = 81\\
4^4  & = 256\\
5^4  & = 625\\
6^4  & = 1296\\
7^4  & = 2401\\
8^4  & = 4096\\
9^4  & = 6561\\
10^4 & = 10000
\end{align*}

\section*{More Useful Powers}
\begin{align*}
2^2    & = 4\\
2^3    & = 8\\
2^4    & = 16\\
2^5    & = 32\\
2^6    & = 64\\
2^7    & = 128\\
2^8    & = 256\\
2^9    & = 512\\
2^{10} & = 1024\\
2^{11} & = 2048\\
2^{12} & = 4096\\
\end{align*}


\newpage
\section*{Highly Composite Numbers}
A positive integer with more divisors than any smaller positive integer has.  Plato set $5,040$ as the ideal number of citizens in a city. In bold, superior highly composite numbers.  
\begin{align*}
& \bm{2}       &&	2\\
& 4 	         && 2^{2}\\
& \bm{6}       && 2 \cdot 3\\
& \bm{12}      && 2^{2} \cdot 3\\
& 24 	         && 2^{3} \cdot 3 \\
& 36 	         && 2^{2} \cdot 3^{2} \\
& 48 	         && 2^{4} \cdot 3 \\
& \bm{60}	     && 2^{2} \cdot 3 \cdot 5\\
& \bm{120}     && 2^{3} \cdot 3 \cdot 5\\
& 180          && 2^{2} \cdot 3^{2} \cdot 5\\
& 240 	       && 2^{4} \cdot 3 \cdot 5\\
& \bm{360}     && 2^{3} \cdot 3^{2} \cdot 5\\
& 720 	       && 2^{4} \cdot 3^{2} \cdot 5\\
& 840 	       && 2^{3} \cdot 3 \cdot 5 \cdot 7\\
& 1,260 	     && 2^{2} \cdot 3^{2} \cdot 5 \cdot 7\\
& 1,680 	     && 2^{4} \cdot 3^{1} \cdot 5 \cdot 7\\
& \bm{2,520}   && 2^{3} \cdot 3^{2} \cdot 5 \cdot 7\\
& \bm{5,040}   && 2^{4} \cdot 3^{2} \cdot 5 \cdot 7\\
& 7,560 	     && 2^{3} \cdot 3^{3} \cdot 5 \cdot 7\\
& 10,080       && 2^{5} \cdot 3^{2} \cdot 5 \cdot 7\\
& 15,120	     && 2^{4} \cdot 3^{3} \cdot 5 \cdot 7\\
& 20,160	     && 2^{6} \cdot 3^{2} \cdot 5 \cdot 7\\
& 25,200	     && 2^{4} \cdot 3^{2} \cdot 5^{2} \cdot 7\\
& 27,720	     && 2^{3} \cdot 3^{2} \cdot 5 \cdot 7 \cdot 11\\
& 45,360	     && 2^{4} \cdot 3^{4} \cdot 5 \cdot 7\\
& 50,400	     && 2^{5} \cdot 3^{2} \cdot 5^{2} \cdot 7\\
& \bm{55,440}  && 2^{4} \cdot 3^{2} \cdot 5 \cdot 7 \cdot 11\\
& 83,160	     && 2^{3} \cdot 3^{3} \cdot 5 \cdot 7 \cdot 11\\
& 110,880	     && 2^{5} \cdot 3^{2} \cdot 5 \cdot 7 \cdot 11\\
& 166,320	     && 2^{4} \cdot 3^{3} \cdot 5 \cdot 7 \cdot 11\\
& 221,760	     && 2^{6} \cdot 3^{2} \cdot 5 \cdot 7 \cdot 11\\
& 277,200	     && 2^{4} \cdot 3^{2} \cdot 5^{2} \cdot 7 \cdot 11\\
& 332,640	     && 2^{5} \cdot 3^{3} \cdot 5 \cdot 7 \cdot 11\\
& 498,960	     && 2^{4} \cdot 3^{4} \cdot 5 \cdot 7 \cdot 11\\
& 554,400	     && 2^{5} \cdot 3^{2} \cdot 5^{2} \cdot 7 \cdot 11\\
& 665,280	     && 2^{6} \cdot 3^{3} \cdot 5 \cdot 7 \cdot 11\\
& \bm{720,720} && 2^{4} \cdot 3^{2} \cdot 5 \cdot 7 \cdot 11 \cdot 13 
\end{align*}


\section*{Superior Highly Composite Numbers}
\begin{align*}
& 2       &&	2\\
& 6 	    && 2 \cdot 3\\
& 12 	    && 2^{2} \cdot 3\\
& 60 	    && 2^{2} \cdot 3 \cdot 5\\
& 120 	  && 2^{3} \cdot 3 \cdot 5\\
& 360 	  && 2^{3} \cdot 3^{2} \cdot 5\\
& 2,520 	&& 2^{3} \cdot 3^{2} \cdot 5 \cdot 7\\
& 5,040 	&& 2^{4} \cdot 3^{2} \cdot 5 \cdot 7\\
& 55,440  &&	2^{4} \cdot 3^{2} \cdot 5 \cdot 7 \cdot 11\\
& 720,720 &&	2^{4} \cdot 3^{2} \cdot 5 \cdot 7 \cdot 11 \cdot 13\\
\end{align*}
%\ldots $1,441,440$; $4,324,320$; $21,621,600$; $367,567,200$; $6,983,776,800$\ldots 



\section*{Perfect numbers}
A number equal to the sum of its proper divisors.
\begin{align*}
& 6              && 2^{1}(2^{2} -1) \\
& 28             && 2^{2}(2^{3} -1) \\
& 496            && 2^{4}(2^{5} -1) \\
& 8128           && 2^{6}(2^{7} -1) \\
& 33550336       && 2^{12}(2^{13} -1) \\
& 8589869056     && 2^{16}(2^{17} -1) \\
& 137438691328   && 2^{18}(2^{19} -1) \\
\end{align*}
Euclid's Perfect Number Theorem: If $2^{p}-1$ is prime, then $2^{p-1}(2^{p}-1)$ is perfect. Euler's Perfect Number Theorem: If $n$ is an even perfect number, it is of the form $n=2^{p-1}(2^{p}-1)$ for some prime $p$ and Mersenne prime $2^{p}-1$.


\section*{Prime Quadruplets}
The first eight prime quadruplets are: 
\begin{align*}
5, 7, 11, 13\\ 
11, 13, 17, 19\\ 
101, 103, 107, 109\\ 
191, 193, 197, 199\\
821, 823, 827, 829\\
1481, 1483, 1487, 1489\\
1871, 1873, 1877, 1879\\
2081, 2083, 2087, 2089\\ 
\end{align*}


\section*{Prime Quintuplets}
The first few quintuplets that are not also sextuplets are: 
\begin{align*}
5, 7, 11, 13, 17\\ 
1481, 1483, 1487, 1489, 1493\\
3457, 3461, 3463, 3467, 3469\\
5647, 5651, 5653, 5657, 5659\\
15727, 15731, 15733, 15737, 15739\\
21011, 21013, 21017, 21019, 21023\\
22271, 22273, 22277, 22279, 22283\\
55331, 55333, 55337, 55339, 55343\\
79687, 79691, 79693, 79697, 79699\\
88807, 88811, 88813, 88817, 88819\\  
\end{align*}


\section*{Prime Sextuplets}
The first five sextuplets are: 
\begin{align*}
7, 11, 13, 17, 19, 23\\
97, 101, 103, 107, 109, 113\\
16057, 16061, 16063, 16067, 16069, 16073\\
19417, 19421, 19423, 19427, 19429, 19433\\
43777, 43781, 43783, 43787, 43789, 43793\\
\end{align*}



\section*{Mersenne Primes}
Prime numbers of the form $2^{p}-1$, where $p$ is some prime number. Examples:
\begin{align*}
&  p && 2^{p}-1\\
&  2 && 3\\
&  3 && 7\\
&  5 && 31\\
&  7 && 127\\
& 11 && 23 \times 89\\
& 13 && 8,191\\
& 17 && 131,071 \\
& 19 && 524,287 \\
& 23 && 47 \times 178481 \\
\end{align*}
$51$ Mersenne primes are known. The largest known prime number is a Mersenne prime with $p=82,589,933$. 


\newpage
\section*{Rectangle}
Any rectangle of sides $a$ and $b$:
\begin{align*}
P & = 2 \times (a + b)\\
A & = a \times b
\end{align*}

\section*{Triangle}
Any triangle of height $h$ and base $b$:
\begin{align*}
A = \frac{1}{2} (h \times b)
\end{align*}

\section*{Pythagorean Theorem}
For any right-triangle with hypotenuse length $c$ and legs $a$, $b$:
\begin{align*}
c^2 = a^2 + b^2 
\end{align*}

\noindent Example: \vspace{-0.5em}
\begin{align*}
5^2 = 3^2 + 4^2 
\end{align*}

\section*{Pythagorean Triples}
Famous Pythagorean triple: 
\begin{align*}
(3, 4, 5)
\end{align*}

\noindent More triples:
\begin{align*}
(5, 12, 13)~~
(8, 15, 17)~~
(7, 24, 25)\\
(20, 21, 29)~~	
(12, 35, 37)~~	
(9, 40, 41)\\ 	
(28, 45, 53)~~
(11, 60, 61)~~ 
(16, 63, 65)\\
(33, 56, 65)~~ 	
(48, 55, 73)~~
(13, 84, 85)\\
(36, 77, 85)~~
(39, 80, 89)~~
(65, 72, 97) 
\end{align*}

\noindent Non-Pythagorean triple: $(1,1,\sqrt{2})$  ($\sqrt{2}$ is irrational!)

\subsection*{Circle}
Any circle of radius $r$ (or diameter $d = 2r$):
\begin{align*}
P & = \tau r = 2\pi r = \pi 2 r = \pi d ~~ \approx 6.28 r\\
A & = \pi r^2 = \pi \left(\frac{d}{2}\right)^2 ~~ \approx 3.14 r^2 
\end{align*}


\subsection*{Interesting Natural Numbers}
\begin{align*}
3435 = 3^{3} + 4^{4} + 3^{3} + 5^{5}
\end{align*}


\subsection*{Interesting Irrational Numbers}
% options(digits=21)
\begin{align*}
     \pi & \approx 3.14159 26535 89793 11600 \ldots \\
 \varphi & \approx 1.61803 39887 49894 90253 \ldots \\
    \psi & \approx 3.35988 56662 43177 55317 \ldots \\
\sqrt{2} & \approx 1.41421 35623 73095 14548 \ldots \\
\sqrt{3} & \approx 1.73205 08075 68877 19318 \ldots \\
\sqrt{5} & \approx 2.23606 79774 99789 80505 \ldots \\
\sqrt{6} & \approx 2.44948 97427 83177 88134 \ldots \\
\sqrt{7} & \approx 2.64575 13110 64590 71617 \ldots \\
\sqrt{8} & \approx 2.82842 71247 46190 29095 \ldots \\
       e & \approx 2.71828 18284 59045 09080 \ldots \\
  \gamma & \approx 0.57721 56649 01532 86061 \ldots
\end{align*}



\subsection*{More Interesting Numbers}
\begin{align*}
    \tau & = 2 \pi \\
 \varphi & = \frac{\sqrt{5}}{2} \\
    \psi & = \frac{1}{\varphi} \\
   F_{n} & = \frac{\varphi^{n}-(1-\varphi)^{n}}{\sqrt{5}} \\
       e & = \frac{1}{1} + \frac{1}{1} + \frac{1}{1 \times 2} + \frac{1}{1 \times 2 \times 3} + \frac{1}{1 \times 2 \times 3 \times 4} + \ldots 
         = \lim_{n\rightarrow\infty}\left(1+\frac{1}{n}\right)^{n} \\
       i & = \sqrt{-1}\\
\text{googol} 
         & = 10^{100} \\
\text{googolplex} 
         & = 10^{\text{googol}} = 10^{10^{100}} \\
C_{10}
         & = 0.123456789101112131415116171819202122232425\ldots \\
\end{align*}



\subsection*{Riemann zeta Numbers}
\begin{align*}
 \zeta(s) 
  & = \frac{1}{1^{s}} + \frac{1}{2^{s}} + \frac{1}{3^{s}} + \frac{1}{4^{s}} + \frac{1}{5^{s}} + \ldots \\
 \zeta(1) 
  & = \frac{1}{1} + \frac{1}{2} + \frac{1}{3} + \frac{1}{4} + \ldots \rightarrow \infty \\
 \zeta(2) 
  & = \frac{1}{1^{2}} + \frac{1}{2^{2}} + \frac{1}{3^{2}} + \frac{1}{4^{2}} + \frac{1}{5^{2}} + \ldots \approx 1.64493 40668 48226 43647 \\
 \zeta(3) 
  & = \frac{1}{1^{3}} + \frac{1}{2^{3}} + \frac{1}{3^{3}} + \frac{1}{4^{3}} + \frac{1}{5^{3}} + \ldots \approx 1.20205 69031 59594 28540 \\
 \zeta(4) 
  & = \frac{1}{1^{4}} + \frac{1}{2^{4}} + \frac{1}{3^{4}} + \frac{1}{4^{4}} + \frac{1}{5^{4}} + \ldots \approx 1.08232 32337 11138 19152 \\
 \zeta(1/2) 
  & = \frac{1}{\sqrt{1}} + \frac{1}{\sqrt{2}} + \frac{1}{\sqrt{3}} + \frac{1}{\sqrt{4}} + \frac{1}{\sqrt{5}} + \ldots  -1.46035 45088 09586 81289 \\
 \zeta(0) 
  & = 1 + 1 + 1 + 1 + 1 + \ldots = -\frac{1}{2} \\
\zeta(-1) 
  & = 1 + 2 + 3 + 4 + 5 + \ldots = -\frac{1}{12} \\
\end{align*}



\end{document}