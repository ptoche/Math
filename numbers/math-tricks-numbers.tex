\documentclass[12pt]{article}
\usepackage[a4paper, centering, height=250mm, width=180mm, noheadfoot]{geometry}
\usepackage{amsfonts}
\usepackage[fleqn]{amsmath}
\usepackage{mathtools}
\usepackage{booktabs, cellspace, multirow}
\usepackage{array}% \newcolumntype macro
\newcolumntype{X}{>{$}l<{$}}
\newcolumntype{Y}{>{$}c<{$}}
\newcolumntype{Z}{>{$}r<{$}}
\newcolumntype{L}[1]{>{\raggedright\let\newline\\\arraybackslash\hspace{0pt}}m{#1}}
\newcolumntype{C}[1]{>{\centering\let\newline\\\arraybackslash\hspace{0pt}}m{#1}}
\newcolumntype{R}[1]{>{\raggedleft\let\newline\\\arraybackslash\hspace{0pt}}m{#1}}

\usepackage{bm}% \boldsymbol
\setlength\parindent{0pt}% no indents
\usepackage{parskip}
\setlength{\parskip}{0.75\baselineskip plus 2pt}
\usepackage{tikz}
\usetikzlibrary{decorations.text}

\newcommand{\tikzmark}[1]{\tikz[baseline={(#1.base)},remember picture] \node[draw=red,thick,circle,inner sep=0.5pt,minimum width=3ex] (#1) {$#1$};}
\tikzset{connect/.style={
    red,in=225,out=315,thick}}

\pagenumbering{gobble}

\title{Math Competition Tricks}
\author{Clairbourn School Grade 7/8}

\begin{document}
\maketitle
\begin{minipage}{\textwidth}
\maketitle
\begin{abstract}
This note reviews a selection of tricks that may be useful in math competitions.
\end{abstract}
\end{minipage}

\newpage

\section*{Mental Arithmetic}
It is useful to add/multiply/divide fast. There are too many tricks to review, but here are a few basic ones. With practice you will be able to use these tricks while calculating in your head. 

Add numbers by grouping them:

\begin{minipage}{\textwidth}
\begin{align*}
  \tikzmark{1} + \tikzmark{2} + \tikzmark{3} + \tikzmark{4} + 5 + \tikzmark{6} + \tikzmark{7} + \tikzmark{8} + \tikzmark{9}
  \begin{tikzpicture}[overlay,remember picture]
    \draw[-,connect] (1) to (9);
    \draw[-,connect] (2) to (8);
    \draw[-,connect] (3) to (7);
    \draw[-,connect] (4) to (6);
  \end{tikzpicture}
= 4 \times 10 + 5
= 45
\end{align*}
\vspace{4ex}
\end{minipage}

Add numbers by rounding them:
\begin{align*}
978 + 237 = (980-2) + (220+17) = 1200 + 15 = 1215
\end{align*}


Multiply by $5$:
\begin{align*}
978 \times 5 
 = (1000-20-2) \times \frac{10}{2} 
 = \frac{500-10-1}{2} \times 10
 = 4890
\end{align*}
where we have also decomposed $978$ to make the division by $2$ even easier (skip this step if you can quickly halve $978$).


Multiply numbers by decomposing them:
\begin{align*}
14 \times 16
 & = (15-1) \times (15+1) \\
 & = 15^2 -1 = 224
\end{align*}
where we have used $15^2=225$ and the difference-of-squares formula:
\begin{align*}
(a+b) (a-b) = a^2 - b^2
\end{align*}

Similarly, $13 \times 17$ $=$ $15^2 -4$ $=$ $221$ (if hesitant, check that the last digit matches: $3\times7=21$, so the last digit $1$ is indeed correct).
The difference-of-squares formula can always be applied when multiplying numbers that differ by a multiple of $2$ (multiplying two even numbers or multiplying two odd numbers). 

Multiply numbers by rounding up:
\begin{align*}
19 \times 18
 & = 20 \times 18 - 18\\
 & = 360 - 18 = 342
\end{align*}

Multiply numbers by rounding up and down:
\begin{align*}
19 \times 23
 & = (20-1) \times (20+3) \\
 & = 20^2 + (3-1) \times 20 -3 = 400 + 40 - 3 = 437
\end{align*}

Square numbers by rounding up:
\begin{align*}
99^2 
 & = (100-1)^2 \\
 & = 10000 - 200 + 1 = 9801
\end{align*}
where we have used:
\begin{align*}
(a + b)^2 = a^2 + 2ab + b^2
\end{align*}

Square numbers by rounding up or down:
\begin{align*}
13^2 
 & = (15-2)^2 \\
 & = 200 + 25 - 60 + 4 = 140 + 29 = 169
\end{align*}
where we suppose you have memorized $15^2=225=200+25$ (but forgotten $13^2$). Because it is easier to subtract $60$ from $200$ than from $225$, we also split $225$ as $200+25$. These manipulations are to be done in your head or very quickly on a scrap of paper. 

\section*{Useful Sums}
The sum of the first $n$ natural numbers:
\begin{align*}
& 1+2+3+\ldots+n = \frac{n(n+1)}{2} \\
& 1+2+3+\ldots+10 = 55 \\
& 1+2+3+\ldots+100 = 505
\end{align*}

The sum of the first $(2n-1)$ odd numbers:
\begin{align*}
& 1+3+5+\ldots+(2n-1) = n^2 \\
& 1+3+5+\ldots+9 = 1+3+5+\ldots+(2\times5-1) = 5^2 = 25 \\
& 1+3+5+\ldots+99 = 1+3+5+\ldots+(2\times50-1) = 50^2 =  2500
\end{align*}

The sum of the first $2n$ even numbers:
\begin{align*}
& 2+4+6+\ldots+(2n) = n(n+1) \\
& 2+4+6+\ldots+10 = 2+4+6+\ldots+(2\times5) = 5 \times 6 = 30 \\
& 2+4+6+\ldots+100 = 2+4+6+\ldots+(2\times50) = 50\times51 = 2550
\end{align*}

The sum of the first $n$ squares formula and first ten sums:
\begin{align*}
& 1^2+2^2+\ldots+n^2 = \frac{n(n+1)(2n+1)}{6} \\
& 1, 5, 14, 30, 55, 91, 140, 204, 285, 385.
\end{align*}

The sum of the first $n$ cubes formula and first ten sums:
\begin{align*}
& 1^3+2^3+\ldots+n^3 = \left(\frac{n(n+1)}{2}\right)^{3} \\
& 1, 9, 36, 100, 225, 441, 784, 1296, 2025, 3025.
\end{align*}


The sum of $n$ terms of a geometric series:
\begin{align*}
1 + a + a^2 + \ldots + a^n
 = \frac{1-a^n}{1-a}
\end{align*}


The Fibonacci numbers are the sum of the two preceding numbers in the Fibonacci sequence:
\begin{align*}
F_{n} = F_{n-1} + F_{n-2}
\end{align*}
where the first two numbers in the sequence are typically $0$ and $1$:
\begin{align*}
0, 1 , 1, 2, 3, 5, 8, 13, 21, 34, 55, 89, 144, \ldots
\end{align*}
The Lucas numbers are Fibonacci numbers with starting values $2$ and $0$:
\begin{align*}
2, 1, 3, 4, 7, 11, 18, 29, 47, 76, 123, 199, \ldots
\end{align*}


\section*{Useful Products}
These factorial products are worth remembering:
\begin{align*}
 3! & = 6 \\
 4! & = 24 \\
 5! & = 120 \\
 6! & = 720 \\
 7! & = 5040 \\
 8! & = 40320 \\
 9! & = 362880 \\
10! & = 3628800 
\end{align*}

\newpage

\section*{Prime Numbers}
The first few prime numbers with their index:

\begin{minipage}{\textwidth}
\centering
\resizebox{\columnwidth}{!}{%
\begin{tabular}{X|ZZZZZZZZZZZZZZZZZZZZZ}
i &   1 &   2 &   3 &   4 &   5 &   6 &   7 &   8 &   9 &  10
  &  11 &  12 &  13 &  14 &  15 &  16 &  17 &  18 &  19 &  20 \\
p &   2 &   3 &   5 &   7 &  11 &  13 &  17 &  19 &  23 &  29
  &  31 &  37 &  41 &  43 &  47 &  53 &  59 &  61 &  67 &  71 \bigskip\\
i &  21 &  22 &  23 &  24 &  25 &  26 &  27 &  28 &  29 &  30
  &  31 &  32 &  33 &  34 &  35 &  36 &  37 &  38 &  39 &  40 \\
p &  73 &  79 &  83 &  89 &  97 & 101 & 103 & 107 & 109 & 113
  & 127 & 131 & 137 & 139 & 149 & 151 & 157 & 163 & 167 & 173 \bigskip\\
i &  41 &  42 &  43 &  44 &  45 &  46 &  47 &  48 &  49 &  50
  &  51 &  52 &  53 &  54 &  55 &  56 &  57 &  58 &  59 &  60 \\
p & 179 & 181 & 191 & 193 & 197 & 199 & 211 & 223 & 227 & 229
  & 233 & 239 & 241 & 251 & 257 & 263 & 269 & 271 & 277 & 281 \\
\end{tabular}}
\end{minipage}

Prime numbers of the twentieth and twenty-first centuries (problems involving numbers close to the current year are popular: the closest prime numbers to $2020$ are $2017$ and $2027$):
\begin{align*}
& 1901, 1907, 1913, 1931, 1933, 1949, 1951, 1973, 1979, 1987, 1993, 1997, 1999, \\ 
& 2003, 2011, 2017, 2027, 2029, 2039, 2053, 2063, 2069, 2081, 2083, 2087, 2089, 2099
\end{align*}
Mersenne primes are prime numbers of the form $2^{p}-1$, for some prime number $p$. The first few Mersenne primes are:
\begin{center}
\begin{tabular}{X|XXXXXXXXX}
p       & 2 & 3 & 5  & 7   & 11   & 13   & 17     & 19     & 31 \\
2^{p}-1 & 3 & 7 & 31 & 127 & 8191 & 131071 & 524287 & 2147483647 \\
\end{tabular}
\end{center}

Some Mersenne numbers that are not prime include:
\begin{center}
\begin{tabular}{X|XXX}
p       & 11                  & 23                          & 29 \\
2^{p}-1 & 2047 = 23 \times 89 & 8388607  = 47 \times 178481 & 536870911 = 233 \times 1103 \times 2089 \\
\end{tabular}
\end{center}
     
Fermat primes are prime numbers of the form $2^{2^{n}}+1$. There are only five known Fermat primes:
\begin{center}
\begin{tabular}{X|XXXXXXXXX}
n           & 0 & 1 & 2  & 3   & 4 \\
2^{2^{n}}+1 & 3 & 5 & 17 & 257 & 65537 \\
\end{tabular}
\end{center}
The famous mathematician Euler showed that
\begin{align*}
2^{2^{5}}+1 = 2^{32}+1 = 4294967297 = 641 \times 6700417
\end{align*}

Some Fibonacci numbers are prime. Here are the first few:
\begin{align*}
2, 3, 5, 13, 89, 233, 1597, 28657
\end{align*}

\section*{Useful Squares}
These are the first ten squares: $1, 4, 9, 16, 25, 36, 49, 64, 81, 100$. Here are a few more:
\begin{align*}
11^2  & = 121\\
12^2  & = 144\\
13^2  & = 169\\
14^2  & = 196\\
15^2  & = 225\\
16^2  & = 256\\
17^2  & = 289\\
18^2  & = 324\\
19^2  & = 361\\
20^2  & = 400\\
21^2  & = 441\\
22^2  & = 484\\
23^2  & = 529\\
24^2  & = 576\\
25^2  & = 625\\
30^2  & = 900\\
35^2  & = 1225\\
40^2  & = 1600\\
45^2  & = 2025\\
50^2  & = 2500\\
55^2  & = 3025\\
60^2  & = 3600 \\
65^2  & = 4225\\
70^2  & = 4900\\
75^2  & = 5625\\
80^2  & = 6400\\
85^2  & = 7225\\
90^2  & = 8100\\
95^2  & = 9025\\
100^2 & = 10000
\end{align*}

\newpage

\section*{Useful Cubes}
\begin{align*}
2^3  & = 8\\
3^3  & = 27\\
4^3  & = 64\\
5^3  & = 125\\
6^3  & = 216\\
7^3  & = 343\\
8^3  & = 512\\
9^3  & = 729\\
10^3 & = 1000\\
11^3 & = 1331\\
12^3 & = 1728\\
13^3 & = 2197\\
14^3 & = 2744\\
15^3 & = 3375\\
16^3 & = 4096\\
17^3 & = 4913\\
18^3 & = 5832\\
19^3 & = 6859\\
20^3 & = 8000
\end{align*}

\section*{Useful Fourth Powers}
\begin{align*}
2^4  & = 16\\
3^4  & = 81\\
4^4  & = 256\\
5^4  & = 625\\
6^4  & = 1296\\
7^4  & = 2401\\
8^4  & = 4096\\
9^4  & = 6561\\
10^4 & = 10000
\end{align*}

\section*{More Useful Powers}
Memorizing powers can come in handy:
\begin{align*}
2^2    & = 4\\
2^3    & = 8\\
2^4    & = 16\\
2^5    & = 32\\
2^6    & = 64\\
2^7    & = 128\\
2^8    & = 256\\
2^9    & = 512\\
2^{10} & = 1024\\
2^{11} & = 2048\\
2^{12} & = 4096\\
2^{13} & = 8192\\
2^{14} & = 16384\\
2^{15} & = 32768
\end{align*}

Powers of $3$:
\begin{align*}
3^2    & = 9\\
3^3    & = 27\\
3^4    & = 81\\
3^5    & = 243\\
3^6    & = 729\\
3^7    & = 2187\\
3^8    & = 6561\\
3^9    & = 19683\\
3^{10} & = 59049
\end{align*}

Powers of $5$:
\begin{align*}
5^2    & = 25\\
5^3    & = 125\\
5^4    & = 625\\
5^5    & = 3125\\
5^6    & = 15625\\
5^7    & = 78125\\
5^8    & = 390625
\end{align*}

Powers of $6$:
\begin{align*}
6^2    & = 36\\
6^3    & = 216\\
6^4    & = 1296\\
6^5    & = 7776
\end{align*}

Powers of $7$:
\begin{align*}
7^2    & = 49\\
7^3    & = 343\\
7^4    & = 2401\\
7^5    & = 16807
\end{align*}

Powers of $11$:
\begin{align*}
11^2    & = 121\\
11^3    & = 1331\\
11^4    & = 14641\\
11^5    & = 161051
\end{align*}

\newpage

\section*{Useful Conversion Rates}

\bigskip 
\begin{center}
\renewcommand{\arraystretch}{1.5}
\begin{tabular}{L{1.1cm}L{0.8cm}L{1.8cm}}
\toprule
$1$\,ft       & $=$    & $12$\,in \\
$1$\,yd       & $=$    & $3$\,ft \\
$1$\,in       & $=$    & $2.54$\,cm \\
$1$\,m        & $=$    & $3.28$\,ft \\
$1$\,mi       & $=$    & $1760$\,yd \\
$1$\,mi       & $=$    & $1609$\,m \\
\midrule
$1$\,ft$^2$   & $=$    & $144$\,in$^2$ \\
$1$\,yd$^2$   & $=$    & $9$\,ft$^2$ \\
$1$\,mi$^2$   & $=$    & $2.59$\,km$^2$ \\
$1$\,mi$^2$   & $=$    & $640$\,ac$^2$ \\
$1$\,ha       & $=$    & $2.47$\,ac \\
\midrule
$1$\,ft$^3$   & $=$    & $1728$\,in$^3$ \\
$1$\,yd$^3$   & $=$    & $27$\,ft$^3$ \\
$1$\,in$^3$   & $=$    & $16.39$\,cm$^3$ \\
$1$\,m$^3$    & $=$    & $35.31$\,ft$^3$ \\
\midrule
$1$\,qt       & $=$    & $2$~pt \\
$1$\,pt       & $=$    & $16$\,fl\,oz \\
$1$\,gal      & $=$    & $128$\,fl\,oz \\
$1$\,gal      & $=$    & $4$\,qt \\
$1$\,pt       & $=$    & $473.18$\,ml \\
$1$\,fl\,oz   & $=$    & $29.57$\,ml \\
\bottomrule
\end{tabular}
\end{center}

\newpage

\section*{Highly Composite Numbers}
A positive integer that has more divisors than any smaller positive integer. A selection:
\begin{align*}
& 2            &&	2\\
& 4 	         && 2^{2}\\
& 6            && 2 \cdot 3\\
& 12           && 2^{2} \cdot 3\\
& 24 	         && 2^{3} \cdot 3 \\
& 36 	         && 2^{2} \cdot 3^{2} \\
& 48 	         && 2^{4} \cdot 3 \\
& 60	         && 2^{2} \cdot 3 \cdot 5\\
& 120          && 2^{3} \cdot 3 \cdot 5\\
& 180          && 2^{2} \cdot 3^{2} \cdot 5\\
& 240 	       && 2^{4} \cdot 3 \cdot 5\\
& 360          && 2^{3} \cdot 3^{2} \cdot 5\\
& 720 	       && 2^{4} \cdot 3^{2} \cdot 5\\
& 840 	       && 2^{3} \cdot 3 \cdot 5 \cdot 7\\
& 1,260 	     && 2^{2} \cdot 3^{2} \cdot 5 \cdot 7\\
& 1,680 	     && 2^{4} \cdot 3^{1} \cdot 5 \cdot 7\\
& 2,520        && 2^{3} \cdot 3^{2} \cdot 5 \cdot 7\\
& 5,040        && 2^{4} \cdot 3^{2} \cdot 5 \cdot 7\\
\end{align*}


\section*{Pythagorean Triples}
Famous Pythagorean triples: 
\begin{align*}
(3, 4, 5)~~
(5, 12, 13)~~
(8, 15, 17)~~
(7, 24, 25)\\
(20, 21, 29)~~	
(12, 35, 37)~~	
(9, 40, 41)\\ 	
(28, 45, 53)~~
(11, 60, 61)~~ 
(16, 63, 65)\\
(33, 56, 65)~~ 	
(48, 55, 73)~~
(13, 84, 85)\\
(36, 77, 85)~~
(39, 80, 89)~~
(65, 72, 97) 
\end{align*}

\newpage

\subsection*{Useful Irrational Numbers}
\begin{align*}
     \pi & \approx 3.14159 \ldots \\
       e & \approx 2.71828 \ldots \\
 \varphi & \approx 1.61803 \ldots \\
  \gamma & \approx 0.57722 \ldots \\
\end{align*}

\subsection*{Useful Square Roots}
\begin{align*}
 \sqrt{2} & \approx 1.414214 \\
 \sqrt{3} & \approx 1.732051 \\
 \sqrt{4} & = 2\\
 \sqrt{5} & \approx 2.236068 \\
 \sqrt{6} & \approx 2.449490 \\
 \sqrt{7} & \approx 2.645751 \\
 \sqrt{8} & \approx 2.828427 \\
 \sqrt{9} & = 3\\
\sqrt{10} & \approx 3.162278 \\
\sqrt{11} & \approx 3.316625 \\
\sqrt{12} & \approx 3.464102 \\
\sqrt{13} & \approx 3.605551 \\
\sqrt{14} & \approx 3.741657 \\
\sqrt{15} & \approx 3.872983 \\
\sqrt{16} & = 4\\
\sqrt{17} & \approx 4.123106 \\
\sqrt{18} & \approx 4.242641 \\
\sqrt{19} & \approx 4.358899 \\
\sqrt{20} & \approx 4.472136 \\
\end{align*}

\end{document}