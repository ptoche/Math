
\section*{Prime Numbers}
The first few prime numbers with their index:

\begin{minipage}{\textwidth}
\centering
\resizebox{\columnwidth}{!}{%
\begin{tabular}{X|ZZZZZZZZZZZZZZZZZZZZZ}
i &   1 &   2 &   3 &   4 &   5 &   6 &   7 &   8 &   9 &  10
  &  11 &  12 &  13 &  14 &  15 &  16 &  17 &  18 &  19 &  20 \\
p &   2 &   3 &   5 &   7 &  11 &  13 &  17 &  19 &  23 &  29
  &  31 &  37 &  41 &  43 &  47 &  53 &  59 &  61 &  67 &  71 \bigskip\\
i &  21 &  22 &  23 &  24 &  25 &  26 &  27 &  28 &  29 &  30
  &  31 &  32 &  33 &  34 &  35 &  36 &  37 &  38 &  39 &  40 \\
p &  73 &  79 &  83 &  89 &  97 & 101 & 103 & 107 & 109 & 113
  & 127 & 131 & 137 & 139 & 149 & 151 & 157 & 163 & 167 & 173 \bigskip\\
i &  41 &  42 &  43 &  44 &  45 &  46 &  47 &  48 &  49 &  50
  &  51 &  52 &  53 &  54 &  55 &  56 &  57 &  58 &  59 &  60 \\
p & 179 & 181 & 191 & 193 & 197 & 199 & 211 & 223 & 227 & 229
  & 233 & 239 & 241 & 251 & 257 & 263 & 269 & 271 & 277 & 281 \\
\end{tabular}}
\end{minipage}

Prime numbers of the twentieth and twenty-first centuries (problems involving numbers close to the current year are popular: the closest prime numbers to $2020$ are $2017$ and $2027$):
\begin{align*}
& 1901, 1907, 1913, 1931, 1933, 1949, 1951, 1973, 1979, 1987, 1993, 1997, 1999, \\ 
& 2003, 2011, 2017, 2027, 2029, 2039, 2053, 2063, 2069, 2081, 2083, 2087, 2089, 2099
\end{align*}
Mersenne primes are prime numbers of the form $2^{p}-1$, for some prime number $p$. The first few Mersenne primes are:
\begin{center}
\begin{tabular}{X|XXXXXXXXX}
p       & 2 & 3 & 5  & 7   & 11   & 13   & 17     & 19     & 31 \\
2^{p}-1 & 3 & 7 & 31 & 127 & 8191 & 131071 & 524287 & 2147483647 \\
\end{tabular}
\end{center}

Some Mersenne numbers that are not prime include:
\begin{center}
\begin{tabular}{X|XXX}
p       & 11                  & 23                          & 29 \\
2^{p}-1 & 2047 = 23 \times 89 & 8388607  = 47 \times 178481 & 536870911 = 233 \times 1103 \times 2089 \\
\end{tabular}
\end{center}
     
Fermat primes are prime numbers of the form $2^{2^{n}}+1$. There are only five known Fermat primes:
\begin{center}
\begin{tabular}{X|XXXXXXXXX}
n           & 0 & 1 & 2  & 3   & 4 \\
2^{2^{n}}+1 & 3 & 5 & 17 & 257 & 65537 \\
\end{tabular}
\end{center}
The famous mathematician Euler showed that
\begin{align*}
2^{2^{5}}+1 = 2^{32}+1 = 4294967297 = 641 \times 6700417
\end{align*}

Some Fibonacci numbers are prime. Here are the first few:
\begin{align*}
2, 3, 5, 13, 89, 233, 1597, 28657
\end{align*}
