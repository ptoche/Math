
\subsection*{Simplifying Numbers}
Bombelli, 1572 (« L'algebra »):
\begin{align*}
\sqrt[\leftroot{-1}\uproot{2}\scriptstyle 3]{2 + \sqrt{-121}}
+ \sqrt[\leftroot{-1}\uproot{2}\scriptstyle 3]{2 - \sqrt{-121}}
= 4
\end{align*}

Leibniz, 1675:
\begin{align*}
\sqrt{1 + \sqrt{-3}}
+ \sqrt{1 - \sqrt{-3}}
= \sqrt{6}
\end{align*}

A simple complex example:
\begin{align*}
(1+i)(1-i)
= 2
\end{align*}
where $i$ is the imaginary unit and satisfies $i^{2}=-1$.

Trigonometric angles:
\begin{table}[H]
\begin{tabular}{@{}*{3}{M@{\hspace{5em}}}@{}} 
\cos(\pi)   & -1 & 
\cos(\pi/2) & 0  &
\cos(\pi/3) & \frac{1}{2} \\[1ex]
\cos(\pi/4) & \frac{\sqrt{2}}{2} &
\cos(\pi/6) & \frac{\sqrt{3}}{2} &
\cos(0)     & 1 \\
\end{tabular}
\end{table}
