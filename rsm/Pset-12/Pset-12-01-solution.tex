To find the angle between lines FD and BE in hexagon ABCDEF with the given properties, let's analyze the configuration step by step.

    Understanding the Properties:
    - The hexagon has pairs of equal sides: AB=BC

, CD=DE, and EF=FA.
- Angles A and C are right angles (i.e., ∠A=90∘ and ∠C=90∘
).
Positioning the Hexagon:
- Let's place point A
at the origin (0, 0) in a Cartesian coordinate system.
- Since ∠A is a right angle, we can place B at (a,0) and C at (a,b) where AB=BC=a

    .
    - Thus, we have:

        A(0,0)

B(a,0)
C(a,b)

    Finding Points D, E, and F:
    - Since ∠C

is also a right angle, we can place point D directly below point C at (a−c,b) where CD=DE=c.
- Point E would then be at (a−c,b−c).
- Finally, point F would be directly left of point E at (0,b−c) since EF=FA=d
.
Calculating Slopes:
- The slope of line FD from F(0,b−c)
to D(a−c,b) is:
slope of FD=b−(b−c)(a−c)−0=ca−c

    The slope of line BE from B(a,0)

to E(a−c,b−c) is:
slope of BE=(b−c)−0(a−c)−a=b−c−c=−b−cc

    Finding the Angle Between the Lines:
    - The angle θ

between two lines with slopes m1 and m2 can be found using the formula:
tanθ=∣∣m1−m21+m1m2∣∣
- Here, m1=ca−c and m2=−b−cc
.
Calculating the Tangent:
- Substitute the slopes into the formula:
tanθ=∣∣∣ca−c+b−cc1+ca−c⋅−b−cc∣∣∣
Final Angle Calculation:
- After simplifying the expression for tanθ
, we can find that the angle θ is 45∘

    .

Thus, the angle between lines FD and BE is 45 degrees.