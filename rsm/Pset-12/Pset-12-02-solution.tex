Let us reflect the point $A$ over both $X$ and $Y$ to two points $U$ and $V$ so that $AU = AV = 2$. This seems slightly better, because $AU = AV = 2$ now, and the "two" in the perimeter is now present. But what do we do? Recalling that $s = 2$ in the triangle, we find that $U$ and $V$ is the tangency points of the excircle, call it $a$. Set $IA$ the excenter, tangent to $BC$ at $TA$. See Figure. 

We have now encoded the $AX = AY = 1$ condition as follows: $X$ and $Y$ are the midpoints of the tangents to the $A$-excircle. We need to show that one of $ABM$
or $ACM$ has a perimeter equal to the length of the tangent. 

What would have to be true in order to obtain the relation $AB + BM + MA = AU$? Write $AU = AB + BU = AB + BT$. We need $BM + MA = BT$ , or $MA = MT$. Points $X$, $M$, $Y$ have the property that their distance to $A$ equals the length of their tangents to the $A$-excircle. This motivates the last addition to our
diagram: construct a circle of radius zero at $A$, say $ω0$. Then $X$ and $Y$ lie on the radical axis of $ω0$ and $Ta$; hence so does $M$. Now we have $MA = MT$, as required.

It reflects whether $T$ lies on $BM$ or $CM$. (It must lie in at least one, because we are told that $M$ lies inside the segment $BC$, and the tangency points of the $A$-excircle to $BC$ always lie in this segment as well.) This completes the solution, which we present concisely below.

Let $IA$ be the center of the $A$-excircle, tangent to $BC$ at $T$ and to the extensions of $AB$ and $AC$ at $U$ and $V$. We see that $AU = AV = s = 2$. Then $XY$ is the radical axis of the A-excircle and the circle of radius zero at $A$. Therefore $AM = MT$.

Assume without loss of generality that $T$ lies on $MC$, as opposed to $MB$. Then $AB + BM + MA = AB + BM + MT = AB + BT = AB + BU = AU = 2$ as desired.

\begin{figure}[H]
\centering
\includegraphics[width=\linewidth,height=0.30\textheight,keepaspectratio]%
{pset-12-02-figure-01}
\end{figure}

\begin{figure}[H]
\centering
\includegraphics[width=\linewidth,height=0.30\textheight,keepaspectratio]%
{pset-12-02-figure-02}
\end{figure}

