Since $n$ is small, we can check small integers. Recall that if $p$ is prime, $\varphi(p)=p-1$; if $p$ is prime, $\varphi(p^{a})=p^{a}-p^{a-1}$, for $a>0$; and if $m$ and $n$ are relatively prime, $\varphi(m \mult n)=\varphi(m)\mult\varphi(n)$.
\begin{align*}
\varphi(1) & = 1
\\
\varphi(2) & = 2-1 = 1
\\ 
\varphi(3) & = 3-1 = 2
\\ 
\varphi(4) & = \varphi(2^{2}) = 2^{2}-2^{1} = 2
\\ 
\varphi(5) & = 5-1 = 4
\\ 
\varphi(6) & = \varphi(2 \mult 3) = \varphi(2) \mult \varphi(3) = 1 \mult 2 = 2
\\
\varphi(7) & = 7-1 = 6
\\
\varphi(8) & = \varphi(2^{3}) = 2^{3}-2^{2} = 4
\end{align*}

Solution: There are three values of $n$ such that $\varphi(n)=2$: 
\fbox{$3; 4; 6$.}