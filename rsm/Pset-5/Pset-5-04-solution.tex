\begin{itemize}[label={-}]
\item An obvious solution is $(m,n)=(m,1)$. For $n=1$, $\varphi(\varphi(1^{m}))=\varphi(1)=1$. 

\item Another solution is $(m,n)=(3,2)$.
For $n=2$, $\varphi(\varphi(2^{m}))=\varphi(2^{m-1}(2-1))=\varphi(2^{m-1})=2^{m-2}(2-1)=2^{m-2}$. And $\varphi(\varphi(2^{m}))=2$ iff $m=3$. 

\item Let $n=3$, $\varphi(\varphi(3^{m}))=\varphi(3^{m-1}(3-1))=\varphi(2) \mult \varphi(3^{m-1})=2 \mult 3^{m-2}(3-1)=4 \mult 3^{m-2}$. There is no value of $m$ such that $\varphi(\varphi(3^{m}))=3$. 

\item If $n$ is an odd prime, $\varphi(n^{m})=n^{m-1}(n-1)$, where $(n-1)$ is even, implying
\begin{align*}
\varphi(\varphi(n^{m})) = \varphi((n-1)n^{m-1})
\le \varphi(n-1) \varphi(n^{m-1})
= (n-1) n^{m-2} \varphi(n-1)
\end{align*}
\end{itemize}

Solution: $(m,n) \in \{ (m,1), (3,2), (X,X), (X,X) \}$. 