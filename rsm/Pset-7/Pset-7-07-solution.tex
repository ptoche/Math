$F$ denotes the point of tangency with the circle. There are two possible configurations, one where the circle lies inside the square (\ref{pset:7:07:fig:1}), one where it lies outside (\ref{pset:7:07:fig:2}). Given the value of $AF$, the second configuration applies. 

\begin{figure}[H]
\centering
\includegraphics[width=0.3\linewidth]%
{pset-7-07-figure-02}
\caption{\label{pset:7:07:fig:1}}
\end{figure}

\begin{figure}[H]
\centering
\includegraphics[width=0.3\linewidth]%
{pset-7-07-figure-01}
\caption{\label{pset:7:07:fig:2}}
\end{figure}


Let $B=(0,0)$, $C=(s,0)$, $A=(0,s)$, $D=(s,s)$, and $E=\left(s+\frac{r}{\sqrt{2}},s+\frac{r}{\sqrt{2}} \right)$. By the Pythagorean Theorem, 
\begin{align*}
r^2 + \biggl(9 + 5\sqrt{2}\biggr)
& = \biggl(s+\frac{r}{\sqrt{2}}\biggr)^2 + \biggl(\frac{r}{\sqrt{2}}\biggr)^2
\\[1ex]
s^2+rs\sqrt{2} 
& = 9+5\sqrt{2}
\end{align*}
Since $r$ and $s$ are rational, we must have $s^2=9$ and $rs=5$.

Solution: \fbox{$r/s=5/9$.}
