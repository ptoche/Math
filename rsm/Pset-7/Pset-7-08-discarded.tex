Cubic equations have at least one real root. The three roots are real if the turning points are on opposite sides of the $x$-axis. Let $f(x)=x^{3}-ax^{2}+bx-a$. The roots of $f(x)$ are real if the roots of $f'(x)=0$ satisfy $f(r_{1})f(r_{2})<0$. Solve for $f'(x)=0$:
\begin{align*}
f'(x) = 3 x^{2} - 2ax + b = 0
\implies
f(r_{1}) \mult f(r_{2})
& = f\left(\frac{a + \sqrt{a^{2}-3b}}{3}\right) \mult f\left(\frac{a - \sqrt{a^{2}-3b}}{3}\right)
\\
& = \frac{a^{2}-(\sqrt{a^{2}-3b})^{2}}{9}
  = \frac{3b}{9}
  = \frac{b}{3}
\end{align*}
The roots of $f'(x)=0$ have opposite sign if, and only if, $b<0$.

Solution: \fbox{$b<0$.}