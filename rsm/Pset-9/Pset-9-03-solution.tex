This appears as 2014 AIME II Problems/Problem 5.

First, derive conditions on $r$ and $s$ from the stated restrictions on polynomial $P(x)$.
\begin{align*}
\left\{\begin{aligned}
r^{3} + ar + b & = 0
\\
s^{3} + as + b & = 0
\end{aligned}\right.
\implies
r^{3} - s^{3} + a (r-s) = 0
\implies
(r - s) (r^{2} + rs + s^{2} + a) = 0
\end{align*}
A similar relation can be derived from $Q(x)$:
\begin{align*}
(r - s + 7) \bigl((r+4)^{2} + (r+4)(s-3) + (s-3)^{2} + a\bigr) = 0
\end{align*}
Assuming $r \ne s$ and $r \ne s-7$, subtract the two equalities and simplify:
\begin{align*}
5r - 2s + 13 = 0
\implies 
s = \frac{13+5r}{2}
\end{align*}
Let $r$, $s$, and $t$ be the three roots of $P(x)$. Applying Vieta's formula for the product and sum of the three roots:
\begin{align*}
\begin{aligned}
rst & = -b
\\ 
r+s+t & = 0
\end{aligned}
\implies
rs(r+s) = b
\end{align*}
Applying the same logic to $Q(x)$ and substituting $b=rs(r+s)$ to eliminate $b$:
\begin{align*}
(r+4)(s-3)(r+s+1) = b+240 = rs(r+s) + 240
\end{align*}
Substituting $s$ in terms of $r$ into the above yields:
\begin{align*}
(r+4) \biggl(\frac{13+5r}{2}-3\biggr)\biggl(r+\frac{13+5r}{2}+1\biggr) 
& = r \biggl(\frac{13+5r}{2}\biggr)\biggl(r+\frac{13+5r}{2}\biggr) + 240
\\[1ex]
(r+4) (5r+7) (7r+15)
& = r (5r+13) (7r+13) + 960
\\[1ex]
108 (r+5) (r-1)
% solve (r+4)*(5*r+7)*(7*r+15)-r*(5*r+13)*(7*r+13)-960
% r = -5, r = 1
\end{align*}
The roots are $r=1$ and $r=-5$. Using $s=(13+5r)/2$ gives $s=9$, $s=-6$. The root pairs $(r,s)$ are $(1,9)$ and $(-5,-6)$. It follows that
\begin{align*}
b = rs(r+s) 
\to 90, -330
\to |90|+|-330|=420.
\end{align*}


\fbox{$|b|=420$.}