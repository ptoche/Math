\subsubsection*{Notation:}
\begin{figure}[H]
\centering
\begin{subfigure}{0.48\linewidth}
\centering
\includegraphics[page=1,height=3cm]%
{Figures/problem-3-solution-figure-1}
\caption*{Absolute Node Positions}
\end{subfigure}
\begin{subfigure}{0.48\linewidth}
\centering
\includegraphics[page=2,height=3cm]%
{Figures/problem-3-solution-figure-1}
\caption*{Starting Position}
\end{subfigure}
\end{figure}

Let $a$, $b$, $c$, $d$, $e$ denote the vertices associated with the starting position, so that we can identify the position of a coin as a label/position pair. For instance, move $X$ results in $(A,a) \to (A,d)$, $(D,d) \to (D,c)$, $(C,c) \to (C,a)$. 

\begin{enumerate}

\item Position (a) may be reached by the following sequence of moves: $YXXYYXYXY$.

\begin{figure}[H]
\centering
\begin{subfigure}{0.40\linewidth}
\centering
\includegraphics[page=2,height=3cm]%
{Figures/problem-3-solution-figure-1}
\end{subfigure}
\tikz[baseline=-\baselineskip]\draw[thick,->] (0,1)-- node[midway,above,yshift={10pt}]{$YXXYYXYXY$} (3,1);
\begin{subfigure}{0.40\linewidth}
\centering
\includegraphics[page=3,height=3cm]%
{Figures/problem-3-solution-figure-1}
\end{subfigure}
\end{figure}

The effect of the moves is summarized in the table:
\begin{table}[H]
\centering
\begin{tblr}{
  width = \linewidth,
  colspec = {Q[l,25pt,$]*{19}{Q[c,11pt,$]}},
}
\toprule  
\text{coin} ~\backslash~\text{rot}
&   & Y && X && X && Y && Y && X && Y && X && Y \\
\midrule 
A & a & \to 
  & a & \to 
  & d & \to 
  & c & \to 
  & c & \to 
  & c & \to 
  & a & \to 
  & a & \to 
  & d & \to 
  & b & \\
B & b & \to 
  & e & \to 
  & e & \to 
  & e & \to 
  & d & \to 
  & b & \to 
  & b & \to 
  & e & \to 
  & e & \to 
  & d & \\
C & c & \to 
  & c & \to 
  & a & \to 
  & d & \to 
  & b & \to 
  & e & \to 
  & e & \to 
  & d & \to 
  & c & \to 
  & c & \\
D & d & \to 
  & b & \to 
  & b & \to 
  & b & \to 
  & e & \to 
  & d & \to 
  & c & \to 
  & c & \to 
  & a & \to 
  & a & \\
E & e & \to 
  & d & \to 
  & c & \to 
  & a & \to 
  & a & \to 
  & a & \to 
  & d & \to 
  & b & \to 
  & b & \to 
  & e & \\
\bottomrule
\end{tblr}
\end{table}

\item Position (b) cannot be reached from the starting position. It may be approached by the sequence $XYYXX$, where only $E$ and $A$ are out of position. We then show that no adjacent vertices such as $E$ and $A$ can be exchanged without causing other vertices to be displaced, completing the proof. 

\begin{figure}[H]
\centering
\begin{subfigure}{0.40\linewidth}
\centering
\includegraphics[page=2,height=3cm]%
{Figures/problem-3-solution-figure-1}
\end{subfigure}
\tikz[baseline=-\baselineskip]\draw[thick,->] (0,1)-- node[midway,above,yshift={10pt}]{$XYYXX$} (3,1);
\begin{subfigure}{0.40\linewidth}
\centering
\includegraphics[page=4,height=3cm]%
{Figures/problem-3-solution-figure-1}
\end{subfigure}
\end{figure}

The effect of the moves is summarized in the table:
\begin{table}[H]
\centering
\begin{tblr}{
  width = \linewidth,
  colspec = {Q[l,25pt,$]*{11}{Q[c,11pt,$]}},
}
\toprule  
\text{coin} ~\backslash~\text{rot}
&   & X && Y && Y && X && X \\
\midrule 
A & e & \to 
  & d & \to 
  & c & \to 
  & c & \to 
  & c & \to 
  & b & \\
B & a & \to 
  & a & \to 
  & d & \to 
  & b & \to 
  & e & \to 
  & d & \\
C & c & \to 
  & c & \to 
  & a & \to 
  & a & \to 
  & a & \to 
  & c & \\
D & d & \to 
  & b & \to 
  & b & \to 
  & e & \to 
  & d & \to 
  & a & \\
E & b & \to 
  & e & \to 
  & e & \to 
  & d & \to 
  & b & \to 
  & e & \\
\bottomrule
\end{tblr}
\end{table}


Any attempt to swap the $A$ and $E$ coins forces a swap elsewhere, e.g. a swap of $B$ and $D$.
\begin{figure}[H]
\centering
\begin{subfigure}{0.40\linewidth}
\centering
\includegraphics[page=3,height=3cm]%
{Figures/problem-3-solution-figure-1}
\end{subfigure}
\tikz[baseline=-\baselineskip]\draw[thick,->] (0,1)-- node[midway,above,yshift={10pt}]{$YXXYYXYXY$} (3,1);
\begin{subfigure}{0.40\linewidth}
\centering
\includegraphics[page=4,height=3cm]%
{Figures/problem-3-solution-figure-1}
\end{subfigure}
\end{figure}


The effect of the moves is summarized in the table:
\begin{table}[H]
\centering
\begin{tblr}{
  width = \linewidth,
  colspec = {Q[l,25pt,$]*{15}{Q[c,11pt,$]}},
}
\toprule  
\text{coin} ~\backslash~\text{rot}
&   & Y && X && Y && Y && X && X && Y \\
\midrule 
A & e & \to 
  & d & \to 
  & c & \to 
  & c & \to 
  & c & \to 
  & a & \to 
  & d & \to 
  & b & \\
B & a & \to 
  & a & \to 
  & d & \to 
  & b & \to 
  & e & \to 
  & e & \to 
  & e & \to 
  & d & \\
C & c & \to 
  & c & \to 
  & a & \to 
  & a & \to 
  & a & \to 
  & d & \to 
  & c & \to 
  & c & \\
D & d & \to 
  & b & \to 
  & b & \to 
  & e & \to 
  & d & \to 
  & c & \to 
  & a & \to 
  & a & \\
E & b & \to 
  & e & \to 
  & e & \to 
  & d & \to 
  & b & \to 
  & b & \to 
  & b & \to 
  & e & \\
\bottomrule
\end{tblr}
\end{table}

\end{enumerate}
