Every positive integer $N$ can be represented in base $2$ by 
$(N)_{2} = b_{n}b_{n-1}\ldots b_{1}b_{0}$, where 
$$N = b_{0} + b_{1} \cdot 2 + \ldots + b_{n-1}  \cdot 2^{n-1} + b_{n}  \cdot 2^{n}$$
such that for each $i=0,\ldots,n$, $b_{i}=0$ or $b_{i}=1$, and $b_{n}\ne0$. In this problem, we consider writing numbers in an alternative form of base $2$ where $(N)^{\text{alt}}_{b}=b_{n}b_{n-1}\ldots b_{1}b_{0}$ if
$$N = b_{0} + b_{1} \cdot (2^{0} + 1) + \ldots + b_{n-1} \cdot (2^{n-2}+1) + b_{n} \cdot (2^{n-1}+1)$$
such that for each $i=0,\ldots,n$, $b_{i}=0$ or $b_{i}=1$ and $b_{n}\ne0$. Noting
\begin{itemize}
\item $2^{0} + 1 = 2$
\item $2^{1} + 1 = 3$
\item $2^{2} + 1 = 5$
\item $2^{3} + 1 = 9$
\item $2^{4} + 1 = 17$
\end{itemize}
we get
$$ (12)^{\text{alt}}_{2} = 10100 \text{ since } 12 = 9 + 3 = 1 \cdot 9 + 0 \cdot 5 + 1 \cdot 3 + 0 \cdot 2 + 0 \cdot 1,$$ 
and
$$(29)^{\text{alt}}_{2} = 110100 \text{ since } 29 = 17 + 9 + 3 = 1 \cdot 17 + 1 \cdot 9 + 0 \cdot 5 + 1 \cdot 3 + 0 \cdot 2 + 0 \cdot 1.$$

\begin{enumerate}
\item Show that for any positive integer $N$ there are $b_{0},\ldots,b_{n}$ such that $(N)^{\text{alt}}_{2} = b_{n}b_{n-1}\ldots b_{1}b_{0}$.
\item Using this definition, $3$ has two representations in alternative base $2$:
$$ (3)^{\text{alt}}_{2} = 11 \text{ and } (3)^{\text{alt}}_{2} = 100$$
However, some numbers, such as $4$ and $7$, have unique representations in alternative base $2$. For example $(4)^{\text{alt}}_{2}=101$ and $(7)^{\text{alt}}_{2}=1010$, and neither of these numbers have a second representation in alternative base $2$. Show that there are infinitely many positive integers that have a unique representation in alternative base $2$, and there are infinitely many numbers that have at least two representations in alternative base $2$.
\end{enumerate}