\documentclass[12pt]{article}
\usepackage{../../preamble2}

\newcommand{\modulo}[1]{~(\mathrm{mod}~#1)}
%\newcommand{\modulo}[1]{~\%#1}

\newcolumntype{C}{>{$}c<{$}}% \usepackage{array}

\setenumerate{label=(\roman*),itemjoin={\quad},before={\mbox{}\bigskip\\},after={\bigskip\\}}

\usepackage[explicit]{titlesec}
\renewcommand{\thesubsubsection}{\arabic{subsubsection}}% Numbering without sub-numbers
\titleformat{\subsubsection}{\normalfont\Large\bfseries}{}{0em}{#1\ \thesubsubsection}

\title{UCLA Math Circle}
\author{James Toche (and family)}
\date{16 August 2020 \\(Last revision: \today)}

\begin{document}
\begin{minipage}{\textwidth}
\maketitle
\begin{abstract}
More problems on modular arithmetic to accompany the UCLA Math Circle Intermediate-2 for Summer Session 2020, August 16th. 
\end{abstract}
\end{minipage}


\section*{Diophantine Equations}

\section*{Problem~1}
\begin{question}
Find integer solutions to the Diophantine equation $2x+3y = 0$. What about $2x+3y = 1$? And $2x+3y = 31$? Think about how you can get the third equation from the second equation.
\end{question}


\section*{Problem~2}
\begin{question}
Suppose we have a solution $(x_{0},y_{0})$ to the Diophantine equation $ax+by=1$. Let $n$ be an arbitrary integer. Show there is a solution to the Diophantine equation $ax+by=n$. Find a solution.
\end{question}


\end{document}
