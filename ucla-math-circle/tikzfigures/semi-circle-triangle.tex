\documentclass[border=3mm]{standalone}
\usepackage{tikz}
\usetikzlibrary{calc}
\usetikzlibrary{intersections}% 'name path='

\newenvironment{standalonepage}{}{}%keep pictures on same page
\standaloneenv{standalonepage}

\newcommand{\scale}{0.6}

\newcommand{\base}{16cm}
\newcommand{\mbase}{8cm}
\newcommand{\height}{15cm}
\newcommand{\side}{17cm}

% arc centered at point (rather than starting at point)
% \arcus(center)(radius)(start angle:end angle)
\newcommand\arcus{}
\def\arcus(#1)(#2)(#3:#4){%
  ($(#1)+({#2*cos(#3)},{#2*sin(#3)})$)arc(#3:#4:#2)}

% calculate distance: recommended version
\usetikzlibrary{fpu}% precise calculations
\newcommand{\pgfmathparseFPU}[1]{\begingroup%
\pgfkeys{/pgf/fpu,/pgf/fpu/output format=fixed}%
\pgfmathparse{#1}%
\pgfmathsmuggle\pgfmathresult\endgroup}
\makeatletter
\pgfmathdeclarefunction{distance}{2}{%
\begingroup%
\pgfextractx{\pgf@xa}{\pgfpointanchor{#1}{center}}%
\pgfextracty{\pgf@ya}{\pgfpointanchor{#1}{center}}%
\pgfextractx{\pgf@xb}{\pgfpointanchor{#2}{center}}%
\pgfextracty{\pgf@yb}{\pgfpointanchor{#2}{center}}%
\pgfmathparseFPU{sqrt((\pgf@xa-\pgf@xb)*(\pgf@xa-\pgf@xb)+(\pgf@ya-\pgf@yb)*(\pgf@ya-\pgf@yb))}%
\pgfmathsmuggle\pgfmathresult\endgroup%
}%
\makeatother

% START basic figure
% no empty line between \tikzset and myfige/.pic={
\tikzset{% 
  myfig/.pic={%  
    \begin{scope}[scale=\scale]% must match scale in tikzpicture below

      % coordinates of triangle
      \coordinate (B) at (0,0);
      \coordinate (A) at (0,\height);
      \coordinate (Z) at (-\mbase,0);
      \coordinate (C) at (+\mbase,0);
      
      % triangle
      \draw (A) -- (Z) -- (C) -- cycle;
      
      % coordinates (D) of perpendicular line from B to AC
      \coordinate (D) at ($(A)!(B)!(C)$);
      
      % semi-circle, alternative version
      %\draw [black] let \p1=(B), \p2=(D), \n1={veclen(\x2-\x1,\y2-\y1)} in (B) \arcus(B)(\n1)(0:180);

      % semi-circle
      \pgfmathsetmacro{\radius}{distance("B","D")}
      \draw [black] \arcus(B)(\radius pt)(0:180);
      
    \end{scope}
  }%
}%
% END basic figure


\begin{document}


\begin{standalonepage}
  \begin{tikzpicture}[scale=\scale]% must match scope in myfig
  
    \pic {myfig};
  
  \end{tikzpicture}
\end{standalonepage}


\begin{standalonepage}
  \begin{tikzpicture}[scale=\scale]% must match scope in myfig
  
    \pic (original) {myfig};
    
    % mark the tangency right angles
     \begin{scope}[rotate=29]
       \draw [blue,thick] (D) +(-5mm,0) |- +(0,5mm);
     \end{scope}

    % mark the tangency right angles
     \begin{scope}[rotate=119]
       \draw [red,thick] (D) +(-5mm,0) |- +(0,5mm);
     \end{scope}
    
    % draw part of the triangle ABD
    \pgfmathsetmacro{\radius}{distance("B","D")}% must define it again
    \draw [blue,thick] (B)--(A)--(D);
    
    % draw part of the triangle BDC
    \draw [red,thick] (B)--(C)--(D);
    
    % draw segment BD in two colors
    \draw [red,thick] (B)--(D);
    \draw [blue,thick] (B)++(0,1.8pt)--+($(D)-(B)$);
        
    % label the radius
    \path (B)--(D) node[black, above, midway] {$r$};

    % label points
    \node [black, above] at (A) {$A$};
    \node [black, below right] at (C) {$C$};
    \node [black, above right] at (D) {$D$};
    \node [black, below] at (B) {$B$};
    
    % known lengths
    \draw [<->, green!50!black,thick] (10,0) -- (10,15) node[green!50!black, midway, right] {$15$};
    \draw [<->, green!50!black,thick] (0,-1.5) -- (8,-1.5) node[green!50!black, midway, below] {$8$};
    \draw [<->, blue,thick, transform canvas={xshift=0.8cm,yshift=0.3cm}] (A) -- (D) node[blue,midway, above right] {$17-x$};
    \draw [<->, red,thick, transform canvas={xshift=0.8cm,yshift=0.3cm}] (D) -- (C) node[red,midway, above right] {$x$};        
  \end{tikzpicture}
\end{standalonepage}

\end{document}
