% !TEX root = solutions.tex

A natural starting point for this problem is to draw a hexagon inscribed in a circle (Figure~\ref{pb-2-fig-1a}) and to translate it so that one of its vertices coincides with the center of the circle (Figure~\ref{pb-2-fig-1b}). It is then clear that the hexagon must be shrunk to the point where the short-diagonal of the hexagon coincides with the radius $r$ of the circle (Figure~\ref{pb-2-fig-1c}).

\begin{figure}[H]
\centering
\begin{subfigure}[t]{0.32\linewidth}
\centering 
\includegraphics[width=\linewidth]%
{figures-2/problem-2-question-figure-1}%
\caption{\label{pb-2-fig-1a}}
\end{subfigure}
\hfill%
\begin{subfigure}[t]{0.32\linewidth}
\includegraphics[width=\linewidth]%
{figures-2/problem-2-question-figure-2}%
\caption{\label{pb-2-fig-1b}}
\end{subfigure}
\hfill%
\begin{subfigure}[t]{0.32\linewidth}
\includegraphics[width=\linewidth]%
{figures-2/problem-2-question-figure-3}%
\caption{\label{pb-2-fig-1c}}
\end{subfigure}
\caption{\label{pb-2-fig-1}There exist two polygons with one vertex at the center of the circle and two vertices on the circumference of the circle: see panel~(b) and~(c). Only one of them is such that only one vertex is outside the circle: see panel~(c).}
\end{figure}


In hexagon $ABCDEF$ (Figure~\ref{pb-2-fig-2}), consider first the triangle $ACE$ inscribed inside the hexagon. Claim: \textit{Triangle $ACE$ is equilateral}. Proof: Since $A$ is the center of the circle and $C$ and $E$ are on the circumference, $AC=r$ and $AE=r$. By the half-angle formula, the chord length $CE$ is equal to $2r\sin(\theta/2)$, where  $\theta={\angle}CAE=\pi/3$. Calculating $\sin(\theta/2)=\sin(\pi/6)=1/2$ and substituting back gives a chord length $CE=2r\sin(\theta/2)=r$. It follows that all three sides of the triangle have length $r$ (it is actually obvious by considerations of symmetry). 

\begin{figure}[H]
\centering
\includegraphics[width=0.5\linewidth,keepaspectratio]%
{figures-2/problem-2-question-figure-4}%
\caption{\label{pb-2-fig-2}The area of the polygon may be calculated as the sum of areas of the equilateral triangle $ACE$ and of the isosceles triangles $ABC$, $CDE$, and $EFA$.}
\end{figure}

Since triangle $ACE$ is equilateral, its area is $\frac{\sqrt{3}}{4}r^{2}$ (the side lengths of the equilateral triangle coincide with the radius of the circle). This well-known formula can be proved by applying the Pythagorean theorem: See Figure~\ref{pb-2-fig-3}. 

\begin{figure}[H]
\centering
\includegraphics[width=0.6\linewidth,keepaspectratio]%
{figures-2/problem-2-equilateral-triangle}%
\caption{\label{pb-2-fig-3}The area of equilateral triangle $ACE$ is $\frac{\sqrt{3}}{4}r^{2}$.}
\end{figure}

\begin{figure}[H]
\centering
\includegraphics[width=\linewidth,keepaspectratio]%
{figures-2/problem-2-isosceles-triangle}%
\caption{\label{pb-2-fig-4}The area of isosceles triangle $ABC$ is $\frac{\sqrt{3}}{12}r^{2}$.}
\end{figure}

Consider now the isosceles triangle $ABC$ in hexagon $ABCDEF$ (Figure~\ref{pb-2-fig-2}). The base of triangle $ABC$ coincides with the chord $AC=r$. The angle at the vertex is $\varphi={\angle}ABC=\frac{2\pi}{3}$. The angle $\varphi$ follows from the sum-of-angles formula. The sum of the angles of a regular polygon with $n$ sides is $(n-2)\pi$. For a hexagon, $n=6$, the angle is therefore $\varphi=\frac{(6-2)\pi}{6}=\frac{2\pi}{3}$. The isosceles triangle $ABC$ fits exactly three times into the equilateral triangle $ACE$. And therefore its area is $\frac{\sqrt{3}}{12}r^{2}$. As a sanity check, this area may also be calculated from trigonometry:
\begin{align*}
\frac{r^{2}}{4\tan(\varphi/2)}
= \frac{r^{2}}{4\tan(\pi/3)}
= \frac{r^{2}}{4\sqrt{3}}
= \frac{\sqrt{3}}{12}r^{2}.
\end{align*}

Adding together the area of triangle $ACE$ and triangles $ABC$, $AEF$, and $CDE$ gives the area of hexagon $ABCDEF$:
\begin{align*}
\frac{\sqrt{3}}{4}r^{2}
+ 3 \ \frac{\sqrt{3}}{12} r^{2}
= \frac{\sqrt{3}}{2}
\end{align*}
where we have substituted $r=1$, the radius of the unit circle.

Conclusion: The area of the hexagon is \fbox{$\frac{\sqrt{3}}{2}$}.

\subsubsection*{Problem 2: Sanity Check}
A quick check is shown in Figure~\ref{pb-2-fig-5}. 
\begin{figure}[H]
\centering
\begin{subfigure}[t]{0.49\linewidth}
\centering 
\includegraphics[width=\linewidth]%
{figures-2/hexagon-circle-grid-outer}%
\caption{\label{pb-2-fig-5a}Outer grid.}
\end{subfigure}
\hfill%
\begin{subfigure}[t]{0.49\linewidth}
\includegraphics[width=\linewidth]%
{figures-2/hexagon-circle-grid-inner}%
\caption{\label{pb-2-fig-5b}Inner grid.}
\end{subfigure}
\caption{\label{pb-2-fig-5}Place a regular grid on top of the figure such that the radius of the circle is divided into $10$ squares. The outer grid in Figure~\ref{pb-2-fig-5a} covers $100$ squares. The inner grid in  Figure~\ref{pb-2-fig-5b} covers $72$ squares. The polygon's area is strictly inside $(0.72, 1)$ and approximately $0.86$. This is consistent with $\frac{\sqrt{3}}{2}\approx0.866$.}
\end{figure}

%$\sin(\pi/3)=\frac{\sqrt{3}}{2}$.

% % !TEX root = ../usamts-round-1-solutions.tex

\subsection*{Useful formulas for regular polygons}

Let $n$ denote the number of sides; $s$ the side length; $r$ the radius; $a$ the apothem; and $A$ the area. 

\begin{align*}
A 
& = n s^{2} \times \frac{1}{4\tan(\pi/n)}
\\
& = n r^{2} \times \frac{\sin(2\pi/n)}{2}
\\
& = n a^{2} \times \tan(\pi/n)
\end{align*}
