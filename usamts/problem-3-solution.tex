% !TEX root = solutions.tex

The question does not state who splits the integers into two groups: If the split is engineered to maximize $M$, the answer is $M=1012$. If the split is engineered to minimize $M$, the answer is $M=2$. By another interpretation, the answer is $M=16$.

\subsubsection*{General Considerations}

The Fibonacci numbers $F_{0},F_{1},F_{2},\ldots$ are defined inductively, for $n \ge 1$, by 
\begin{align*}
F_{0} & = 0, \\
F_{1} & = 1, \\ 
F_{n+1} & = F_{n} + F_{n-1}.
\end{align*}
On the closed interval $[0,2024]$, the Fibonacci subsequence is 
\begin{align*}
0, 1, 1, 2, 3, 5, 8, 13, 21, 34, 55, 89, 144, 233, 377, 610, 987, 1597.
\end{align*}
where $F_{17}=1597$.
Clearly, the largest possible difference $|a_{k+1}-a_{k}|$, calculated from consecutive terms in the sequence $a_{1},a_{2},\ldots,a_{m}$, cannot exceed $1597$ and likewise for $b_{1},b_{2},\ldots,b_{n}$. 


\subsubsection*{Amicable Split}

Consider the arithmetic sequence with first term $1$ and common difference $F_{1}$. Split the integers as
\begin{align*}
& \text{Group A}: 
\quad
a_{1},a_{2},a_{3}\ldots,a_{1012}
&& = 1,2,3\ldots,1012
\\
& \text{Group B}: 
\quad
b_{1},b_{2},b_{3}\ldots,b_{1012}
&& = 1013,1014,1015\ldots,2024
\end{align*}
where $m=n=1012$.
The difference between any two consecutive elements in the sequence $a_{1},a_{2},a_{3}\ldots,a_{1012}$ is the Fibonacci number $F_{1}$. Likewise for $b_{1},b_{2},b_{3}\ldots,b_{1012}$. Every consecutive element from either Group A or Group B is a fibtastic sequence. The stated conditions are therefore satisfied and the maximum (max-max) is $M=1012$.

\subsubsection*{Adversarial Split}
If $m$ and $n$ are chosen to minimize $M$, we must have $m\ge2$ and $n\ge2$ to ensure that the difference between any two consecutive numbers is defined. An adversarial split that minimizes $M$ is:
\begin{align*}
& \text{Group A}: 
\quad
a_{1}, a_{2}
&& = 1, 2
\\
& \text{Group B}: 
\quad
b_{1},b_{2},b_{3}\ldots,b_{2022}
&& = 3,4,5\ldots,2024
\end{align*}
These sequences are fibtastic, as shown above. The stated conditions are therefore satisfied and the maximum (max-min) is $M=2$. 

\subsubsection*{Increasing Fibonacci Subsequences}
Adding a constraint to the problem makes it more interesting. Consider the revised definition:
A sequence of integers $x_{1},x_{2},...,x_{k}$ is called \textit{fibtastic} if the difference between any two consecutive elements in the sequence is a Fibonacci number \textbf{\textit{and if these Fibonacci numbers form a strictly increasing sequence}}.

The largest increasing sequence formed by the consecutive differences would be a subsequence of the first seventeen Fibonacci numbers: $F_{1},F_{2},\ldots,F_{17}$. However, $F_{17}=1597$ is unreachable for $m=n=1012$, so the increasing Fibonacci subsequence is reduced to $F_{1},F_{2},\ldots,F_{16}$, where $F_{16}=987$. With these additional restrictions, a candidate solution is $M=16$. It remains to be shown that a subsequence of that length does exist. 

We start by constructing a subsequence on $[1,2024]$. 

\begin{table}[H]
\centering
\begin{tblr}{
  colspec = {X[c,2cm] @{\ } X[c,15pt] X[c] @{\ } X[c,20pt] @{\ } X[c]},
  cell{2-Z}{1-Z}={mode=math},
  row{1}={font=\bfseries},
  column{3}={halign=l},
  column{5}={halign=r},
  rowsep = 0.2ex,
  stretch=0,
}
\specialrule{.8pt}{0pt}{5pt}% .4pt = width of rule
Number 
& &
\parcell{Fibonacci\\Representation}
& &
\parcell{Consecutive\\Difference}\\
\specialrule{.4pt}{5pt}{5pt}% .4pt = width of rule
 1 & = & F_{1}                 && \\
   &   &                       & \to & F_{3} - F_{1} = F_{2} \\
 2 & = & F_{3}                 && \\
   &   &                       & \to & F_{4} - F_{3} = F_{2} \\
 3 & = & F_{4}                 && \\
   &   &                       & \to & F_{2} \\
 4 & = & F_{4} + F_{2}         && \\
   &   &                       & \to & F_{5} - F_{4} - F_{2} = F_{1} \\
 5 & = & F_{5}                 && \\
   &   &                       & \to & F_{2} \\
 6 & = & F_{5} + F_{2}         && \\
   &   &                       & \to & F_{3} - F_{2} = F_{1} \\
 7 & = & F_{5} + F_{3}         && \\
   &   &                       & \to & F_{6} - F_{5} - F_{3} = F_{2} \\
 8 & = & F_{6}                 && \\
   &   &                       & \to & F_{2} \\
 9 & = & F_{6} + F_{2}         && \\
   &   &                       & \to & F_{3} - F_{2} = F_{1} \\
10 & = & F_{6} + F_{3}         && \\
   &   &                       & \to & F_{4} - F_{3} = F_{2} \\
11 & = & F_{6} + F_{4}         && \\
   &   &                       & \to & F_{2} \\
12 & = & F_{6} + F_{4} + F_{2} && \\
   &   &                       & \to & F_{7} - F_{6} - F_{4} - F_{2} = F_{1} \\
13 & = & F_{7}                 && \\
   &   &                       & \to & F_{1} \\
14 & = & F_{7} + F_{2}         && \\
   &   &                       & \to & F_{3} - F_{2} = F_{1} \\
15 & = & F_{7} + F_{3}         && \\
   &   &                       & \to & F_{4} - F_{3} = F_{2} \\
16 & = & F_{7} + F_{4}         && \\
   &   &                       & \to & F_{2} \\
17 & = & F_{7} + F_{4} + F_{2} && \\
   &   &                       & \to & F_{5} - F_{4} - F_{2} = F_{1} \\
18 & = & F_{7} + F_{5}         && \\
   &   &                       & \to & F_{2} \\
19 & = & F_{7} + F_{5} + F_{2} && \\
   &   &                       & \to & F_{3} - F_{2} = F_{1} \\
20 & = & F_{7} + F_{5} + F_{3} && \\
   &   &                       & \to & F_{8} - F_{7} - F_{5} - F_{3} = F_{2} \\
21 & = & F_{8}                 && \\
\specialrule{.8pt}{5pt}{0pt}% .4pt = width of rule
\end{tblr}
\caption{\label{pb-3-tbl-1}}
\end{table}
