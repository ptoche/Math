% !TEX root = solutions.tex

\subsubsection*{Introduction}

The question does not explicitly state a restriction on the sequence of Fibonacci numbers. We consider strictly increasing sequences, monotonically increasing sequences (increasing, but not strictly), and non-monotonic sequences. If the Fibonacci numbers must form a strictly increasing sequence, the answer is $M=14$. If they must form a monotonically increasing sequence, the answer is $M=15$. If non-monotonic sequences are permitted, $M$ can be larger, as shown below. 

The question does not state who splits the integers into two groups: If the split is engineered to maximize $M$, the answer is $M=1012$. If the split is engineered to minimize $M$, the answer is $M=2$. 

\subsubsection*{General Considerations}

The Fibonacci numbers $F_{0},F_{1},F_{2},\ldots$ are defined inductively, for $n \ge 1$, by 
\begin{align*}
F_{0} & = 0, \\
F_{1} & = 1, \\ 
F_{n+1} & = F_{n} + F_{n-1}.
\end{align*}
The first few terms of the Fibonacci sequence are:
\begin{align*}
0, 1, 1, 2, 3, 5, 8, 13, 21, 34, 55, 89, 144, 233, 377, 610, 987, 1597.
\end{align*}
where $F_{17}=1597$.

The largest possible difference $|a_{k+1}-a_{k}|$, calculated from consecutive terms in any subsequence $a_{1},a_{2},\ldots,a_{m}$, is at most $F_{17}=1597$, since the next Fibonacci number, $F_{18}=2584$, is too large for $m<2024$, and likewise for $b_{1},b_{2},\ldots,b_{n}$. 
%But $F_{17}=1597$ is unreachable if the subsequences are split into $1012$ elements each. Thus, $F_{16}=987$ is a candidate for the largest number that could appear in the difference sequence. 

The question does not explicitly state a constraint on the Fibonacci numbers. We consider several interpretations. 

\subsubsection*{Amicable Split}

Consider the arithmetic sequence with first term $1$ and common difference $F_{1}=1$. Split the integers as
\begin{align*}
& \text{Group A}: 
\quad
a_{1},a_{2},a_{3}\ldots,a_{1012}
&& = 1,2,3\ldots,1012
\\
& \text{Group B}: 
\quad
b_{1},b_{2},b_{3}\ldots,b_{1012}
&& = 1013,1014,1015\ldots,2024
\end{align*}
where $m=n=1012$.
The difference between any two consecutive elements in the sequence $a_{1},a_{2},a_{3}\ldots,a_{1012}$ is the Fibonacci number $F_{1}=1$. Likewise for $b_{1},b_{2},b_{3}\ldots,b_{1012}$. Every consecutive element from either Group A or Group B is a fibtastic sequence. The stated conditions are therefore satisfied and the maximum (max-max) is $M=1012$.

\subsubsection*{Adversarial Split}
If $m$ and $n$ are chosen to minimize $M$, we must have $m\ge2$ and $n\ge2$ to ensure that the difference between any two consecutive numbers is defined. An adversarial split that minimizes $M$ is:
\begin{align*}
& \text{Group A}: 
\quad
a_{1}, a_{2}
&& = 1, 2
\\
& \text{Group B}: 
\quad
b_{1},b_{2},b_{3}\ldots,b_{2022}
&& = 3,4,5\ldots,2024
\end{align*}
These sequences are fibtastic, as shown above. The stated conditions are therefore satisfied and the maximum (max-min) is $M=2$. 

\subsubsection*{Increasing Fibonacci Numbers}
Consider the revised definition:
A sequence of integers $x_{1},x_{2},...,x_{k}$ is called \textit{increasing fibtastic} if the difference between any two consecutive elements in the sequence is a Fibonacci number \textbf{\textit{and if these Fibonacci numbers form an increasing sequence}}. 

Method of construction:
\begin{itemize}
\item Construct a fibtastic sequence of length $n$ starting from $3$ (strict) or $2$ (not strict), that will be used for $(a)_{k}$.
\item Construct another fibtastic sequence of length $n$ starting after the $n$th term of the fibtastic sequence created for $(a)_{k}$.
\item For each fibtastic sequence, place all the integers that fall in the "gaps" into the other subsequence. 
\item Any integer smaller or greater than the terms of the fibtastic sequences can be placed into any one of $(a)_{k}$ or $(b)_{k}$.
\end{itemize}

This method of construction guarantees that no integer "interferes" with the fibtastic sequences. 

\subsubsection*{Solution for "strictly increasing" fibtastic sequence:}
$M=14$: The terms in red form a strictly increasing fibtastic sequence. The terms in blue could be moved from subsequence $(b)_{k}$ to subsequence $(a)_{k}$ without altering the solution. 
\begin{align*}
(a)_{k} 
& = \textcolor{red}{3, 4, 6, 9, 14, 22, 35, 56, 90, 145, 234, 378, 611, 988},
\\ 
& \quad\
993, 995, 996, \ldots, 1973, 1974, 1975.
\\[1ex] 
(b)_{k} 
& = \textcolor{blue}{1, 2,} \
5, 7, 8, 10, 11, \ldots, 985, 986, 987, \
\textcolor{blue}{989, 990,} 
\\
& \quad\
\textcolor{red}{991, 992, 994, 997, 1002, 1010, 1023, 1044, 1078, 1133,}
\\
& \quad\
\textcolor{red}{1222, 1366, 1599, 1976}, 
\\
& \quad\
\textcolor{blue}{1977, 1978, 1979, 1980, 1981, \ldots, 2022, 2023, 2024}.
\end{align*}
The difference between consecutive terms in the fibtastic sequence is:
\begin{align*}
1, 2, 3, 5, 8, 13, 21, 34, 55, 89, 144, 233, 377
\end{align*}
The difference sequence is the same for $(a)_{k}$ and $(b)_{k}$. It is strictly increasing. 

The difference sequence uses the Fibonacci numbers from $F_{2}=1$ to $F_{14}=377$. The first two Fibonacci numbers, $F_{0}=0$ and $F_{1}=1$ cannot be used, because they would create a difference of $0$ between consecutive terms.


\subsubsection*{Solution for "monotonically increasing" fibtastic sequence:}
$M=15$: The terms in red form a monotonically increasing fibtastic sequence. The terms in blue could be moved from subsequence $(b)_{k}$ to subsequence $(a)_{k}$ without altering the solution. 
\begin{align*}
(a)_{k} 
& = \textcolor{red}{3, 4, 5, 7, 10, 15, 23, 36, 57, 91, 146, 235, 379, 612, 989,}
\\ 
& \quad\
994, 996, 997, \ldots, 1974, 1975, 1976.
\\[1ex] 
(b)_{k} 
& = \textcolor{blue}{1, 2,} \
6, 8, 9, 11, \ldots, 986, 987, 988, \ 
\textcolor{blue}{990,} 
\\
& \quad\
\textcolor{red}{991, 992, 993, 995, 998, 1003, 1011, 1024, 1045, 1079, 1134,}
\\
& \quad\
\textcolor{red}{1223, 1367, 1600, 1977}, 
\\
& \quad\
\textcolor{blue}{1978, 1979, 1980, \ldots, 2022, 2023, 2024}.
\end{align*}
The difference between consecutive terms in the fibtastic sequence is:
\begin{align*}
1, 1, 2, 3, 5, 8, 13, 21, 34, 55, 89, 144, 233, 377
\end{align*}
The difference sequence is the same for $(a)_{k}$ and $(b)_{k}$. It is monotonically increasing, but not strictly increasing since the value $1$ is repeated. 

The difference sequence uses the Fibonacci numbers from $F_{1}=1$ to $F_{14}=377$. The next Fibonacci number after $F_{14}=377$ is $F_{15}=610$. Since $1976+610>2024$, it is clearly not possible to insert $F_{15}$. 

\subsubsection*{Conclusion}
\begin{itemize}[label=-,nosep]
\item adversarial split: $M=2$ or no solution.
\item amicable split / no constraint on the difference sequence: $M=1012$.
\item monotonically increasing difference sequence: $M=15$.
\item strictly increasing difference sequence: $M=14$.
\end{itemize}
