% !TEX root = ../usamts-round-1-solutions.tex

\subsubsection*{Solution:}
%
\begin{figure}[H]
\centering
\includegraphics[width=\linewidth,height=0.25\textheight,keepaspectratio]%
{grid-df-v1}%
%\caption{\label{pb-1-fig-1}}
\end{figure}

\subsubsection*{Introduction}
The solution can be found by trial and error. Writing a good algorithm for solving the problem is challenging. Ideas that I found interesting and motivated me to write code to investigate the problem:
\begin{itemize}
\item the effect of shuffling the numbers in the starting grid.
\item the conditions under which the solution is unique.
\item find situations where the completed grid has no holes.  
\item change the condition "for each cell containing a number $N$ in the grid, exactly two other cells containing $N$ are at a Manhattan distance of $N$" to "at most two other cells" or to "at least two other cells" or to some other quantity. 
\item measure the level of difficulty of a starting grid.
\end{itemize}


\subsubsection*{Naive Algorithm}
I coded a basic algorithm and developed a few tools to assist in solving. A naive algorithm is:
\begin{enumerate}
\item For each number $N$ in the grid, identify all the cells that are at a distance $N$. These cells are potential "connections". 
\item Among these cells, if there are fewer than $2$ with the number $N$, then create a list of the empty cells: the "candidates".
\item Iterate through the lits of candidates: insert the value if the insertion is consistent with the rules. Keep a record of the sequence of insertions. 
\item At the end of the iteration, check if the condition "for each cell containing a number $N$ in the grid, exactly two other cells containing $N$ are at a Manhattan distance of $N$" is satisfied. If the condition is not satisfied, go back to the start and attempt another sequence of insertions. 
\end{enumerate}
This algorithm is very expensive. A single iteration yields the outcome shown in Figure~\ref{pb-1-fig-2}. To get the solution requires multiple iterations. 

\begin{figure}[H]
\centering
\includegraphics[width=\linewidth,height=0.25\textheight,keepaspectratio]%
{grid-dv2-mistakes-v2}%
\caption{\label{pb-1-fig-2}A single iteration of the naive algorithm yields this outcome. Subsequent iterations have to backtrack to the initial state. Incorrect insertions and omissions are highlighted in color.}
\end{figure}

\subsubsection*{Improved Algorithm}
\begin{itemize}
\item Before inserting a value in one of the candidate cells, order the candidates by the number of distinct values it could be filled with ("potential insertions"). To reduce the risk of a future clash, start inserting into candidate cells that have the smallest number of potential insertions (least attractive "targets").
\item For each cell in the grid, calculate the number of potential connections to it. Update the "heatmap" before every step in the iteration. Start inserting values into cells that have the smallest "potential degree", as they are least likely to result in an incorrect insertion.
\end{itemize}


Figure~\ref{pb-1-fig-3} shows the heatmap at the first and last iteration. 
\begin{figure}[H]
\centering
\begin{subfigure}[t]{0.45\linewidth}
\centering 
\includegraphics[width=\linewidth]%
{grid-d0-heatmap}%
\caption{\label{pb-1-fig-3a}Initial grid.}
\end{subfigure}
\hfill%
\begin{subfigure}[t]{0.45\linewidth}
\includegraphics[width=\linewidth]%
{grid-df-heatmap}%
\caption{\label{pb-1-fig-3b}Final grid.}
\end{subfigure}
\caption{\label{pb-1-fig-3}A heatmap of the potential number of connections to each cell in the grid. This information can be used to improve the speed of the algorithm.}
\end{figure}

