% !TEX root = usamts-round-1-solutions.tex

\subsubsection*{Introduction}

Let $g(x)=f(f(x)+x)$. 
Substituting $f(x)=x^{2}+bx+1$ into $f(f(x)+x)$ yields a polynomial of degree $4$. Except for special values of $b$, this polynomial has two turning points and could satisfy $g(x)<0$ on two disconnected intervals. Figure~\ref{pb-5-fig-1a} shows the graph of $g(x)$ for $b=5$. We analyze $f(x)$ and used its special properties to analyze the nested expression $g(x)$. 

\subsubsection*{Sign of $f(x)$}

\begin{minipage}[t]{0.49\linewidth}
\begin{figure}[H]
\centering
\includegraphics[width=\linewidth,keepaspectratio]%
{problem-5-fig-1a}%
\caption{\label{pb-5-fig-1a}$g(x)=f(f(x)+x)$.}
\end{figure}
\end{minipage}
\hfill%
\begin{minipage}[t]{0.49\linewidth}
\begin{figure}[H]
\centering
\includegraphics[width=\linewidth,keepaspectratio]%
{problem-5-fig-2a}%
\caption{\label{pb-5-fig-2a}$f(x)=x^{2}+bx+1$.}
\end{figure}
\bigskip
\end{minipage}

The graph of $f(x)$ is a U-shaped parabola whose vertex is located at point $(0,-b)$. Figure~\ref{pb-5-fig-2a} shows the graph of $f(x)$ for $b=5$. The discriminant associated with the equation $f(x)=0$ is: 
\begin{align*}
\delta = (b^{2}-4)=(b-2)(b+2). 
\end{align*}
It follows that:
\begin{align*}
& b \in (-2,2) 
&&\implies \delta < 0 
   &&\implies f(x) > 0 \quad \forall x \in \mathbb{R}
\\
& b = -2
&&\implies \delta = 0 
   &&\implies f(x) = x^{2} - 2x + 1 = (x-1)^{2} \ge 0 \quad \forall x \in \mathbb{R}
\\
& b = +2
&&\implies \delta = 0 
   &&\implies f(x) = x^{2} + 2x + 1 = (x+1)^{2} \ge 0 \quad \forall x \in \mathbb{R}
\\
& b \in (-\infty, -2) \cup (2, +\infty)
&&\implies \delta > 0 
   &&\implies f(x) > 0 \quad \forall x \in (-\infty, \tfrac{-b-\sqrt{\delta}}{2}) \cup (\tfrac{-b+\sqrt{\delta}}{2}, +\infty)
\\&&&
   &&\implies f(x) = 0 \quad \text{for}~ x \in \{\tfrac{-b-\sqrt{\delta}}{2}, \tfrac{-b+\sqrt{\delta}}{2}\}
\\&&&
   &&\implies f(x) < 0 \quad \forall x \in (\tfrac{-b-\sqrt{\delta}}{2}, \tfrac{-b+\sqrt{\delta}}{2})
\end{align*}
%The problem is to count the number of integer values $n\in\mathbb{N}$ such that:
%\begin{align*}
%(f(n) + n) \in  
%  \Bigl(\tfrac{-b-\sqrt{\delta}}{2},\
%        \tfrac{-b+\sqrt{\delta}}{2}
%   \Bigr)
%\end{align*}

\subsubsection*{Sign of $g(x)=f(f(x)+x)$}
If $\delta>0$, $f$ and $g$ can be factorized as:
\begin{align*}
f(x) 
& = \Bigl(x+\tfrac{b+\sqrt{\delta}}{2}\Bigr)\Bigl(x+\tfrac{b-\sqrt{\delta}}{2}\Bigr)
\\[1ex]
g(x) 
& = f(f(x)+x))
\\[1ex]
& = \Bigl(f(x)+x+\tfrac{b+\sqrt{\delta}}{2}\Bigr) 
    \Bigl(f(x)+x+\tfrac{b-\sqrt{\delta}}{2}\Bigr)
\\[1ex]
%& = \Bigl(x^{2}+(b+1)x+1 - \tfrac{- b - \sqrt{(b-2)(b+2)}}{2}\Bigr) 
%    \Bigl(x^{2}+(b+1)x+1 - \tfrac{- b + \sqrt{(b-2)(b+2)}}{2}\Bigr)
%\\[1ex]
& = \Bigl(x^{2}+(b+1)x + \tfrac{(b+2) + \sqrt{\delta}}{2}\Bigr) 
    \Bigl(x^{2}+(b+1)x + \tfrac{(b+2) - \sqrt{\delta}}{2}\Bigr)
\\
& = \ell(x) \ h(x), \quad \text{where}
\\[1ex]
\ell(x) 
  & = x^{2}+(b+1)x+\tfrac{(b+2)+\sqrt{\delta}}{2}
\\
h(x) 
  & = x^{2}+(b+1)x+\tfrac{(b+2)-\sqrt{\delta}}{2}
\end{align*}
The sign of $g(x)$ depends on the sign of the product of the two quadratic factors: $g(x)<0$ if and only if $\ell(x)h(x)<0$. 

Figure~\ref{pb-5-fig-3} shows the graphs of $g(x)$, $h(x)$, and $\ell(x)$ for $b=3$. The roots of $h(x)$ and $\ell(x)$ determine the range of values for which $f(x)<0$: $h(x)$ and $\ell(x)$ must have opposite signs for $f(x)<0$. For $b=3$, the only two integer solutions of $f(x)<0$ are $x=-3$ and $x=-1$. 

\begin{figure}[H]
\centering
\includegraphics[width=0.8\linewidth,keepaspectratio]%
{problem-5-fig-3}%
\caption{\label{pb-5-fig-3}$g(x)=\ell(x)h(x)$: $g(x)<0$ iff $h(x)$ and $\ell(x)$ have opposite signs.}
\end{figure}

We first analyze $h(x)$. The discriminant associated with $h(x)$ is: 
\begin{align*}
\alpha 
  = (b+1)^{2} -4 \ \tfrac{(b+2) - \sqrt{\delta}}{2}
  = b^{2}-3+2\sqrt{\delta}
\end{align*}
where $b>2$ implies $\alpha>0$. The quadratic $h(x)$ may be factorized as follows:
\begin{align*}
h(x) 
& = \Bigl(x-\tfrac{-(b+1)-\sqrt{b^{2}-3+2\sqrt{\delta}}}{2}\Bigr)
    \Bigl(x-\tfrac{-(b+1)+\sqrt{b^{2}-3+2\sqrt{\delta}}}{2}\Bigr)
\end{align*}

The nested square-roots can be simplified (proof at the end of this section):
\begin{align*}
\sqrt{b^{2}-3+2\sqrt{(b-2)(b+2)}}
= 1 + \sqrt{(b-2)(b+2)}
\end{align*}

Going back to the factorization of $h(x)$, we have
\begin{align*}
h(x) 
& = \Bigl(x-\tfrac{-(b+1)-(1+\sqrt{(b-2)(b+2)}\,)}{2}\Bigr)
    \Bigl(x-\tfrac{-(b+1)+(1+\sqrt{(b-2)(b+2)}\,)}{2}\Bigr)
\\
& = \Bigl(x-\tfrac{-(b+2)-\sqrt{(b-2)(b+2)}\,)}{2}\Bigr)
    \Bigl(x-\tfrac{-b+\sqrt{(b-2)(b+2)}\,)}{2}\Bigr)
\end{align*}
Both roots of $h(x)=0$ are negative with $-(b+2)-\sqrt{(b-2)(b+2)}\,)<-b+\sqrt{(b-2)(b+2)}\,)$ for all $b>2$.
The problem reduces to counting the number of integer values $n\in\mathbb{N}$ such that:
\begin{align*}
n \in \Bigl(\tfrac{-(b+2)-\sqrt{(b-2)(b+2)}\,)}{2},\
            \tfrac{-b+\sqrt{(b-2)(b+2)}\,)}{2}
      \Bigr)
\end{align*}
for $b>2$. This is an interval of width
\begin{align*}
\tfrac{-b+\sqrt{(b-2)(b+2)}\,)}{2}-\tfrac{-(b+2)-\sqrt{(b-2)(b+2)}\,)}{2}=1.
\end{align*}
This interval contains typically exactly $1$ integer value such that $g(x)<0$. However, since the bounds of the interval are excluded, whenever $f(f(x)+x)=0$ for some integer value of $x$, the interval contains exactly $0$ integers. 

The analysis of $\ell(x)$ is similar. 
The problem reduces to counting the number of integer values $n\in\mathbb{N}$ such that:
\begin{align*}
n \in \Bigl(\tfrac{-b-\sqrt{(b-2)(b+2)}\,)}{2},\
            \tfrac{-(b+2)+\sqrt{(b-2)(b+2)}\,)}{2}
      \Bigr)
\end{align*}
for $b>2$. This interval also has width of exactly $1$.

In conclusion, the generic situation is $0$ solutions or $2$ solutions, with $1$ solution occurring in cases where exactly one of $\ell(x)$ or $h(x)$ is zero at an integer value of $x$.


Figure~\ref{pb-5-fig-1e} shows the graphs of $g(x)$ for $b=2$, a case with no integer solution: in this case $\delta=(b-2)(b+2)=0$. 
Figure~\ref{pb-5-fig-1g} shows the graphs of $g(x)$ for $b=5/2$, a case with exactly one integer solution, $x=-1$: in this case, $\delta>0$, but the smallest roots of $h(x)$ and $\ell(x)$ are integers: $h(-3)=\ell(-2)=0$. 
\begin{minipage}[t]{0.49\linewidth}
\begin{figure}[H]
\centering
\includegraphics[width=\linewidth,keepaspectratio]%
{problem-5-fig-1e}%
\caption{\label{pb-5-fig-1e}$0$ integer solution.}
\end{figure}
\end{minipage}
\hfill%
\begin{minipage}[t]{0.49\linewidth}
\begin{figure}[H]
\centering
\includegraphics[width=\linewidth,keepaspectratio]%
{problem-5-fig-1g}%
\caption{\label{pb-5-fig-1g}$1$ integer solution.}
\end{figure}
\bigskip
\end{minipage}

Conclusion: The solutions are \fbox{$0,1,2$}.


\subsubsection*{Proof: Simplifying the Nested Square-Roots}
Suppose that there exist $x>0$ and $y>0$ such that:
\begin{align*}
\sqrt{b^{2}-3+2\sqrt{(b-2)(b+2)}}
= \sqrt{x} + \sqrt{y}
\end{align*}
Square both sides of the equality:
\begin{align*}
b^{2}-3+2\sqrt{(b-2)(b+2)}
= x + y + 2\sqrt{xy}
\end{align*}
Split the equality and equate each part:
\begin{align*}
x + y & = b^{2}-3
\\
x y & = (b-2)(b+2)
\end{align*}
Reduce the system to a single quadratic in $x$:
\begin{align*}
x^{2} - (b^{2}-3) x + (b-2)(b+2) = 0
\end{align*}
The discriminant of this quadratic is ${(b^{2}-3)^{2}-4(b-2)(b+2)=(b^{2}-5)^{2}}$. The roots are $1$ and $(b-2)(b+2)$. Substituting back for $y$ gives
\begin{align*}
\sqrt{b^{2}-3+2\sqrt{(b-2)(b+2)}}
= 1 + \sqrt{(b-2)(b+2)}
\end{align*}
