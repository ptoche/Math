% !TEX root = ../usamts-round-2-solutions.tex

\textbf{Solution:}
The expected value is $\sum_{T \subseteq S_{n}} \mathcal{I} (T) = 0$.

\textbf{Proof:}
\begin{align*}
\sum_{T \subseteq S_{n}} \mathcal{I} (T)
& 
= \ \sum_{\mathclap{\substack{T \subseteq S_{n-1}}}} \mathcal{I} (T)
\ +\ \sum_{\mathclap{\substack{T \subseteq S_{n} \\[0.5ex] x_{n} \in T}}} \mathcal{I} (T)
\\[1ex]
&
= \ \sum_{\mathclap{\substack{T \subseteq S_{n-1}}}} \mathcal{I} (T)
\ +\ \sum_{\mathclap{\substack{T \subseteq S_{n} \\[0.5ex] x_{n} \in T}}} (-1)^{n+1}(n)T[n]-\mathcal{I} (T\setminus\{x_{n}\}))
\\[1ex]
&
= \ \sum_{\mathclap{\substack{T \subseteq S_{n-1}}}} \mathcal{I} (T)
\ +\ \sum_{\mathclap{\substack{T \subseteq S_{n} \\[0.5ex] x_{n} \in T}}} (-1)^{n+1}(n)T[n]
\ -\ \sum_{\mathclap{\substack{T \subseteq S_{n-1}}}} \mathcal{I} (T))
\\[1ex]
&
= \ \sum_{k=0}^{n} (-1)^{k+1}\binom{n}{k}
\\[1ex]
&
= \ 0
\end{align*}
To prove the last step, set $a=1,b=-1$ in the Binomial expansion formula
\begin{align*}
(a+b)^{n} = \sum_{k=0}^{n} \binom{n}{k} a^{n-k} b^{k}
\end{align*}
