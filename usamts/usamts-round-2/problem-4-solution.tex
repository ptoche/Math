% !TEX root = ../usamts-round-2-solutions.tex

\textbf{Solution:}
The expected value is $\sum_{T \subseteq S_{n}} \mathcal{I} (T) = (-1)^{n+1}\,n\,2^{n-1}\,x_{n}$.

\begin{proof}
Let $S_{n}$ denote the given set $S$ where the subscript denotes the number of elements in the set (the cardinal). Let $S_{n-1}$ denote the set $S\setminus\{x_{n}\}$.
\begin{align}
\sum_{T \subseteq S_{n}} \mathcal{I} (T)
& = \ \sum_{\mathclap{\substack{T \subseteq S_{n-1}}}} \mathcal{I} (T)
\ +\ \sum_{\mathclap{\substack{T \subseteq S_{n} \\[0.5ex] x_{n} \in T}}} \mathcal{I} (T)
\\[1ex]
& = \ \sum_{\mathclap{\substack{T \subseteq S_{n-1}}}} \mathcal{I} (T)
\ +\ \sum_{\mathclap{\substack{T \subseteq S_{n} \\[0.5ex] x_{n} \in T}}} \left[(-1)^{n+1}(n)T[n]-\mathcal{I} (T\setminus\{x_{n}\})\right]
\\[1ex]
& = \ \sum_{\mathclap{\substack{T \subseteq S_{n-1}}}} \mathcal{I} (T)
\ +\ \sum_{\mathclap{\substack{T \subseteq S_{n-1}}}} (-1)^{n+1}(n)T[n]
\ -\ \sum_{\mathclap{\substack{T \subseteq S_{n-1}}}} \mathcal{I} (T)
\\[1ex]
& = \ \sum_{\mathclap{\substack{T \subseteq S_{n-1}}}} (-1)^{n+1} \, n \, x_{n}
\\[1ex]
& = \ (-1)^{n+1} \, n \, x_{n} \cdot 2^{n-1}
\end{align}
\end{proof}

\subsubsection*{Explanation}
The proof is adapted from \cite{aops:1983}. 
\begin{itemize}
\item Step~(1): Split the expected value into two terms: one term is the expected value when the $n$th element is excluded from every subset; the other term is the expected value when the $n$th element is included in every subset.
\item Step~(2): Substitute the formula for the index and separate the $n$th term from the rest of the alternating sum.
\item Step~(3): The index of the middle sum is lowered to $n-1$, because the inner term amounts to $1-1$. 
\item Step~(4): The terms up to $n-1$ cancel out. Substitute $T[n]=x_{n}$.
\item Step~(5): There are $2^{n-1}$ subsets of $S_{n-1}$.
\end{itemize}
