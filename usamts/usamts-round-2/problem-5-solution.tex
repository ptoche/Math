% !TEX root = ../usamts-round-2-solutions.tex

The proof is adapted from \cite{j23:2013}. The proof is by contradiction. We suppose a polynomial $P(x)$ exists and derive a contradiction.

Let $\alpha$ denote the irrational number $\alpha=\sqrt[3]{5}+\sqrt[3]{25}$, $\alpha\in\mathbb{R}/\mathbb{Q}$. Let $\beta$ denote the irrational number $\beta=2\sqrt[3]{5}+3\sqrt[3]{25}$, $\beta\in\Ix$. 
The problem is to prove that there is no polynomial $P(x)$ with integer coefficients such that $P(\alpha)=\beta$.

\begin{proof}
Suppose a polynomial $P(x)$ exists with integer coefficients and such that $P(\alpha)=\beta$. There exist polynomials $Q(x)$, $R(x)$ and $w(x)$ with integer coefficients such that 
\begin{align*}
P(x) = Q(x) R(x) + w(x)
\end{align*}
where $Q(x)$ satisfies $Q(\alpha)=0$ (Lemma~\ref{lemma:3}) and where $w(x)$ is a polynomial of degree either $1$ or $2$ such that $w(\alpha)=\beta$ (Lemma~\ref{lemma:2}), implying
\begin{align*}
P(\alpha) 
= Q(\alpha) R(\alpha) + w(\alpha) 
= w(\alpha)
= \beta
\end{align*}
Thus, if $P(x)$ exists, $P(\alpha)=\beta$ implies $w(\alpha)=\beta$, a contradiction with lemma~\ref{lemma:1} and lemma~\ref{lemma:2}, since the coefficients of $w(x)$ are in $\Q$ but not in $\N$. Conclusion: $P(x)$ does not exist.
\end{proof}

\setcounter{lemma}{0}
\begin{lemma}\label{lemma:1}
There exists no polynomial $w(x)$ of degree $1$, with integer coefficients, such that $w(\sqrt[3]{5}+\sqrt[3]{25})=2\sqrt[3]{5}+3\sqrt[3]{25}$.
\end{lemma}

\begin{lemma}\label{lemma:2}
There exists exactly one polynomial $w(x)$ of degree $2$, with rational coefficients, such that $w(\sqrt[3]{5}+\sqrt[3]{25})=2\sqrt[3]{5}+3\sqrt[3]{25}$.
\end{lemma}

\begin{lemma}\label{lemma:3}
There exists one polynomial $Q(x)$ of degree $3$ with integer coefficients such that $Q(\alpha)=0$.
\end{lemma}

\begin{proof}[Proof of Lemma \ref{lemma:1}]
Proof by contradiction. Suppose a polynomial $w(x)=ax+b$ exists, $a,b\in\mathbb{N}$. Since $\alpha\in\Ix$, it must be that $a\ne0$ and $a\ne1$. Moreover $w(\alpha)=\beta$ implies
\begin{align*}
&
a (\sqrt[3]{5}+\sqrt[3]{25}) + b
= 2\sqrt[3]{5}+3\sqrt[3]{25}
\\
& \implies
(a-2)\sqrt[3]{5} + (a-3)\sqrt[3]{25} 
\in \N \subset \Q
\\
& \implies
\bigl((a-2)\sqrt[3]{5} + (a-3)\sqrt[3]{25}\bigr)^{2}
\in \Q
\\
& \implies
(a-2)^{2}\sqrt[3]{5^{2}} + (a-3)^{2}\sqrt[3]{25^{2}} + 2(a-2)(a-3)\sqrt[3]{5^{3}} 
\in \Q
\\
& \implies
(a-2)^{2}\sqrt[3]{25} + 5(a-3)^{2}\sqrt[3]{5} 
\in \Q
\end{align*}
A weighted average of the second and last expressions, with rational weights, is also in $\Q$. Next, select weights so that the $\sqrt[3]{5}$ term is eliminated:
\begin{align*}
&
5(a-3)^{2} \times 
\bigl((a-2)\sqrt[3]{5} + (a-3)\sqrt[3]{25}\bigr)
-
(a-2) \times
\bigl((a-2)^{2}\sqrt[3]{25} + 5(a-3)^{2}\sqrt[3]{5}\bigr)
\in \Q
\\
& \implies
\bigl(5(a-3)^{2}(a-3) - (a-2)^{3}\bigr)\sqrt[3]{25}
\in \Q
\\
& \implies
\sqrt[3]{25}
\in \Q
\end{align*}
where we have used the result that products, sums, and powers of expressions like $(a-2)$ and $(a-3)$ are in $\Q$ if $a\in\Q$, under the assumption that $w$ exists. Since $\sqrt[3]{25}\notin\Q$, we have reached a contradiction.
\end{proof}

\begin{proof}[Proof of Lemma \ref{lemma:2}]
Construct the polynomial $w(x)=ax^{2}+bx+c$ by equating the coefficients derived by developing $w(\alpha)=\beta$:
\begin{align*}
a (\sqrt[3]{5}+\sqrt[3]{25})^{2} + b (\sqrt[3]{5}+\sqrt[3]{25}) + c
& = 2\sqrt[3]{5} + 3\sqrt[3]{25}
\\[1ex]
a \bigl(\sqrt[3]{25} + 10 + 5\sqrt[3]{5}\bigr)
+ b (\sqrt[3]{5}+\sqrt[3]{25}) + c
& = 2\sqrt[3]{5} + 3\sqrt[3]{25}
\\
\implies \quad
{\scriptscriptstyle
\begin{cases}
\quad
a + b
& = \ 3
\\\quad
5a + b
& = \ 2
\\\quad
10a + c
& = \ 0
\end{cases}}
\quad \implies \quad & 
a = -\frac{1}{4}, \
b = \frac{13}{4}, \
c = \frac{5}{2}.
\end{align*}
The desired degree-$2$ polynomial $w(x)$ is: $w(x)=-(1/4)x^{2}+(13/4)x+5/24$. No other coefficients satisfy $w(\alpha)=\beta$.
\end{proof}

\begin{proof}[Proof of Lemma \ref{lemma:3}]
The polynomial $Q(x)$ is constructed from its root $\alpha$:
\begin{align*}
\alpha^{3} 
= \bigl(\sqrt[3]{5}+\sqrt[3]{25}\bigr)^{3}
& = 5 + 3(\sqrt[3]{5})^{2}\sqrt[3]{25} + 3\sqrt[3]{5}(\sqrt[3]{25})^{2} + 25
\\
& = 30 + 15(\sqrt[3]{5}+\sqrt[3]{25})
= 30 + 15\alpha
\end{align*}
Thus, $Q(x)=x^{3}-15x-30$ satisfies $Q(\alpha)=0$.
\end{proof}
