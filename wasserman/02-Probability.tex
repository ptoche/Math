\section{2. Probability}\label{probability}

\subsection{2.2 Sample Spaces and
Events}\label{sample-spaces-and-events}

The \textbf{sample space} \(\Omega\) is the set of possible outcomes of
an experiment. Points \(\omega\) in \(\Omega\) are called \textbf{sample
outcomes} or \textbf{realizations}. \textbf{Events} are subsets of
\(\Omega\).

Given an event \(A\), let
\(A^c = \{ \omega \in \Omega : \text{not } (\omega \in A) \}\) denote
the complement of \(A\). The complement of \(\Omega\) is the empty set
\(\varnothing\). The union of events \(A\) and \(B\) is defined as
\(A \cup B = \{ \omega \in \Omega : \omega \in A \text{ or } \omega \in B \}\).
If \(A_1, A_2, \dots\) is a sequence of sets, then

\[ \cup_{i=1}^\infty A_i = \left\{ \omega \in \Omega : \omega \in A_i \text{ for some } i \right\}\]

The intersection of \(A\) and \(B\) is
\(A \cap B = \{ \omega \in \Omega : \omega \in A \text{ and } \omega \in B \}\).
If \(A_1, A_2, \dots\) is a sequence of sets then

\[ \cap_{i=1}^\infty A_i = \left\{ \omega \in \Omega : \omega \in A_i \text{ for all } i \right\}\]

Let
\(A - B = \left\{ \omega \in \Omega : \omega \in A \text{ and not } (\omega \in B) \right\}\).
If every element of \(A\) is contained in \(B\) we write \(A \subset B\)
or \(B \supset A\). If \(A\) is a finite set, let \(|A|\) denote the
number of elements in \(A\).

\begin{longtable}[]{@{}
  >{\raggedright\arraybackslash}p{(\linewidth - 2\tabcolsep) * \real{0.2778}}
  >{\raggedright\arraybackslash}p{(\linewidth - 2\tabcolsep) * \real{0.7222}}@{}}
\toprule\noalign{}
\begin{minipage}[b]{\linewidth}\raggedright
notation
\end{minipage} & \begin{minipage}[b]{\linewidth}\raggedright
meaning
\end{minipage} \\
\midrule\noalign{}
\endhead
\bottomrule\noalign{}
\endlastfoot
\(\Omega\) & sample space \\
\(\omega\) & outcome \\
\(A\) & event (subset of \(\Omega\)) \\
\(\vert A \vert\) & number of elements in \(A\) (if finite) \\
\(A^c\) & complement of \(A\) (not \(A\)) \\
\(A \cup B\) & union (\(A\) or \(B\)) \\
\(A \cap B\) or \(AB\) & intersection (\(A\) and \(B\)) \\
\(A - B\) & set difference (points in \(A\) but not in \(B\)) \\
\(A \subset B\) & set inclusion (\(A\) is a subset of or equal to
\(B\)) \\
\(\varnothing\) & null event (always false) \\
\(\Omega\) & true event (always true) \\
\end{longtable}

We say that \(A_1, A_2, \dots\) are \textbf{disjoint} or
\textbf{mutually exclusive} if \(A_i \cap A_j = \varnothing\) whenever
\(i \neq j\). A \textbf{partition} of \(\Omega\) is a sequence of
disjoint sets \(A_1, A_2, \dots\) such that
\(\cup_{i=1}^\infty A_i = \Omega\). Given an event \(A\), define the
\textbf{indicator function of \(A\)} by

\[ I_A(\omega) = I(\omega \in A) = \begin{cases}
1 &\text{if } \omega \in A \\
0 &\text{otherwise}
\end{cases}\]

A sequence of sets \(A_1, A_2, \dots\) is \textbf{monotone increasing}
if \(A_1 \subset A_2 \subset \dots\), and we define
\(\lim_{n \rightarrow \infty} A_n = \cup_{i=1}^\infty A_i\). A sequence
of sets \(A_1, A_2, \dots\) is \textbf{monotone decreasing} if
\(A_1 \supset A_2 \supset \dots\) and then we define
\(\lim_{n \rightarrow \infty} A_n = \cap_{i=1}^n A_i\). In either case,
we will write \(A_n \rightarrow A\).

\subsection{2.3 Probability}\label{probability}

A function \(\mathbb{P}\) that assign a real number \(\mathbb{P}(A)\) to
each event \(A\) is a \textbf{probability distribution} or a
\textbf{probability measure} if it satisfies the following three axioms:

\begin{itemize}[tightlist]
\item
  \textbf{Axiom 1}: \(\mathbb{P}(A) \geq 0\) for every \(A\)
\item
  \textbf{Axiom 2}: \(\mathbb{P}(\Omega) = 1\)
\item
  \textbf{Axiom 3}: If \(A_1, A_2, \dots\) are disjoint then
\end{itemize}

\[ \mathbb{P} \left( \cup_{i=1}^\infty A_i \right) = \sum_{i=1}^\infty \mathbb{P}(A_i) \]

A few properties that can be derived from the axioms:

\begin{itemize}[tightlist]
\item
  \(\mathbb{P}(\varnothing) = 0\)
\item
  \(A \subset B \Rightarrow \mathbb{P}(A) \leq \mathbb{P}(B)\)
\item
  \(0 \leq \mathbb{P}(A) \leq 1\)
\item
  \(\mathbb{P}\left(A^c\right) = 1 - \mathbb{P}(A)\)
\item
  \(A \cap B = \varnothing \Rightarrow \mathbb{P}(A \cup B) = \mathbb{P}(A) + \mathbb{P}(B)\)
\end{itemize}

\textbf{Lemma 2.6}. For any events \(A\) and \(B\),
\(\mathbb{P}(A \cup B) = \mathbb{P}(A) + \mathbb(B) - \mathbb{P}(AB)\).

\textbf{Proof}.

\begin{align}
\mathbb{P}(A \cup B) &= \mathbb{P}\left( (AB^c) \cup (AB) \cup (A^cB) \right) \\
&= \mathbb{P}(AB^c) + \mathbb{P}(AB) + \mathbb{P}(A^cB) \\
&= \mathbb{P}(AB^c) + \mathbb{P}(AB) + \mathbb{P}(A^cB) + \mathbb{P}(AB) - \mathbb{P}(AB) \\
&= \mathbb{P}((AB^c) \cup (AB)) + \mathbb{P}((A^cB) \cup (AB)) - \mathbb{P}(AB) \\
&= \mathbb{P}(A) + \mathbb{P}(B) - \mathbb{P}(AB)
\end{align}

\textbf{Theorem 2.8 (Continuity of Probabilities)}. If
\(A_n \rightarrow A\) then \(\mathbb{P}(A_n) \rightarrow \mathbb{P}(A)\)
as \(n \rightarrow \infty\).

\textbf{Proof}. Suppose that \(A_n\) is monotone increasing,
\(A_1 \subset A_2 \subset \dots\). Let \(B_1 = A_1\), and
\(B_{n+1} = A_{n+1} - A_n\) for \(n > 1\). The \(B_i\)'s are disjoint by
construction, and \(A_n = \cup_{i=1}^n A_i = \cup_{i=1}^n B_i\) for all
\(n\). From axiom 3,

\[ \mathbb{P}(A_n) = \mathbb{P}\left( \cup_{i=1}^n B_i \right)  = \sum_{i=1}^n \mathbb{P}(B_i) \]

and so

\[ \lim_{n \rightarrow \infty} \mathbb{P}(A_n) = \lim_{n \rightarrow \infty} \sum_{i=1}^n \mathbb{P}(B_n) = \sum_{i=1}^\infty \mathbb{P}(B_n) = \mathbb{P}\left( \cup_{i=1}^\infty B_i \right) = \mathbb{P}(A) \]

\subsection{2.4 Probability on Finite Sample
Spaces}\label{probability-on-finite-sample-spaces}

If \(\Omega\) is finite and each outcome is equally likely, then

\[ \mathbb{P}(A) = \frac{|A|}{|\Omega|} \]

which is called the \textbf{uniform probability distribution}.

We will need a few facts from counting theory later.

\begin{itemize}
\item
  Given \(n\) objects, the number of way or ordering these objects is
  \(n! = n \cdot (n - 1) \cdot (n - 2) \cdots 3 \cdot 2 \cdot 1\). We
  define \(0! = 1\).
\item
  We define

  \[ \binom{n}{k} = \frac{n!}{k! (n - k)!} \]

  read ``n choose k'', which is the number of different ways of choosing
  \(k\) objects from \(n\).
\item
  Note that choosing a subset \(k\) objects can be mapped to choosing
  the complement set of \(n - k\) objects, so

  \[ \binom{n}{k} = \binom{n}{n - k} \]

  and that there is only one way of choosing the empty set, so

  \[ \binom{n}{0} = \binom{n}{n} = 1\]
\end{itemize}

\subsection{2.5 Independent Events}\label{independent-events}

\begin{itemize}
\item
  Two events \(A\) and \(B\) are \textbf{independent} if

  \[ \mathbb{P}(AB) = \mathbb{P}(A) \mathbb{P}(B) \]

  and we write \(A \text{ ⫫ } B\). A set of events
  \(\{ A_i : i \in I \}\) is independent if

  \[ \mathbb{P} \left( \cap_{i \in J} A_i \right) = \prod_{i \in J} \mathbb{P}(A_i) \]

  for every finite subset \(J\) of \(I\).
\item
  Independence is sometimes assumed and sometimes derived.
\item
  Disjoint events with positive probability are not independent.
\end{itemize}

\subsection{2.6 Conditional
Probability}\label{conditional-probability}

\begin{itemize}
\item
  If \(\mathbb{P}(B) > 0\) then the \textbf{conditional probability} of
  \(A\) given \(B\) is

  \[ \mathbb{P}(A | B) = \frac{\mathbb{P}(AB)}{\mathbb{P}(B)} \]
\item
  \(\mathbb{P}(\cdot | B)\) satisfies the axioms of probability, for
  fixed \(B\). In general, \(\mathbb{P}(A | \cdot)\) does \textbf{not}
  satisfies the axioms of probability for fixed \(A\).
\item
  In general, \(\mathbb{P}(B | A) \neq \mathbb{P}(A | B)\).
\item
  \(A\) and \(B\) are independent if and only if
  \(\mathbb{P}(A | B) = \mathbb{P}(A)\).
\end{itemize}

\subsection{2.7 Bayes' Theorem}\label{bayes-theorem}

\textbf{Theorem 2.15 (The Law of Total Probability)}. Let
\(A_1, \dots, A_k\) be a partition of \(\Omega\). Then, for any event
\(B\),

\[ \mathbb{P}(B) = \sum_{i=1}^k \mathbb{P}(B | A_i) \mathbb{P}(A_i) \]

\textbf{Proof}. Let \(C_j = BA_j\). Note that the \(C_j\)'s are disjoint
and that \(B = \cup_{i=1}^k C_j\). Hence

\[ \mathbb{P}(B) = \sum_j \mathbb{P}(C_j)  = \sum_j \mathbb{P}(BA_j) = \sum_j \mathbb{P}(B | A_j) \mathbb{P}(A_j) \]

\textbf{Theorem 2.16 (Bayes' Theorem)}. Let \(A_1, \dots, A_k\) be a
partition of \(\Omega\) such that \(\mathbb{P}(A_i) > 0\) for each
\(i\). If \(\mathbb{P}(B) > 0\), then, for each \(i = 1, \dots, k\),

\[ \mathbb{P}(A_i | B) = \frac{\mathbb{P}(B | A_i) \mathbb{P}(A_i)}{\sum_j \mathbb{P}(B | A_j) \mathbb{P}(A_j)} \]

We call \(\mathbb{P}(A_i)\) the \textbf{prior probability} of \(A_i\)
and \(\mathbb{P}(A_i | B)\) the \textbf{posterior probability} of
\(A_i\).

\textbf{Proof}. We apply the definition of conditional probability
twice, followed by the law of total probability:

\[ \mathbb{P}(A_i | B) = \frac{\mathbb{P}(A_i B) }{\mathbb{P}(B)} = \frac{\mathbb{P}(B | A_i) \mathbb{P}(A_i)}{\mathbb{P}(B)} = \frac{\mathbb{P}(B | A_i) \mathbb{P}(A_i)}{\sum_j \mathbb{P}(B | A_j) \mathbb{P}(A_j)}\]

\subsection{2.9 Technical Appendix}\label{technical-appendix}

Generally, it is not feasible to assign probabilities to all the subsets
of a sample space \(\Omega\). Instead, one restricts attention to a set
of events called a \textbf{\(\sigma\)-algebra} or a
\textbf{\(\sigma\)-field}, which is a class \(\mathcal{A}\) that
satisfies:

\begin{itemize}[tightlist]
\item
  \(\varnothing \in \mathcal{A}\)
\item
  If \(A_1, A_2, \dots \in \mathcal{A}\), then
  \(\cup_{i=1}^\infty A_i \in \mathcal{A}\)
\item
  \(A \in \mathcal{A}\) implies that \(A^c \in \mathcal{A}\)
\end{itemize}

The sets in \(\mathcal{A}\) are said to be \textbf{measurable}. We call
\((\Omega, \mathcal{A})\) a \textbf{measurable space}. If \(\mathbb{P}\)
is a probability measure defined in \(\mathcal{A}\) then
\((\Omega, \mathcal{A}, \mathbb{P})\) is a \textbf{probability space}.
When \(\Omega\) is the real line, we take \(\mathcal{A}\) to be the
smallest \(\sigma\)-field that contains all off the open sets, which is
called the \textbf{Borel \(\sigma\)-field}.

\subsection{2.10 Exercises}\label{exercises}

\textbf{Exercise 2.10.1}. Fill in the details in the proof of Theorem
2.8. Also, prove the monotone decreasing case.

If \(A_n \rightarrow A\) then
\(\mathbb{P}(A_n) \rightarrow \mathbb{P}(A)\) as
\(n \rightarrow \infty\).

\textbf{Solution}.

Suppose that \(A_n\) is monotone increasing,
\(A_1 \subset A_2 \subset \dots\). Let \(B_1 = A_1\), and
\(B_{i+1} = A_{i+1} - A_i\) for \(i > 1\).

The \(B_i\)'s are disjoint by construction: assuming without loss of
generality \(i < j\), \(\omega \in B_i \cap B_j\) implies that
\(\omega\) is in \(A_j\), \(A_i\), but not in \(A_{j - 1}\),
\(A_{i - 1}\), where \(A_0 = \varnothing\). In particular, this means
that \(\omega \in A_{i}\) but not \(\omega \in A_{j - 1}\). Since
\(A_i \subset A_{j - 1}\), this implies that no such \(\omega\) can
satisfy those properties, and so \(B_i\) and \(B_j\) are disjoint.

Note that \(A_n = \cup_{i=1}^n A_i = \cup_{i=1}^n B_i\) for all \(n\):

\[ \cup_{i=1}^n B_i = \cup_{i=1}^n (A_i - A_{i - 1}) \subset \cup_{i=1}^n A_i = A_n \]

Also note that \(A_n \subset \cup_{i=1}^n B_i\), since, if
\(f(\omega) = \min \{ k : \omega \in A_k \}\), then
\(\omega \in B_{f(\omega)}\), so all elements of \(A_n\) are in some
\(B_k\).

The proof follows as given; from axiom 3,

\[ \mathbb{P}(A_n) = \mathbb{P}\left( \cup_{i=1}^n B_i \right) = \sum_{i=1}^n \mathbb{P}(B_i) \]

and so

\[ \lim_{n \rightarrow \infty} \mathbb{P}(A_n) = \lim_{n \rightarrow \infty} \sum_{i=1}^n \mathbb{P}(B_n) = \sum_{i=1}^\infty \mathbb{P}(B_n) = \mathbb{P}\left( \cup_{i=1}^\infty B_i \right) = \mathbb{P}(A) \]

The monotone decreasing case can be obtained by looking at the
complementary series \(A_1^c, A_2^c, \dots\),which is monotone
increasing. We get

\begin{align}
\lim_{n \rightarrow \infty} \mathbb{P}(A_n^c) &= \mathbb{P}(A^c) \\
\lim_{n \rightarrow \infty} 1 - \mathbb{P}(A_n^c) &= 1 - \mathbb{P}(A^c) \\
\lim_{n \rightarrow \infty} \mathbb{P}(A_n) &= \mathbb{P}(A)
\end{align}

\textbf{Exercise 2.10.2}. Prove the statements in equation (2.1).

\begin{itemize}[tightlist]
\item
  \(\mathbb{P}(\varnothing) = 0\)
\item
  \(A \subset B \Rightarrow \mathbb{P}(A) \leq \mathbb{P}(B)\)
\item
  \(0 \leq \mathbb{P}(A) \leq 1\)
\item
  \(\mathbb{P}\left(A^c\right) = 1 - \mathbb{P}(A)\)
\item
  \(A \cap B = \varnothing \Rightarrow \mathbb{P}(A \cup B) = \mathbb{P}(A) + \mathbb{P}(B)\)
\end{itemize}

\textbf{Solution}.

\begin{itemize}
\item
  By partitioning the event space \(\Omega\) into disjoint partitions
  \((\Omega, \varnothing)\) we get

  \[ \mathbb{P}(\Omega) + \mathbb{P}(\varnothing) = \mathbb{P}(\Omega) \Rightarrow \mathbb{P}(\varnothing) = 0 \]
\item
  Assuming \(A \subset B\) and partitioning \(B\) as \((A, B - A)\), we
  get

  \[ \mathbb{P}(A) + \mathbb{P}(B - A) = \mathbb{P}(B) \Rightarrow \mathbb{P}(A) \leq \mathbb{P}(B) \]
\item
  \(\mathbb{P}(A) \geq 0\) from axiom 1. By partitioning \(\Omega\) as
  \((A, A^c)\), we get

  \[ \mathbb{P}(A) + \mathbb{P}(A^c) = \mathbb{P}(\Omega) = 1 \Rightarrow \mathbb{P}(A) \leq 1 \]
\item
  By partitioning \(\Omega\) as \((A, A^c)\), we get

  \[ \mathbb{P}(A) + \mathbb{P}(A^c) = \mathbb{P}(\Omega) = 1 \Rightarrow \mathbb{P}(A) = 1 - \mathbb{P}(A^c) \]
\item
  Assuming \(A\), \(B\) are disjoint, we partition \(A \cup B\) in
  \((A, B)\) and get:

  \[ \mathbb{P}(A \cup B) = \mathbb{P}(A) + \mathbb{P}(B) \]
\end{itemize}

\textbf{Exercise 2.10.3}. Let \(\Omega\) be a sample space and let
\(A_1, A_2, \dots\) be events. Define \(B_n = \cup_{i=n}^\infty A_i\)
and \(C_n = \cap_{i=n}^\infty A_i\)

\textbf{(a)} Show that \(B_1 \supset B_2 \supset \cdots\) and
\(C_1 \subset C_2 \subset \cdots\).

\textbf{(b)} Show that \(\omega \in \cap_{n = 1}^\infty B_n\) if and
only if \(\omega\) belongs to an infinite number of the events
\(A_1, A_2, \dots\).

\textbf{(c)} Show that \(\omega \in \cup_{n = 1}^\infty C_n\) if and
only if \(\omega\) belongs to all of the events \(A_1, A_2, \dots\)
except possibly a finite number of those events.

\textbf{Solution}.

\textbf{(a)} By construction, \(B_{n+1} = A_n \cup B_n\) and so
\(B_{n + 1} \supset B_n\). Similarly, \(C_{n+1} = A_n \cap C_n\) and so
\(C_{n + 1} \subset C_n\).

\textbf{(b)}

\begin{itemize}
\item
  Assume \(\omega\) belongs to an infinite number of the events,
  \(\omega \in A_j\) for \(j \in J(\omega)\). Then, for every \(n\),
  there is a \(m \geq n\) such that \(m \in J(\omega)\), and so
  \(\omega \in B_n\) for every \(n\). This implies that
  \(\omega \in \cap_{n = 1}^\infty B_n\).
\item
  Assume that \(\omega \in \cap_{n = 1}^\infty B_n\). Then, for every
  \(n\), \(\omega \in B_n\), so for every \(n\) there is a \(m \geq n\)
  such that \(\omega \in A_m\). This implies there is an infinite number
  of such events \(A_m\).
\end{itemize}

\textbf{(c)}

Let's prove the contrapositive.

\begin{itemize}[tightlist]
\item
  Assume that \(\omega\) does not belong to an infinite number of events
  \(A_i\). Then, for every \(n\), there is a \(m \geq\) such that
  \(\omega \in A_m^c\), and so \(\omega\) is not in \(C_n\). Since
  \(\omega\) is not in none of the \(C_n\)'s, it is not in the union of
  all \(C_n\)'s either.
\item
  Assume that \(\omega\) is not in the union of all \(C_n\). This
  implies that \(\omega\) is is not in any event \(C_n\). This implies
  that, for every \(n\), there is a \(m \geq n\) such that \(\omega\) is
  not in \(A_m\). This implies that there is an infinite number of such
  events \(A_m\).
\end{itemize}

\textbf{Exercise 2.10.4}. Let \(\{ A_i : i \in I \}\) be a collection of
events where \(I\) is an arbitrary index set. Show that

\[ \
\left( \cup_{i \in I} A_i \right)^c = \cap_{i \in I} A_i^c 
\quad \text{and} \quad
\left( \cap_{i \in I} A_i \right)^c = \cup_{i \in I} A_i^c 
\]

Hint: First prove this for \(I = \{1, \dots, n\}\).

\textbf{Solution}.

We can prove the result directly by noting that every outcome \(\omega\)
belongs to or does not belong to both sides of each equality:

\begin{align}
& \omega \in \left( \cup_{i \in I} A_i \right)^c \\
\Longleftrightarrow& \; \text{not }\left( \omega \in \cup_{i \in I} A_i  \right) \\
\Longleftrightarrow& \; \forall i \in I, \text{not} \left( \omega \in A_i \right) \\
\Longleftrightarrow& \; \forall i \in I, \omega \in A_i^c \\
\Longleftrightarrow& \; \omega \in \cap_{i \in I} A_i^c
\end{align}

and

\begin{align}
& \omega \in \left( \cap_{i \in I} A_i \right)^c  \\
\Longleftrightarrow&\; \text{not }\left( \omega \in \cap_{i \in I} A_i  \right) \\
\Longleftrightarrow&\; \text{not } \left( \forall i \in I, \omega \in A_i \right) \\
\Longleftrightarrow&\; \exists i \in I, \text{not } \omega \in A_i \\
\Longleftrightarrow&\; \exists i \in I, \omega \in A_i^c \\
\Longleftrightarrow&\; \omega \in \cup_{i \in I} A_i^c 
\end{align}

\textbf{Exercise 2.10.5}. Suppose we toss a fair coin until we get
exactly two heads. Describe the sample space \(S\). What is the
probability that exactly \(k\) tosses are required?

\textbf{Solution}. The sample space is a set of coin toss results
sequences containing two heads, and ending in heads:

\[ S = \left\{ (r_1, \dots, r_k) : r_i \in \left\{ \text{head}, \text{tails} \right\} , 
\Big| \left\{ r_j = \text{head} \right\} \Big|= 2, r_k = \text{head} \right\} \]

The probability of requiring exactly \(k\) tosses is 0 if \(k < 2\), as
there are no such sequences in the event space.

The probability of stopping after \(k\) tosses is the probability of
obtaining exactly 1 head in the first \(k - 1\) tosses, in a procedure
that would not stop after any number of tosses, followed by the
probability of getting a head in the \(k\)-th toss. This value is

\[ \left((k-1) \left(\frac{1}{2}\right)^{k - 1} \right) \left(\frac{1}{2}\right) = \frac{k - 1}{2^k}\]

Note that, besides this combinatorial argument, we can verify that these
probabilities do indeed add up to 1:

\begin{align}
\frac{1}{1 - x} &= \sum_{k = 0}^\infty x^k \\
\frac{d}{dx} \frac{1}{1 - x} &= \sum_{k = 0}^\infty \frac{d}{dx} x^k \\
\frac{1}{(1 - x)^2} &= \sum_{k = 0}^\infty k x^{k - 1} \\
\frac{x}{(1 - x)^2} &= \sum_{k = 0}^\infty k x^k 
\end{align}

so, for \(x = 1/2\), \(\sum_{k = 0}^\infty 2^{-k} k = 2\), and so

\[ \sum_{k = 0}^\infty \frac{k}{2^{k + 1}} = 1 \]

\textbf{Exercise 2.10.6}. Let \(\Omega = \{1, 2, \dots\}\). Prove that
there does not exist a uniform distribution on \(\Omega\), i.e.~if
\(\mathbb{P}(A)  =\mathbb{P}(B)\) whenever \(|A| = |B|\) then
\(\mathbb{P}\) cannot satisfy the axioms of probability.

\textbf{Solution}. Assume that such a distribution exists, and let
\(\mathbb{P}(\{1\}) = p\). Since the distribution is uniform, the
probability associated with any set of size 1 is \(p\), and the
probability associated with any set of size \(n\) is \(np\).

\begin{itemize}
\item
  If \(p > 0\), then a finite set \(A\) of size
  \(|A| = \lceil 2 / p \rceil\) would have probability value
  \(\mathbb{P}(A) = \lceil 2 / p \rceil p \geq (2 / p) p = 2\), which is
  greater than \(1\) -- a contradiction.
\item
  If \(p = 0\), then any finite set \(A\) must have
  \(\mathbb{P}(A) = 0\). But then
  \(\mathbb{P}(\Omega) = \sum_i \mathbb{P}(\{ i \}) = \sum_i 0 = 0\),
  instead of \(1\) -- a contradiction.
\end{itemize}

\textbf{Exercise 2.10.7}. Let \(A_1, A_2, \dots\) be events. Show that

\[ \mathbb{P}\left( \cup_{n=1}^\infty A_n \right) \leq \sum_{n=1}^\infty \mathbb{P}(A_n) \]

Hint: Define \(B_n = A_n - \cup_{i=1}^{n-1} A_i\). Then show that the
\(B_n\) are disjoint and that
\(\cup_{n=1}^\infty A_n  =\cup_{n=1}^\infty B_n\).

\textbf{Solution}. Following the hint, let
\(B_n = A_n - \cup_{i=1}^{n-1} A_i\).

\begin{itemize}[tightlist]
\item
  Note that, for \(i < j\), \(B_i\) and \(B_j\) are disjoint, since all
  elements of \(B_i\) must be elements of \(A_i\), and all elements of
  \(A_i\) are explicitly excluded on the definition of \(B_j\).
\item
  Also note that \(\cup_{n=1}^\infty A_n = \cup_{n=1}^\infty B_n\):
  \(A_n = \cup_{i=1}^n B_i\) by construction, so
  \(\cup_{n=1}^\infty A_n = \cup_{n=1}^\infty \cup_{i=1}^n B_i = \cup_{n=1}^\infty B_n\),
  since \(B_i \cup B_i = B_i\) and we can include each \(B_i\) only once
  in the expression.
\end{itemize}

Now, we have:

\[ \mathbb{P}\left( \cup_{n=1}^\infty A_n \right) = \mathbb{P}\left( \cup_{n=1}^\infty B_n \right) = \sum_{n=1}^\infty \mathbb{P}(B_n) \leq \sum_{n=1}^\infty \mathbb{P}(A_n) \]

since \(B_n \cup \left(\cup_{i=1}^{n-1} A_i\right) = A_n\) and so
\(\mathbb{P}(B_n) \leq \mathbb{P}(A_n)\) for every \(n\).

\textbf{Exercise 2.10.8}. Suppose that \(\mathbb{P}(A_i) = 1\) for each
\(i\). Prove that

\[ \mathbb{P}\left( \cap_{i=1}^\infty A_i \right) = 1 \]

\textbf{Solution}. Using the result from exercise 4,

\[ \mathbb{P}\left( \cap_{i=1}^\infty A_i \right) = 1 - \mathbb{P}\left(\left( \cap_{i=1}^\infty A_i \right)^c\right)
= 1 - \mathbb{P}\left( \cup_{i=1}^\infty A_i^c \right)\]

Using the result from exercise 7,

\[ \mathbb{P}\left( \cup_{i=1}^\infty A_i^c \right) \leq \sum_{i=1}^\infty \mathbb{P}(A_i^c) =  \sum_{i=1}^\infty \left(1 - \mathbb{P}\left( A_i \right) \right) = \sum_{i=1}^\infty 0 = 0 \]

so the equality holds, since a probability is non-negative. Therefore,

\[ \mathbb{P}\left( \cap_{i=1}^\infty A_i \right) = 1 - \mathbb{P}\left( \cup_{i=1}^\infty A_i^c \right) = 1 - 0 = 1 \]

\textbf{Exercise 2.10.9}. For fixed \(B\) such that
\(\mathbb{P}(B) > 0\), show that \(\mathbb{P}(\cdot | B)\) satisfies the
axioms of probability.

\textbf{Solution}.

\begin{itemize}[tightlist]
\item
  Axiom 1:
  \(\mathbb{P}(\cdot | B) = \frac{\mathbb{P}(\cdot B)}{\mathbb{P}(B)} \geq 0\),
  since \(\mathbb{P}(\cdot B) > 0\).
\item
  Axiom 2:
  \(\mathbb{P}(\Omega | B) =  \frac{\mathbb{P}(\Omega B)}{\mathbb{P}(B)} = \frac{\mathbb{P}(B)}{\mathbb{P}(B)} =1\).
\item
  Axiom 3: Assuming \(A_1, A_2, \dots\) are disjoint,
  \[ \mathbb{P} \left( \cup_{i=1}^\infty A_i | B \right) = \frac{\mathbb{P} \left( B \left( \cup_{i=1}^\infty A_i \right) \right)}{\mathbb{P}(B)} 
  =  \frac{\mathbb{P}\left( \cup_{i=1}^\infty \left( A_i B \right) \right)}{\mathbb{P}(B)}
  = \frac{\sum_{i=1}^\infty \mathbb{P}(A_i B)}{\mathbb{P}(B)} = \sum_{i=1}^\infty \frac{\mathbb{P}(A_i B)}{\mathbb{P}(B)}
  = \sum_{i=1}^\infty \mathbb{P}(A_i | B)\]
\end{itemize}

\textbf{Exercise 2.10.10}. You have probably heard it before. Now you
can solve it rigorously. It is called the ``Monty Hall Problem''. A
prize is placed at random between one of three doors. You pick a door.
To be concrete, let's suppose you always pick door 1. Now Monty Hall
chooses one of the other two doors, opens it and shows to you that it is
empty. He then gives you the opportunity to keep your door or switch to
the other unopened door. Should you stay or switch? Intuition suggests
it doesn't matter. The correct answer is that you should switch. Prove
it. It will help to specify the sample space and the relevant events
carefully. Thus write
\(\Omega = \{ (\omega_1, \omega_2) : \omega_i \in \{ 1, 2, 3 \} \}\)
where \(\omega_1\) is where the prize is and \(\omega_2\) is the door
Monty opens.

\textbf{Solution}. Following the provided notation, the event space is

\[ \Omega = \{ (1, 2), (1, 3), (2, 3), (3, 2) \} \]

\(\mathbb{P}[ \omega_2]\) = probability of opening an empty door. The
probability and the reward associated with switching for each outcome
are:

\begin{longtable}[]{@{}lll@{}}
\toprule\noalign{}
\(\omega\) & \(\mathbb{P}\) & \(R\) \\
\midrule\noalign{}
\endhead
\bottomrule\noalign{}
\endlastfoot
\((1, 2)\) & \(\frac{1}{3}\frac{1}{2}\) & \(0\) \\
\((1, 3)\) & \(\frac{1}{3}\frac{1}{2}\) & \(0\) \\
\((2, 3)\) & \(\frac{1}{3} 1\) & \(1\) \\
\((3, 2)\) & \(\frac{1}{3} 1\) & \(1\) \\
\end{longtable}

Therefore,

\[\mathbb{P}[ R | \omega_2 = 2 ] = \frac{\mathbb{P}(\{(3, 2)\})}{\mathbb{P}(\{ (3, 2), (1, 2) \})}= \frac{\frac{1}{3} 1}{\frac{1}{3} 1 + \frac{1}{3}\frac{1}{2}} = \frac{2}{3}\]

and, similarly, \(\mathbb{P}[ R | \omega_3 = 3 ]\), and so
\(\mathbb{P}[R] = \frac{2}{3}\).

\textbf{Exercise 2.10.11}. Suppose that \(A\) and \(B\) are independent
events. Show that \(A^c\) and \(B^c\) are independent events.

\textbf{Solution}.

\begin{align}
\mathbb{P}(A^c B^c) &= \mathbb{P}((A \cup B)^c)  = 1 - \mathbb{P}(A \cup B) \\
&= 1 - \left( \mathbb{P}(A) + \mathbb{P}(B) - \mathbb{P}(AB) \right)  \\
&= 1 - \mathbb{P}(A) - \mathbb{P}(B) + \mathbb{P}(A) \mathbb{P}(B) \\
&= 1 - (1 - \mathbb{P}(A^c)) - (1 - \mathbb{P}(B^c)) + (1 - \mathbb{P}(A^c)) (1 - \mathbb{P}(B^c)) \\
&= \mathbb{P}(A^c) \mathbb{P}(B^c)
\end{align}

\textbf{Exercise 2.10.12}. There are three cards. The first card is
green on both sides, the second is red on both sides, and the third is
green on one side and red on the other. We choose a card at random and
we see one side (also chosen at random). If the side we see is green,
what is the probability that the other side is also green? Many people
intuitively answer 1/2. Show that the correct answer is 2/3.

\textbf{Solution}. There are 6 potential card sides to be chosen, all
with equal probability, of which only 3 are green -- one belongs to the
red / green card, and two belong to the green / green card. The
probability that the other side is also green is the probability that
the a side on the green / green card was chosen, which is 2 / 3.

\textbf{Exercise 2.10.13}. Suppose a fair coin is tossed repeatedly
until both a head and a tail have appeared at least once.

\textbf{(a)} Describe the sample space \(\Omega\).

\textbf{(b)} What is the probability that three tosses will be required?

\textbf{Solution}.

\textbf{(a)}. The sample space consists of the sequence of \(k\)
identical coin toss results and a coin toss result with the opposite
value,

\[ \Omega = \{ (r_1, \dots, r_k, r_{k+1}) : r_i \in \{ \text{head}, \text{tails} \}, r_1 = \dots = r_k \neq r_{k + 1} \} \]

\textbf{(b)} Exactly 3 tosses will be required if the first 3 results
are \((h, h, t)\) or \((t, t, h)\).

If we map all infinite coin toss sequences to \(\Omega\) by truncating
it whenever the stop condition occurs, the probability of a
(single-outcome) event in \(\Omega\) is the same as the probability of
all outcomes mapped into it. In particular, the probability of a
sequence with its first 3 symbols being a specific sequence is 1/8, and
so the probability of the desired outcome is 1/8 + 1/8 = 1/4.

\textbf{Exercise 2.10.14}. Show that if \(\mathbb{P}(A) = 0\) or
\(\mathbb{P}(A) = 1\) then \(A\) is independent of every other event.
Show that if \(A\) is independent of itself then \(\mathbb{P}(A)\) is
either 0 or 1.

\textbf{Solution}.

If \(\mathbb{P}(A) = 0\), then
\(\mathbb{P}(AB) = \mathbb{P}(A) - \mathbb{P}(A - B) = 0 -\mathbb{P}(A - B) \leq 0\),
and since probabilities are non-negative we must have
\(\mathbb{P}(AB) = 0\). Therefore
\(\mathbb{P}(AB) = \mathbb{P}(A) \mathbb{P}(B) = 0\) for all events
\(B\), and \(A\) is independent of every other event.

If \(\mathbb{P}(A) = 1\), then \(\mathbb{P}(A^c) = 0\), and so \(A^c\)
and \(B\) are independent for every other event \(B\). Then, from the
result in exercise 10, \(A\) is also independent from every other event
\(B^c\) -- which covers all potential events, since every event has a
complement.

If \(A\) is independent of itself,
\(\mathbb{P}(AA) = \mathbb{P}(A) \mathbb{P}(A)\), so
\(\mathbb{P}(A) = \mathbb{P}(A)^2\) or
\(\mathbb{P}(A) ( \mathbb{P}(A) - 1 ) = 0\). Therefore
\(\mathbb{P}(A) = 0\) or \(\mathbb{P}(A) = 1\).

\textbf{Exercise 2.10.15}. The probability that a child has blue eyes is
1/4. Assume independence between children. Consider a family with 5
children.

\textbf{(a)} If it is known that at least one child has blue eyes, what
is the probability that at least 3 children have blue eyes?

\textbf{(b)} If it is known that the youngest child has blue eyes, what
is the probability that at least 3 children have blue eyes?

\textbf{Solution}.

\textbf{(a)} Represent the sample space as

\[ \Omega = \{ (x_1, x_2, x_3, x_4, x_5) : x_i \in \{ 0, 1 \} \} \]

where \(x_i = 1\) if the \(i-\)th child (youngest to oldest) has blue
eyes.

\begin{itemize}[tightlist]
\item
  ``At least one child has blue eyes'' is the event
  \(A = \Omega - \{ (0, 0, 0, 0, 0) \}\).\\
\item
  ``At least 3 children have blue eyes'' is the event \(B\) with 3
  children with blue eyes, 4 children with blue eyes, or 5 children with
  blue eyes.
\item
  The intersection of these events is \(BA = B\).
\end{itemize}

Let \(p = 1/4\) be the probability at a given child will have blue eyes.
The desired probability is then:

\[\mathbb{P}(B | A) = \frac{\mathbb{P}(BA)}{\mathbb{P}(A)} = \frac{
\binom{5}{3} p^3 (1 - p)^2 + \binom{5}{4} p^4 (1 - p) + \binom{5}{5} p^5
}{1 - \left(1 - p \right)^5} = \frac{106}{781} \approx 0.1357 \]

\textbf{(b)}

\begin{itemize}[tightlist]
\item
  ``The youngest child has blue eyes'' is the event
  \(C = \{ \omega = (1, x_2, x_3, x_4, x_5) : \omega \in \Omega \}\).
\item
  The intersection of events \(B\) and \(C\) is \(BC\), the set of
  outcomes starting with 1 and having the other 4 dimensions having 2,
  3, or 4 values 1;
  \(BC = \{ \omega = (1, x_2, x_3, x_4, x_5) : \omega \in \Omega, x_2 + x_3 + x_4 + x_5 \geq 2 \}\).
\end{itemize}

The desired probability is then

\[ \mathbb{P}(B | C) = \frac{\mathbb{P}(BC)}{\mathbb{P}(C)} = \frac{p \left(\binom{4}{2} p^2 (1-p)^2 + \binom{4}{3} p^3 (1 - p) + \binom{4}{4} p^4 \right)}{ p } = \frac{67}{256} \approx 0.2617 \]

\textbf{Exercise 2.10.16}. Show that

\[  \mathbb{P}(ABC) = \mathbb{P}(A | BC) \mathbb{P}(B | C) \mathbb{P}(C) \]

\textbf{Solution}.

\[ \mathbb{P}(A | BC) \mathbb{P}(B | C) \mathbb{P}(C) = \frac{\mathbb{P}(ABC)}{\mathbb{P}(BC)}  \frac{\mathbb{P}(BC)}{\mathbb{P}(C)} \mathbb{P}(C) = \mathbb{P}(ABC) \]

\textbf{Exercise 2.10.17}. Suppose \(k\) events for a partition of the
sample space \(\Omega\), i.e.~they are disjoint and
\(\cup_{i=1}^k A_i = \Omega\). Assume that \(\mathbb{P}(B > 0)\). Prove
that if \(\mathbb{P}(A_1 | B) < \mathbb{P}(A_1)\) then
\(\mathbb{P}(A_i | B) > \mathbb{P}(A_i)\) for some \(i = 2, \dots, k\).

\textbf{Solution}.

We have

\[ B = B\Omega = B \left( \cup_{i=1}^k A_i \right) = \cup_{i=1}^k A_i B \]

and so

\[ \mathbb{P}\left( \cup_{i=1}^k A_i B \right) = \mathbb{P}(B) 
\Longleftrightarrow \sum_{i=1}^k \frac{\mathbb{P}(A_i B)}{\mathbb{P}(B)} = 1
\Longleftrightarrow \sum_{i=1}^k \mathbb{P}(A_i | B) = 1
\Longleftrightarrow \sum_{i=1}^k \mathbb{P}(A_i | B) = \sum_{i=1}^k \mathbb{P}(A_i)
\]

If we assume that \(\mathbb{P}(A_i | B) \leq \mathbb{P}(A_i)\) for all
\(i\) and \(\mathbb{P}(A_1 | B) < \mathbb{P}(A_1)\), then we must have
\(\sum_{i=1}^k \mathbb{P}(A_i | B) < \sum_{i=1}^k \mathbb{P}(A_i)\), a
contradiction. Therefore the desired statement must hold.

\textbf{Exercise 2.10.18}. Suppose that 30\% of computer users use a
Macintosh, 50\% use Windows and 20\% use Linux. Suppose that 65\% of the
Mac users have succumbed to a computer virus, 82\% of the Windows users
get the virus and 50\% of the Linux users get the virus. We select a
person at random and learn that her system was infected with the virus.
What is the probability that she is a Windows user?

\textbf{Solution}. The event space can be described as:

\begin{longtable}[]{@{}ll@{}}
\toprule\noalign{}
outcome & probability \\
\midrule\noalign{}
\endhead
\bottomrule\noalign{}
\endlastfoot
Mac, no virus & 30\% * 35\% \\
Mac, virus & 30\% * 65\% \\
Windows, no virus & 50\% * 18\% \\
Windows, virus & 50\% * 82\% \\
Linux, no virus & 20\% * 50\% \\
Linux, virus & 20\% * 50\% \\
\end{longtable}

The desired conditional probability is

\[ \mathbb{P}(\text{Windows} | \text{virus}) = \frac{\mathbb{P}(\text{Windows}, \text{virus})}{\mathbb{P}(\text{virus})}
= \frac{0.50 \cdot 0.82}{0.30 \cdot 0.65 + 0.50 \cdot 0.82 + 0.20 \cdot 0.50} \approx 0.5816
\]

\textbf{Exercise 2.10.19}. A box contains 5 coins and each has a
different probability of showing heads. Let \(p_1, \dots, p_5\) denote
the probability of heads on each coin. Suppose that

\[ p_1 = 0, \quad p_2 = 1/4, \quad p_3 = 1/2, \quad p_4 = 3/4, \quad \text{and } p_5 = 1\]

Let \(H\) denote ``heads is obtained'' and let \(C_i\) denote the event
that coin \(i\) is selected.

\textbf{(a)} Select a coin at random and toss it. Suppose a head is
obtained. What is the posterior probability that coin \(i\) was selected
(\(i = 1, \dots, 5\))? In other words, find \(\mathbb{P}(C_i | H)\) for
\(i = 1, \dots, 5\).

\textbf{(b)} Toss the coin again. What is the probability of another
head? In other words find \(\mathbb{P}(H_2 | H_1)\) where \(H_j\) means
``heads on toss \(j\)''.

\textbf{(c)} Find \(\mathbb{P}(C_i | B_4)\) where \(B_4\) means ``first
head is obtained on toss 4''.

\textbf{Solution}.

\textbf{(a)} We have:

\[ \mathbb{P}(C_i | H) = \frac{\mathbb{P}(C_i H)}{\mathbb{P}(H)} = \frac{\mathbb{P}(C_i H)}{\sum_j \mathbb{P}(C_j H)} = \frac{\mathbb{P}(C_i)\mathbb{P}(H | C_i)}{ \sum_j \mathbb{P}(C_j)\mathbb{P}(H | C_j)}\]

Assuming that the coin selection is uniformly random,
\(\mathbb{P}(C_i) = 1/5\) for \(i = 1, \dots, 5\), and the above
simplifies to

\[ \frac{\mathbb{P}(H | C_i)}{ \sum_j \mathbb{P}(H | C_j)} = \frac{p_i}{\sum_j p_j}\]

Therefore,

\[ \mathbb{P}(C_1 | H) = 0
\quad
\mathbb{P}(C_2 | H) = 1/10
\quad
\mathbb{P}(C_3 | H) = 1/5
\quad
\mathbb{P}(C_4 | H) = 3/10
\quad
\mathbb{P}(C_5 | H) = 2/5
\]

\textbf{(b)} We have:

\[ \mathbb{P}(H_2 | H_1) = \frac{\mathbb{P}(H_2 H_1) }{\mathbb{P}(H_1) } = \frac{\sum_j \mathbb{P}(C_j H_1 H_2)}{\sum_j \mathbb{P}(C_j H_1)} = \frac{\sum_j (1/5) p_j^2}{\sum_j (1/5) p_j} = \frac{3}{16} \]

\textbf{(c)} We have:

\[ \mathbb{P}(C_i | B_4) = \frac{\mathbb{P}(C_i B_4)}{\mathbb{P}(B_4)}= \frac{\mathbb{P}(C_i B_4)}{\sum_j \mathbb{P}(C_j B_4)}\]

But \(\mathbb{P}(C_i B_4) = (1/5) (1 - p_i)^3 p_i\) -- selecting coin
\(i\), then obtaining 3 tails followed by a head on that coin -- so

\[
\mathbb{P}(C_1 | B_4) = 0
\quad
\mathbb{P}(C_2 | B_4) = \frac{27}{46}
\quad
\mathbb{P}(C_3 | B_4) = \frac{8}{23}
\quad
\mathbb{P}(C_4 | B_4) = \frac{3}{46}
\quad
\mathbb{P}(C_5 | B_4) = 0
\]

\textbf{Exercise 2.10.20 (Computer Experiment)}. Suppose a coin has
probability \(p\) of falling heads. If we flip the coin many times, we
would expect the proportion of heads to be near \(p\). We will make this
formal later. Take \(p = .3\) and \(n = 1000\) and simulate \(n\) coin
flips. Plot the proportion of heads as a function of \(n\). Repeat for
\(p = .03\).

\begin{python}
import numpy as np

np.random.seed(0)

n = 1000
X1 = np.where(np.random.uniform(low=0, high=1, size=n) < 0.3, 1, 0) 
X2 = np.where(np.random.uniform(low=0, high=1, size=n) < 0.03, 1, 0) 
\end{python}

\begin{python}
import matplotlib.pyplot as plt

nn = np.arange(1, n + 1)

plt.figure(figsize=(12, 8))
plt.plot(nn, np.cumsum(X1) / nn, label='p = 0.3')
plt.plot(nn, np.cumsum(X2) / nn, label='p = 0.03')
plt.legend(loc='upper right')
plt.show()
\end{python}

\begin{figure}[H]
\includegraphics[width=0.9\linewidth,height=0.2\textheight,keepaspectratio]{Figure-02-01}
\end{figure}

\textbf{Exercise 2.10.21 (Computer Experiment)}. Suppose we flip a coin
\(n\) times and let \(p\) denote the probability of heads. Let \(X\) be
the number of heads. We call \(X\) a binomial random variable which is
discussed in the next chapter. Intuition suggests that \(X\) will be
close to \(np\). To see if this is true, we can repeat this experiment
many times and average the \(X\) values. Carry out a simulation and
compare the averages of the \(X\)'s to \(np\). Try this for \(p = .3\)
and \(n = 10, 100, 1000\).

\begin{python}
import numpy as np
from tqdm.notebook import tqdm

B = 50000
p = 0.3

np.random.seed(0)

X1 = np.empty(B)
X2 = np.empty(B)
X3 = np.empty(B)
for i in tqdm(range(B)):
    x1 = np.where(np.random.uniform(low=0,high=1,size=10)<p,1,0)
    x2 = np.where(np.random.uniform(low=0,high=1,size=100)<p,1,0)
    x3 = np.where(np.random.uniform(low=0,high=1,size=1000)<p,1,0)
    X1[i] = np.sum(x1)
    X2[i] = np.sum(x2)
    X3[i] = np.sum(x3)
\end{python}

\begin{python}
print('X1 mean: %.3f' % X1.mean())
print('X1 np:   %.3f' % (0.3 * 10))
print()
print('X2 mean: %.3f' % X2.mean())
print('X2 np:   %.3f' % (0.3 * 100))
print()
print('X3 mean: %.3f' % X3.mean())
print('X3 np:   %.3f' % (0.3 * 1000))
\end{python}

\begin{console}
X1 mean: 3.010
X1 np:   3.000

X2 mean: 30.013
X2 np:   30.000

X3 mean: 300.104
X3 np:   300.000
\end{console}

\begin{python}
import matplotlib.pyplot as plt

plt.figure(figsize=(12, 8))

ax = plt.subplot(3, 1, 1)
ax.hist(X1, density=True, bins=100, label='histogram', color='C0')
ax.vlines(X1.mean(), ymin=0, ymax=5, label=r'$\overline{X}$', color='C1')
ax.vlines(0.3 * 10, ymin=0, ymax=5, label=r'$np$', color='C2')
ax.legend(loc='upper right')
ax.set_title('n = 10')

ax = plt.subplot(3, 1, 2)
ax.hist(X2, density=True, bins=100, label='histogram', color='C0')
ax.vlines(X2.mean(), ymin=0, ymax=0.5, label=r'$\overline{X}$', color='C1')
ax.vlines(0.3 * 100, ymin=0, ymax=0.5, label=r'$np$', color='C2')
ax.legend(loc='upper right')
ax.set_title('n = 100')

ax = plt.subplot(3, 1, 3)
ax.hist(X3, density=True, bins=100, label='histogram', color='C0')
ax.vlines(X3.mean(), ymin=0, ymax=0.05, label=r'$\overline{X}$', color='C1')
ax.vlines(0.3 * 1000, ymin=0, ymax=0.05, label=r'$np$', color='C2')
ax.legend(loc='upper right')
ax.set_title('n = 1000')

plt.tight_layout()
plt.show()
\end{python}

\begin{figure}[H]
\includegraphics[width=0.9\linewidth,height=0.2\textheight,keepaspectratio]{Figure-02-02}
\end{figure}

\textbf{Exercise 2.10.22 (Computer Experiment)}. Here we will get some
experience simulating conditional probabilities. Consider tossing a fair
die. Let \(A = \{2, 4, 6\}\) and \(B = \{1, 2, 3, 4\}\). Then
\(\mathbb{P}(A) = 1/2\), \(\mathbb{P}(B) = 2/3\) and
\(\mathbb{P}(AB) = 1/3\). Since
\(\mathbb{P}(AB) = \mathbb{P}(A) \mathbb{P}(B)\), the events \(A\) and
\(B\) are independent. Simulate draws from the sample space and verify
that \(\hat{P}(AB) = \hat{P}(A) \hat{P}(B)\) where \(\hat{P}\) is the
proportion of times an event occurred in the simulation. Now find two
events \(A\) and \(B\) that are not independent. Compute \(\hat{P}(A)\),
\(\hat{P}(B)\) and \(\hat{P}(AB)\). Compare the calculated values to
their theoretical values. Report your results and interpret.

\begin{python}
import numpy as np

np.random.seed(0)

B = 10000
results = np.random.randint(low=1, high=7, size=B)

A_hat = np.isin(results, [2, 4, 6])
B_hat = np.isin(results, [1, 2, 3, 4])
AB_hat = np.isin(results, [2, 4])
\end{python}

\begin{python}
import matplotlib.pyplot as plt

nn = np.arange(1, B + 1)


f, (a0, a1) = plt.subplots(2, 1, gridspec_kw={'height_ratios': [3, 1]}, figsize=(12, 8))

a0.plot(nn, np.cumsum(A_hat) / nn, label=r'$\hat{P}(A)$')
a0.plot(nn, np.cumsum(B_hat) / nn, label=r'$\hat{P}(B)$')
a0.plot(nn, np.cumsum(AB_hat) / nn, label=r'$\hat{P}(AB)$')
a0.plot(nn, np.cumsum(A_hat) * np.cumsum(B_hat) / (nn * nn), label=r'$\hat{P}(A) \hat{P}(B)$')
a0.legend(loc='upper right')

a1.plot(nn, np.cumsum(A_hat) * np.cumsum(B_hat) / (nn * nn) - np.cumsum(AB_hat) / nn, 
         label=r'$\hat{P}(A) \hat{P}(B)- \hat{P}(AB)$', color='purple')
a1.legend(loc='upper right')

plt.tight_layout()
plt.show()
\end{python}

\begin{figure}[H]
\includegraphics[width=0.9\linewidth,height=0.2\textheight,keepaspectratio]{Figure-02-03}
\end{figure}

For our own choice of non-independent events, let \(A = \{ 2, 4, 6\}\)
and \(B = \{2, 4, 5\}\). Then \(\mathbb{P}(A) = \mathbb{P}(B) = 1/2\)
but \(\mathbb{P}(AB) = 1/3\).

\begin{python}
A_hat = np.isin(results, [2, 4, 6])
B_hat = np.isin(results, [2, 4, 5])
AB_hat = np.isin(results, [2, 4])
\end{python}

\begin{python}
import matplotlib.pyplot as plt

nn = np.arange(1, B + 1)

f, (a0, a1) = plt.subplots(2, 1, gridspec_kw={'height_ratios': [3, 1]}, figsize=(12, 8))

a0.plot(nn, np.cumsum(A_hat) / nn, label=r'$\hat{P}(A)$')
a0.plot(nn, np.cumsum(B_hat) / nn, label=r'$\hat{P}(B)$')
a0.plot(nn, np.cumsum(AB_hat) / nn, label=r'$\hat{P}(AB)$')
a0.plot(nn, np.cumsum(A_hat) * np.cumsum(B_hat) / (nn * nn), label=r'$\hat{P}(A) \hat{P}(B)$')
a0.legend(loc='upper right')

a1.plot(nn, np.cumsum(A_hat) * np.cumsum(B_hat) / (nn * nn) - np.cumsum(AB_hat) / nn, 
         label=r'$\hat{P}(A) \hat{P}(B)- \hat{P}(AB)$', color='purple')
a1.legend(loc='upper right')

plt.tight_layout()
plt.show()
\end{python}

\begin{figure}[H]
\includegraphics[width=0.9\linewidth,height=0.2\textheight,keepaspectratio]{Figure-02-04}
\end{figure}

As noted, the estimates converges to the theoretical value -- and the
product of the estimates only converge to the estimate of the joint
event in the scenario where the events are independent.

